% © Copyright IBM Corporation 2001, 2021
% © Copyright Linux Foundation 2002
%
% SPDX-License-Identifier: GFDL-1.1-no-invariants-only
%
% s390/s390x ABI specification -- LaTeX source
%
% Permission is granted to copy, distribute and/or modify this document
% under the terms of the GNU Free Documentation License, Version 1.1; with
% no Invariant Sections, with no Front-Cover Texts, and with no Back-Cover
% Texts.

\documentclass[english,11pt,twoside,toc=bib,toc=idx]{scrreprt}
\usepackage{scrhack}

% --- Version ---

\newcommand{\Version}{1.x}

% --- Packages ---

\usepackage{babel}
\usepackage{graphicx}
\usepackage{array}
\usepackage{tabularx}
\usepackage{multirow}
\usepackage{threeparttable}
\usepackage{longtable}
\usepackage{booktabs}
\usepackage{makeidx}

% --- Indexing ---

\makeindex

% --- Fonts ---

\ifdefined\ifHtml\else

\usepackage{unicode-math}
\usepackage{fontspec}

\usepackage{microtype}

% Use the IBM Plex font family.

\setmainfont{IBMPlexSerif-Regular.otf}[
  BoldFont = IBMPlexSerif-Bold.otf,
  ItalicFont = IBMPlexSerif-Italic.otf,
  BoldItalicFont = IBMPlexSerif-BoldItalic.otf]
\setsansfont{IBM Plex Sans}
\setmonofont{IBM Plex Mono}

\setkomafont{captionlabel}{%
  \usekomafont{descriptionlabel}%
}

% There's no math font for Plex yet, so choose a font with similar
% proportions and scale it to match the base font's uppercase letters.
% Once a math font for Plex exists (check https://github.com/IBM/plex), it
% should be activated here instead.

\setmathfont[Scale=MatchUppercase]{TeX Gyre DejaVu Math}

\fi

% --- Layout definitions ---

\KOMAoptions{paper=letter}
\KOMAoptions{DIV=13,BCOR=1cm}

\setlength{\parskip}{\smallskipamount}
\setlength{\parindent}{0pt}
\setcounter{secnumdepth}{3}
\setcounter{tocdepth}{3}
\DeclareTOCStyleEntry[numwidth=2em]{default}{chapter}
\DeclareTOCStyleEntry[indent=2em,numwidth=3em]{default}{section}
\DeclareTOCStyleEntry[indent=5em,numwidth=3.5em]{default}{subsection}
\DeclareTOCStyleEntry[indent=8.5em,numwidth=4em]{default}{subsubsection}
\DeclareTOCStyleEntry[numwidth=3em]{default}{figure}
\DeclareTOCStyleEntry[numwidth=3em]{default}{table}

% --- Custom commands ---

\newcommand{\jumplabel}[1]{\textsf{‹#1›}}
\newcommand{\stackit}[2][l]{\setlength{\tabcolsep}{0mm}\begin{tabular}{#1}%
    #2\end{tabular}}

\ifzseries
\newcommand{\NBITS}{64}
\newcommand{\ADDRBITS}{64}
\newcommand{\NBYTES}{8}
\newcommand{\STACKSIZE}{160}
\newcommand{\ABINAME}{s390x}
\newcommand{\ARCH}{z/\kern-1pt Ar\-chi\-tec\-ture}
\newcommand{\ARCHarch}{\ARCH}
\newcommand{\aARCH}{a \ARCH}
\newcommand{\theARCH}{the \ARCH}
\else
\newcommand{\NBITS}{32}
\newcommand{\ADDRBITS}{31}
\newcommand{\NBYTES}{4}
\newcommand{\STACKSIZE}{96}
\newcommand{\ABINAME}{s390}
\newcommand{\ARCH}{ESA/390}
\newcommand{\ARCHarch}{the \ARCH{} ar\-chi\-tec\-ture}
\newcommand{\aARCH}{an \ARCH}
\newcommand{\theARCH}{\ARCHarch}
\fi

\newenvironment{DIFnomarkup}{}{} % For latexdiff

% --- Listings ---

\usepackage{xcolor}
\definecolor{lstnumbers}{rgb}{0.5,0.5,0.5}

\usepackage{listings}
\lstset{%
  basicstyle={\ttfamily\small},
  numberstyle={\sffamily\footnotesize\color{lstnumbers}},
  showstringspaces=false,
  captionpos=b,
  abovecaptionskip=2ex,
  language=C}
\lstdefinestyle{float}{%
  frame=single,
  float=tbp}
\lstdefinestyle{embed}{%
  frame=single}
\lstdefinestyle{long}{%
  frame=single,
  captionpos=t}
\lstdefinestyle{short}{%
  aboveskip=-1.2ex,
  belowskip=-\baselineskip}
\lstdefinelanguage{simpleasm}{%
  comment=[l]\#,
  string=[b]{"}}

% --- Hyperref ---

\usepackage[unicode=true,pdfusetitle,
bookmarks=true,bookmarksnumbered=true,bookmarksopen=true,bookmarksopenlevel=1,
breaklinks=true,pdfborder={0 0 0},backref=false,colorlinks=true]
{hyperref}
\hypersetup{linkcolor=black,pdfstartview={XYZ 0 0 1}}

\usepackage{cleveref}

% Fixed definition of \refstepcounter@noarg from cleveref.sty.  This
% version fixes an endless loop with tex4ht when used together with
% listings and hyperref.

\makeatletter
\def\refstepcounter@noarg#1{%
  \cref@old@refstepcounter{#1}%
  \cref@constructprefix{#1}{\cref@result}%
  \@ifundefined{cref@#1@alias}%
    {\def\@tempa{#1}}%
    {\def\@tempa{\csname cref@#1@alias\endcsname}}%
  \protected@xdef\cref@currentlabel{%   <-- xdef instead of edef
    [\@tempa][\arabic{#1}][\cref@result]%
    \csname p@#1\endcsname\csname the#1\endcsname}}%
\makeatother

% Call out figures with "figure" instead of "fig.".

\crefname{figure}{figure}{figures}
\Crefname{figure}{Figure}{Figures}

% --- TikZ support ---

\newif\ifSkipTikZ
\ifdefined\ifHtml
\SkipTikZtrue
\else
\SkipTikZfalse
\fi

\ifSkipTikZ\else          % BEGIN skip TikZ

\usepackage{tikz}
\usetikzlibrary{arrows}
\usetikzlibrary{decorations.pathreplacing}
\usetikzlibrary{fit}
\usetikzlibrary{patterns}
\usetikzlibrary{positioning}
\usetikzlibrary{shadows}
\usetikzlibrary{shapes}

% --- PGF layer definitions ---

\pgfdeclarelayer{background}

% --- Misc PGF/TikZ style definitions ---

\tikzset{memory layout/.style={fill=yellow!10,draw}}
\tikzset{inactive layout/.style={text=black!60}}

% --- Style definitions for bit layout charts ---

\tikzset{bitnum/.style={text=black!60, font={\footnotesize\sffamily}}}
\tikzset{bytenum/.style={font={\footnotesize}}}
\tikzset{bitchart box/.style={fill=blue!7, draw}}
\tikzset{bitfield label/.style={}}
\tikzset{bitfield padding/.style={draw=black, very thick,
    line cap=round, shorten <=0.5ex, shorten >=0.5ex,
    dash pattern=on 0pt off 6pt, dash phase=3pt}}
\tikzset{>=latex}

% --- Helpers for bit layout charts ---

\newcommand{\bitchartsep} {
  \draw (0, 0) -- (last SE);
}

\makeatletter

\def\one@bitchartfield#1#2{%
  \pgfmathsetmacro{\xlow}{\xprev-\xfirst}
  \pgfmathsetmacro{\xhigh}{#1-\xfirst}
  \pgfmathtruncatemacro{\xhighminus}{#1-1}
  \path (\xlow, 0) node [bitnum, above right] {\xprev};
  \ifx\\#2\\
  \path [bitfield padding] (\xlow, 0.5) -- (\xhigh, 0.5);
  \else
  \path (\xlow, 0) -- node [bitfield label,pos=0.5]
  {\texttt{#2}} (\xhigh, 1);
  \fi
  \path (\xhigh, 0) coordinate (last SE) {}
  node [bitnum, above left] {\xhighminus};
  \edef\xprev{#1}}

\def\bitchartfield@n#1/#2/#3{%
  \one@bitchartfield{#1}{#2}
  \draw (\xlow, 0) -- (\xlow, 1);
  \if,#3\let\next\bitchartfield@n\else\let\next\egroup\fi\next}

\def\bitchartfields#1 #2/#3/#4{%
  \bgroup
  \edef\xfirst{#1}
  \let\xprev\xfirst
  \one@bitchartfield{#2}{#3}
  \if,#4\let\next\bitchartfield@n\else\let\next\egroup\fi\next}

\def\one@bytechartfield#1#2{%
  \pgfmathsetmacro{\xlow}{8*(\prevbyte-\firstbyte)}
  \pgfmathsetmacro{\xhigh}{8*(#1-\firstbyte)}
  \path (\xlow, 1) node [bytenum, below right] (bnum) {\prevbyte};
  \ifx\\#2\\
  \path [bitfield padding] (\xhigh, 0.5) coordinate (m){} -- (bnum|-m);
  \else
  \path (\xlow, 0) -- node [bitfield label,pos=0.5]
  {\texttt{#2}} (\xhigh, 1);
  \fi
  \path (\xhigh, 0) coordinate (last SE) {};
  \edef\prevbyte{#1}}

\def\bytechartfield@n#1/#2/#3{%
  \one@bytechartfield{#1}{#2}
  \draw (\xlow, 1) -- (\xlow, 0);
  \if,#3\let\next\bytechartfield@n\else\let\next\egroup\fi\next}

\def\bytechartfields#1 #2/#3/#4{%
  \bgroup
  \edef\firstbyte{#1}
  \edef\prevbyte{#1}
  \one@bytechartfield{#2}{#3}
  \if,#4\let\next\bytechartfield@n\else\let\next\egroup\fi\next}

\newcommand{\bitchartbytes}[2]{%
  \bgroup
  \edef\first@byte{#1}
  \foreach \b in {#2} {%
    \pgfmathsetmacro{\bit@}{8*(\b-#1)}
    \path (\bit@, 1) node [bytenum, below right] {\b};
    \ifx\b\first@byte\else
    \path (\bit@, 1) -- (\bit@, 0);
    \fi
  }
  \egroup
}

\makeatother

\fi                             % END skip TikZ

% Hyphenation rules for compound words.

\hyphenation{big=endi-an}
\hyphenation{cia=rel-a-tive}
\hyphenation{float-ing=point}
\hyphenation{func-tion=call-ing}
\hyphenation{lit-tle=endi-an}
\hyphenation{non=ze-ro}
\hyphenation{pa-ram-e-ter=pass-ing}
\hyphenation{po-si-tion=inde-pen-dence}
\hyphenation{po-si-tion=inde-pen-dent}
\hyphenation{sig-nal=han-dling}
\hyphenation{sign=ex-tend-ed}
\hyphenation{sys-tem=sup-plied}
\hyphenation{two=el-e-ment}
\hyphenation{ze-ro=ex-tend-ed}

% ------------------------------------------------------------
\begin{document}

\newcommand{\myTitle}{ELF Application Binary Interface \ABINAME{}~Supplement}

\begin{DIFnomarkup}
\title{\myTitle}
\subtitle{Version \Version}

\author{Martin Schwidefsky\and Ulrich Weigand\and Andreas Arnez%
  \and Andreas Krebbel}
\publishers{IBM\textregistered{} Corporation}

\lowertitleback{%
  \noindent \textbf{\myTitle}

  \noindent Version \Version

  \noindent © Copyright IBM Corporation 2001, 2021

  \noindent © Copyright Linux Foundation 2002

  \medskip
  \noindent Permission is granted to copy, distribute and/or modify
  this document under the terms of the GNU Free Documentation License,
  Version 1.1; with no Invariant Sections, with no Front-Cover Texts,
  and with no Back-Cover Texts.  A copy of the license is included in
  the section entitled ``GNU Free Documentation License.''}

\maketitle
\end{DIFnomarkup}

\tableofcontents
\listoffigures
\listoftables
\lstlistoflistings

\chapter*{About This Book}

The \ABINAME{} supplement to the Executable and Linkage
Format Application Binary Interface (or ELF ABI) defines a system
interface for compiled application programs.  Its purpose is to
establish a standard binary interface for application programs on
Linux\textregistered{} for \ARCH{}\textregistered{} systems.

This book is a supplement to the generic ``System V Application Binary
Interface'' and should be read in conjunction with it.

\section*{History}

\begin{description}
\item[1.x] Add information about vector types and registers, based on
  input from Andreas Krebbel.
\item[1.5] \emph{``ELF Application Binary Interface
    \ABINAME{}~Supplement''} -- Conversion to \LaTeX{}; various
  corrections to revision~1.02.  Edited by Andreas Arnez.  Published at
  \url{https://github.com/ibm/s390x-abi}, January~2021.
\item[1.02] \emph{``{\ifzseries zSeries\else S/390\fi} ELF Application
    Binary Interface Supplement''} -- Revised edition.  Published under
  the GNU Free Documentation License~1.1 by The Linux Foundation as a
  ``referenced specification'' at \url{http://refspecs.linuxbase.org/},
  November 2002.
\item[1.0] \emph{``LINUX for {\ifzseries zSeries\else S/390\fi}: ELF
    Application Binary Interface Supplement''} -- First edition.
  Published by IBM as LNUX-1107-{\ifzseries 01\else 02\fi}, March 2001.
\end{description}

\chapter{Low-Level System Information}
\section{Machine Interface}
This section describes the processor-specific information for
\ARCH{} processors.

\subsection{Processor Architecture}
\index{processor architecture}
\index{instruction set}
{\ifzseries
  \cite{sa22} (SA22-7832) defines \theARCH{}.
  \else
  \cite{sa22} (SA22-7201) defines \theARCH{}.
  \fi}

Programs intended to execute directly on the processor use the
\ARCH{} instruction set and the
instruction encoding and semantics of the architecture.

An application program can assume that all instructions defined by the
architecture and that are neither privileged nor optional, exist and work
as documented.

To be ABI conforming, the processor must implement the instructions of
the architecture, perform the specified operations, and produce the
expected results.  The ABI neither places performance constraints on
systems nor specifies what instructions must be implemented in
hardware.  A software emulation of the architecture can conform to
the ABI\@.

{\ifzseries\else
  There are some instructions in \ARCHarch{}
  which are described as ``optional.''  This ABI
  requires some of these to be available; in particular:
  \begin{itemize}
  \item additional floating-point facilities, % checksum instruction
  \item compare and move extended,
  \item immediate and relative instructions,
  \item square root,
  \item string instructions.
  \end{itemize}

  The ABI guarantees that these instructions are present.  In order to
  comply with the ABI the operating system must emulate these
  instructions on machines which do not support them in the hardware.
  Other instructions are not available in some current models; programs
  using these instructions do not conform to the \ABINAME{} ABI and
  executing them on machines without the extra capabilities will result
  in undefined behavior.  \fi}

In \ARCHarch{} a
processor runs in big-endian mode.  (See \cref{byteordering}.)

\subsection{Data Representation}
\subsubsection{Byte Ordering}
\label{byteordering}
\index{byte ordering}

The architecture defines an 8-bit byte\index{byte},
a 16-bit halfword\index{halfword},
a 32-bit word\index{word},{\ifzseries\else{} and\fi}
a 64-bit doubleword\index{doubleword}{\ifzseries ,
  and a 128-bit quadword\index{quadword}\fi}.
Byte ordering defines how the bytes that make up halfwords,
words,{\ifzseries\else{} and\fi} doublewords{\ifzseries , and
  quadwords\fi} are ordered in memory.  Most significant byte (MSB)
ordering, also called ``big-endian,''\index{big-endian} means that
the most significant byte
of a structure is located in the lowest addressed byte position in a
storage unit (byte 0).  By contrast, least significant byte (LSB)
ordering, or ``little-endian,''\index{little-endian} refers to the
reverse byte order, where
the lowest addressed byte position holds the least significant byte.

\Crefrange{fig:halfword}{\ifzseries fig:quadwords\else
  fig:doublewords\fi} illustrate the conventions for bit and byte
numbering within storage units of various widths.  These conventions
apply to both integer data and floating-point data, where the most
significant byte of a floating-point value holds the sign and the
exponent (or at least the start of the exponent).  The figures show
big-endian byte numbers in the upper left corners and bit numbers in
the lower corners.

\begin{figure}
  \centering
  \ifSkipTikZ
\begin{verbatim}
+-------------+-------------+
| 0           | 1           |
|     msb     |     lsb     |
| 0         7 | 8        15 |
+-------------+-------------+
\end{verbatim}
  \else
  \begin{tikzpicture}[x=1.3ex,y=3em]
    \path [bitchart box] (0, 0) rectangle (16, 1);
    \bitchartfields 0 8/\textrm{msb}/, 16/\textrm{lsb}/;
    \bitchartbytes{0}{0,1}
  \end{tikzpicture}
  \fi
  \caption{Bit and byte numbering in halfwords}
  \label{fig:halfword}
\end{figure}

\begin{figure}
  \centering
  \ifSkipTikZ
\begin{verbatim}
+-------------+-------------+-------------+-------------+
| 0           | 1           | 2           | 3           |
|     msb     |             |             |     lsb     |
| 0         7 | 8        15 | 16       23 | 24       31 |
+-------------+-------------+-------------+-------------+
\end{verbatim}
  \else
  \begin{tikzpicture}[x=1.3ex,y=3em]
    \path [bitchart box] (0, 0) rectangle (32, 1);
    \bitchartfields 0 8/\textrm{msb}/, 16/~/, 24/~/, 32/\textrm{lsb}/;
    \bitchartbytes{0}{0,...,3}
  \end{tikzpicture}
  \fi
  \caption{Bit and byte numbering in words}
  \label{fig:words}
\end{figure}

\begin{figure}
  \centering
  \ifSkipTikZ
\begin{verbatim}
+-------------+-------------+-------------+-------------+
| 0           | 1           | 2           | 3           |
|     msb     |             |             |             |
| 0         7 | 8        15 | 16       23 | 24       31 |
+-------------+-------------+-------------+-------------+
| 4           | 5           | 6           | 7           |
|             |             |             |     lsb     |
| 32       39 | 40       47 | 48       55 | 56       63 |
+-------------+-------------+-------------+-------------+
\end{verbatim}
  \else
  \begin{tikzpicture}[x=1.3ex,y=3em]
    \path [bitchart box] (0, -1) rectangle (32, 1);
    \bitchartfields 0 8/\textrm{msb}/, 16/~/, 24/~/, 32/~/;
    \bitchartbytes{0}{0,...,3}
    \bitchartsep
    \begin{scope}[shift={(0,-1)}]
      \bitchartfields 32 40/~/, 48/~/, 56/~/, 64/\textrm{lsb}/;
      \bitchartbytes{4}{4,...,7}
    \end{scope}
  \end{tikzpicture}
  \fi
  \caption{Bit and byte numbering in doublewords}
  \label{fig:doublewords}
\end{figure}

\ifzseries
\begin{figure}
  \centering
  \ifSkipTikZ
\begin{verbatim}
+-------------+-------------+-------------+-------------+
| 0           | 1           | 2           | 3           |
|     msb     |             |             |             |
| 0         7 | 8        15 | 16       23 | 24       31 |
+-------------+-------------+-------------+-------------+
| 4           | 5           | 6           | 7           |
|             |             |             |             |
| 32       39 | 40       47 | 48       55 | 56       63 |
+-------------+-------------+-------------+-------------+
| 8           | 9           | 10          | 11          |
|             |             |             |             |
| 64       71 | 72       79 | 80       87 | 88       95 |
+-------------+-------------+-------------+-------------+
| 12          | 13          | 14          | 15          |
|             |             |             |     lsb     |
| 96      103 | 104     111 | 112     119 | 120     127 |
+-------------+-------------+-------------+-------------+
\end{verbatim}
  \else
  \begin{tikzpicture}[x=1.3ex,y=3.2em]
    \path [bitchart box] (0, -3) rectangle (32, 1);
    \bitchartfields 0 8/\textrm{msb}/, 16/~/, 24/~/, 32/~/;
    \bitchartbytes{0}{0,...,3}
    \bitchartsep
    \begin{scope}[shift={(0,-1)}]
      \bitchartfields 32 40/~/, 48/~/, 56/~/, 64/~/;
      \bitchartbytes{4}{4,...,7}
      \bitchartsep
    \end{scope}
    \begin{scope}[shift={(0,-2)}]
      \bitchartfields 64 72/~/, 80/~/, 88/~/, 96/~/;
      \bitchartbytes{8}{8,...,11}
      \bitchartsep
    \end{scope}
    \begin{scope}[shift={(0,-3)}]
      \bitchartfields 96 104/~/, 112/~/, 120/~/, 128/\textrm{lsb}/;
      \bitchartbytes{12}{12,...,15}
    \end{scope}
  \end{tikzpicture}
  \fi
  \caption{Bit and byte numbering in quadwords}
  \label{fig:quadwords}
\end{figure}
\fi

\subsubsection{Fundamental Types}
\index{type!scalar}
\Cref{tab:scalar} shows how ISO C scalar types correspond to those of
\aARCH{} processor.  To comply with this ABI, objects stored in memory
must be aligned\index{alignment!scalar} as indicated, even though the
architecture permits unaligned storage operands for most instructions.

For all types, a null pointer\index{null pointer} has the value zero (binary).

A Boolean\index{Boolean} object is represented in memory as a single byte
with a value of 0 or 1.  If a byte with any other value is evaluated as a
Boolean, the behavior is undefined.

For each binary floating-point type, there is a corresponding complex
type\index{complex type}.  It is represented as a two-element array with
the real part as its first and the imaginary part as its second element.

Some C dialects permit enumeration constants that exceed the range of an
\texttt{int}.  Then the enumeration type\index{enumeration type} shall be
encoded as the smallest unsigned or signed C integer type that can
represent all of its enumeration constants and is not smaller than
\texttt{int}.

\begin{table}
  \centering
  \begin{DIFnomarkup}
  \begin{threeparttable}
    \begin{tabularx}{\textwidth}{XlrrX}
      \toprule
      \multirow{2}{*}{Type}
      & \multirow{2}{*}{ISO C}
      & Size in & Align- & \ARCH{} \\
      &
      & bytes   & ment   & type \\
      \midrule
      \multirow{8}{\hsize}{Unsigned integer}
      & \texttt{\_Bool} & 1 & 1
      & \multirow{8}{\hsize}{$n$-bit unsigned binary integer\tnote{\dagger}}
      \\
      & \texttt{unsigned char} & 1 & 1 \\
      & \texttt{char} & 1 & 1 \\
      & \texttt{unsigned short} & 2 & 2 \\
      & \texttt{unsigned int} & 4 & 4 \\
      & \texttt{unsigned long} & \NBYTES{} & \NBYTES{} \\
      & \texttt{unsigned long long} & 8 & 8 \\
      & \texttt{unsigned \_\_int128}\tnote{\dagger\dagger} & 16 & 8 \\
      \midrule
      \multirow{12}{\hsize}{Signed integer}
      & \texttt{signed char} & 1 & 1
      & \multirow{12}{\hsize}{$n$-bit signed binary integer\tnote{\dagger}}
      \\
      & \texttt{signed short} & 2 & 2 \\
      & \texttt{short} & 2 & 2 \\
      & \texttt{signed int} & 4 & 4 \\
      & \texttt{int} & 4 & 4 \\
      & \texttt{enum} & 4 & 4 \\
      & \texttt{signed long} & \NBYTES{} & \NBYTES{} \\
      & \texttt{long} & \NBYTES{} & \NBYTES{} \\
      & \texttt{signed long long} & 8 & 8 \\
      & \texttt{long long} & 8 & 8 \\
      & \texttt{\_\_int128}\tnote{\dagger\dagger} & 16 & 8 \\
      & \texttt{signed \_\_int128}\tnote{\dagger\dagger} & 16 & 8 \\
      \midrule
      \multirow{2}{\hsize}{Pointer}
      & \textit{any-type}\texttt{ *} & \NBYTES{} & \NBYTES{}
      & \multirow{2}{\hsize}{\ADDRBITS{}-bit address} \\
      & \textit{any-type}\texttt{ (*) ()} & \NBYTES{} & \NBYTES{} \\
      \midrule
      \multirow{3}{\hsize}{Binary floating-point} &
      \texttt{float} & 4 & 4 & short BFP \\
      & \texttt{double} & 8 & 8 & long BFP \\
      & \texttt{long double} & 16 & 8 & extended BFP \\
      \midrule
      \multirow{3}{\hsize}{Decimal floating-point} &
      \texttt{\_Decimal32}\tnote{\dagger\dagger} & 4 & 4 & short DFP \\
      & \texttt{\_Decimal64}\tnote{\dagger\dagger} & 8 & 8 & long DFP \\
      & \texttt{\_Decimal128}\tnote{\dagger\dagger} & 16 & 8 & extended DFP \\
      \bottomrule
    \end{tabularx}
    \medskip
    \begin{tablenotes}
    \item [\dagger] Here $n$ denotes the bit size, which equals the byte
      size multiplied by 8.
    \item [\dagger\dagger] These types are an extension to C (ISO/IEC
      9899:2011).
    \end{tablenotes}
  \end{threeparttable}
  \end{DIFnomarkup}
  \caption{Scalar types}
  \label{tab:scalar}
\end{table}

\subsubsection{Aggregates and Unions}
Aggregates\index{aggregate}\index{type!aggregate}
(structures\index{structure} and arrays\index{array}) and
unions\index{union}\index{type!union} assume the
alignment\index{alignment!aggregate or union} of their most strictly
aligned component---that is, the component with the largest alignment.
The size\index{size!aggregate or union} of any object, including
aggregates and unions, is always a multiple of the alignment of the
object.  An array uses the same alignment as its elements.  Structure and
union objects may require padding to meet these size and alignment
constraints:

\begin{itemize}
\item An entire structure or union object is aligned on the same
  boundary as its most strictly aligned member.
\item Each member is assigned to the lowest available offset with the
  appropriate alignment.  This may require internal
  padding\index{padding}, depending on the previous member.
\item If necessary, a structure's size is increased to make it a
  multiple of the structure's alignment.  This may require tail padding
  if the last member does not end on the appropriate boundary.
\end{itemize}

In the examples shown in \crefrange{fig:struct1}{fig:struct5},
member byte offsets (for the big-endian implementation) appear in the
upper left corners.

\begin{figure}
  \centering
  \ifSkipTikZ
\begin{verbatim}
             Byte aligned, sizeof is 1
             +-------------+
struct {     | 0           |
    char c;  |             |
};           |      c      |
             +-------------+
\end{verbatim}
  \else
  \begin{tabular}{>{\texttt\bgroup}l<{\texttt\egroup}}
    struct \{\\
    ~~~~char c;\\
    \};\\
  \end{tabular}
  \quad
  \begin{tikzpicture}[baseline=(mid),x=1.3ex,y=3em]
    \path [bitchart box] (0, 0) rectangle (8, 1);
    \bytechartfields 0 1/c/;
    \path (current bounding box.center) coordinate (mid) {};
    \path (current bounding box.north)
    node [above] {Byte aligned, \texttt{sizeof} is 1};
  \end{tikzpicture}
  \fi
  \caption{Structure smaller than a word}
  \label{fig:struct1}
\end{figure}

\begin{figure}
  \centering
  \ifSkipTikZ
\begin{verbatim}
              Word aligned, sizeof is 8
              +-------------+-------------+---------------------------+
struct {      | 0           | 1           | 2                         |
    char c;   |             |             |                           |
    char d;   |       c     |      d      |             s             |
    short s;  |-------------+-------------+---------------------------|
    int n;    | 4                                                     |
};            |                                                       |
              |                           n                           |
              +-------------------------------------------------------+
\end{verbatim}
  \else
  \begin{tabular}{>{\texttt\bgroup}l<{\texttt\egroup}}
    struct \{\\
    ~~~~char c;\\
    ~~~~char d;\\
    ~~~~short s;\\
    ~~~~{\ifzseries int\else long\fi} n;\\
    \};\\
  \end{tabular}
  \quad
  \begin{tikzpicture}[baseline=(mid),x=1.3ex,y=3em]
    \path [bitchart box] (0, -1) rectangle (32, 1);
    \bytechartfields 0 1/c/, 2/d/, 4/s/;
    \bitchartsep
    \begin{scope}[shift={(0,-1)}]
      \bytechartfields 4 8/n/;
    \end{scope}
    \path (current bounding box.center) coordinate (mid) {};
    \path (current bounding box.north)
    node [above] {Word aligned, \texttt{sizeof} is 8};
  \end{tikzpicture}
  \fi
\caption{No padding}
\label{fig:struct2}
\end{figure}

\begin{figure}
  \centering
  \ifSkipTikZ
\begin{verbatim}
              +-------------+-------------+---------------------------+
struct {      | 0           | 1           | 2                         |
    char c;   |             |             |                           |
    short s;  |      c      |     pad     |             s             |
};            +-------------+-------------+---------------------------+
\end{verbatim}
  \else
  \begin{tabular}{>{\texttt\bgroup}l<{\texttt\egroup}}
    struct \{\\
    ~~~~char c;\\
    ~~~~short s;\\
    \};\\
  \end{tabular}
  \quad
  \begin{tikzpicture}[baseline=(mid),x=1.3ex,y=3em]
    \path [bitchart box] (0, 0) rectangle (32, 1);
    \bytechartfields 0 1/c/, 2//, 4/s/;
    \path (current bounding box.center) coordinate (mid) {};
    \path (current bounding box.north)
    node [above] {Halfword aligned, \texttt{sizeof} is 4};
  \end{tikzpicture}
  \fi
  \caption{Internal padding}
  \label{fig:struct3}
\end{figure}

\begin{figure}
  \centering
  \ifSkipTikZ
\begin{verbatim}
               Doubleword aligned, sizeof is 24
               +-------------+-----------------------------------------+
               | 0           | 1                                       |
               |             |                                         |
               |      c      |                  pad                    |
               |-------------+-----------------------------------------|
               | 4                                                     |
               |                                                       |
               |                          pad                          |
               |-------------------------------------------------------|
struct {       | 8                                                     |
    char c;    |                                                       |
    double d;  |                           d                           |
    short s;   |-------------------------------------------------------|
};             | 12                                                    |
               |                                                       |
               |                           d                           |
               |---------------------------+---------------------------|
               | 16                        | 18                        |
               |                           |                           |
               |             s             |            pad            |
               |---------------------------+---------------------------|
               | 20                                                    |
               |                                                       |
               |                          pad                          |
               +-------------------------------------------------------+
\end{verbatim}
  \else
  \begin{tabular}{>{\texttt\bgroup}l<{\texttt\egroup}}
    struct \{\\
    ~~~~char c;\\
    ~~~~double d;\\
    ~~~~short s;\\
    \};\\
  \end{tabular}
  \quad
  \begin{tikzpicture}[baseline=(mid),x=1.3ex,y=2.5em]
    \path [bitchart box] (0, -5) rectangle (32, 1);
    \bytechartfields 0 1/c/, 4//;
    \bitchartsep
    \begin{scope}[shift={(0,-1)}]
      \bytechartfields 4 8//;
      \bitchartsep
    \end{scope}
    \begin{scope}[shift={(0,-2)}]
      \bytechartfields 8 12/d/;
      \bitchartsep
    \end{scope}
    \begin{scope}[shift={(0,-3)}]
      \bytechartfields 12 16/d/;
      \bitchartsep
    \end{scope}
    \begin{scope}[shift={(0,-4)}]
      \bytechartfields 16 18/s/, 20//;
      \bitchartsep
    \end{scope}
    \begin{scope}[shift={(0,-5)}]
      \bytechartfields 20 24//;
      \bitchartsep
    \end{scope}
    \path (current bounding box.center) coordinate (mid) {};
    \path (current bounding box.north)
    node [above] {Doubleword aligned, \texttt{sizeof} is 24};
  \end{tikzpicture}
  \fi
  \caption{Internal and tail padding}
  \label{fig:struct4}
\end{figure}

\begin{figure}
  \centering
  \ifSkipTikZ
\begin{verbatim}
              Word aligned, sizeof is 4
              +-------------+-----------------------------------------+
              | 0           | 1                                       |
              |             |                                         |
              |      c      |                   pad                   |
              +-------------+-----------------------------------------+
union {       +---------------------------+---------------------------+
    char c;   | 0                         | 2                         |
    short s;  |                           |                           |
    int j;    |             s             |            pad            |
};            +---------------------------+---------------------------+
              +-------------------------------------------------------+
              | 0                                                     |
              |                                                       |
              |                           j                           |
              +-------------------------------------------------------+
\end{verbatim}
  \else
  \begin{tabular}{>{\texttt\bgroup}l<{\texttt\egroup}}
    union \{\\
    ~~~~char c;\\
    ~~~~short s;\\
    ~~~~int j;\\
    \};\\
  \end{tabular}
  \quad
  \begin{tikzpicture}[baseline=(mid),x=1.3ex,y=2.5em]
    \path [bitchart box] (0, 0) rectangle (32, 1);
    \bytechartfields 0 1/c/, 4//;
    \begin{scope}[shift={(0,-1.2)}]
      \path [bitchart box] (0, 0) rectangle (32, 1);
      \bytechartfields 0 2/s/, 4//;
    \end{scope}
    \begin{scope}[shift={(0,-2.4)}]
      \path [bitchart box] (0, 0) rectangle (32, 1);
      \bytechartfields 0 4/j/;
    \end{scope}
    \path (current bounding box.south west) coordinate (sw) {}
    (current bounding box.north west) coordinate (nw) {};
    \path (current bounding box.center) coordinate (mid) {};
    \path (current bounding box.north)
    node [above] {Word aligned, \texttt{sizeof} is 4};
    \path (nw) +(-1em,0) coordinate (nw) {};
    \draw [decorate,decoration=brace, thick] (sw) + (-1em,0) -- (nw);
  \end{tikzpicture}
  \fi
  \caption{Union padding}
  \label{fig:struct5}
\end{figure}

\subsubsection{Bit-Fields}
C struct and union definitions may have ``bit-fields,'' defining
integral objects with a specified number of bits
(see \cref{tab:bitfields}).

\begin{table}
  \centering
  \begin{DIFnomarkup}
  \begin{tabular}[t]{lcr@{~\ldots~}l}
    \toprule
    Bit-field type & Width $n$ & \multicolumn{2}{c}{Range} \\
    \midrule
    \texttt{signed char} & \multirow{3}{*}{1…8} & $-2^{n-1}$ & $2^{n-1}-1$ \\
    \texttt{char} &  & $0$ & $2^n-1$ \\
    \texttt{unsigned char} &  & $0$ & $2^n-1$ \\
    \midrule
    \texttt{signed short} & \multirow{3}{*}{1…16} & -$2^{n-1}$ & $2^{n-1}-1$ \\
    \texttt{short} &  & -$2^{n-1}$ & $2^{n-1}-1$ \\
    \texttt{unsigned short} &  & $0$ & $2^n-1$ \\
    \midrule
    \texttt{signed int} & \multirow{\ifzseries 3\else 6\fi}{*}{1…32} &
    $-2^{n-1}$ & $2^{n-1}-1$ \\
    \texttt{int} &  & $-2^{n-1}$ & $2^{n-1}-1$ \\
    \texttt{unsigned int} &  & $0$ & $2^n-1$ \\
    \ifzseries \midrule \fi
    \texttt{signed long} & {\ifzseries\multirow{3}{*}{1…64}\fi} &
    $-2^{n-1}$ & $2^{n-1}-1$ \\
    \texttt{long} &  & $-2^{n-1}$ & $2^{n-1}-1$ \\
    \texttt{unsigned long} &  & $0$ & 2$^n-1$ \\
    \midrule
    \texttt{signed long long} & \multirow{3}{*}{1…64} & $-2^{n-1}$ & $2^{n-1}-1$ \\
    \texttt{long long} &  & $-2^{n-1}$ & $2^{n-1}-1$ \\
    \texttt{unsigned long long} &  & $0$ & $2^n-1$ \\
    \bottomrule
  \end{tabular}
  \end{DIFnomarkup}
  \caption{Bit-fields}
  \label{tab:bitfields}
\end{table}

Bit-fields\index{bit-field} have the signedness of their underlying type.
For example, a bit-field of type \texttt{long} is signed, whereas a
bit-field of type \texttt{char} is unsigned.

Bit-fields obey the same size and alignment rules as other structure and
union members, with the following additions:

\begin{itemize}
\item Bit-fields are allocated from left to right (most to least
  significant).
\item A bit-field must entirely reside in a storage unit appropriate
  for its declared type.  Thus, a bit-field never crosses its unit
  boundary.
\item Bit-fields must share a storage unit with other structure and
  union members (either bit-field or non-bit-field) if and only if
  there is sufficient space within the storage unit.
\item Unnamed bit-fields' types do not affect the alignment of a
  structure or union, although an individual bit-field's member
  offsets obey the alignment constraints.  An unnamed, zero-width
  bit-field shall prevent any further member, bit-field or other, from
  residing in the storage unit corresponding to the type of the
  zero-width bit-field.
\end{itemize}

The examples in \crefrange{fig:bitnum}{fig:unnbitf} show
structure and union member byte offsets in the upper left corners.
Bit numbers appear in the lower corners.

\begin{figure}
  \centering
  \ifSkipTikZ
\begin{verbatim}
             +-------------+-------------+-------------+-------------+
             | 0           | 1           | 2           | 3           |
0x01020304   |      01     |      02     |      03     |      04     |
             | 0         7 | 8        15 | 16       23 | 24       31 |
             +-------------+-------------+-------------+-------------+
\end{verbatim}
  \else
  \begin{tikzpicture}[x=1.3ex,y=3em]
    \path [bitchart box] (0, 0) rectangle (32, 1);
    \bitchartfields 0 8/01/, 16/02/, 24/03/, 32/04/;
    \bitchartbytes{0}{0,...,3}
    \path (current bounding box.west) node[left=1em]{\texttt{0x01020304}};
  \end{tikzpicture}
  \fi
  \caption{Bit numbering}
  \label{fig:bitnum}
\end{figure}

\begin{figure}
  \centering
  \ifSkipTikZ
\begin{verbatim}
               Word aligned, sizeof is 4
struct {      +--------+---------+------------+-----------------------+
    int j:5;  | 0      |         |            |                       |
    int k:6;  |   j    |    k    |     m      |          pad          |
    int m:7;  | 0    4 | 5    10 | 11      17 | 18                 31 |
};            +--------+---------+------------+-----------------------+
\end{verbatim}
  \else
  \begin{tabular}{>{\texttt\bgroup}l<{\texttt\egroup}}
    struct \{ \\
    ~~~~~int~j:5; \\
    ~~~~~int~k:6; \\
    ~~~~~int~m:7; \\
    \}; \\
  \end{tabular}
  \quad
  \begin{tikzpicture}[baseline=(mid),x=1.3ex,y=3em]
    \path [bitchart box] (0, 0) rectangle (32, 1);
    \bitchartfields 0 5/j/, 11/k/, 18/m/, 32//;
    \bitchartbytes{0}{0}
    \path (current bounding box.center) coordinate (mid) {};
    \path (current bounding box.north)
    node [above] {Word aligned, \texttt{sizeof} is 4};
  \end{tikzpicture}
  \fi
  \caption{Left-to-right allocation}
  \label{fig:lralloc}
\end{figure}

\begin{figure}
  \centering
  \ifSkipTikZ
\begin{verbatim}
                 Word aligned, sizeof is 12
struct {        +--------------+--------------+-----------+-------------+
    short s:9;  | 0            |              |           | 3           |
    int   j:9;  |       s      |      j       |    pad    |      c      |
    char  c;    | 0          8 | 9         17 | 18     23 | 24       31 |
    short t:9;  |--------------+------------+-+-----------++------------|
    short u:9;  | 4            |            | 6            |            |
    char  d;    |       t      |    pad     |       u      |    pad     |
};              | 32        40 | 41      47 | 48        56 | 57      63 |
                |-------------++------------+--------------+------------|
                | 8           | 9                                       |
                |      d      |                    pad                  |
                | 64       71 | 72                                   95 |
                +-------------+-----------------------------------------+
\end{verbatim}
  \else
  \begin{tabular}{>{\texttt\bgroup}l<{\texttt\egroup}}
    struct \{ \\
    ~~~~~short~s:9; \\
    ~~~~~int~~~j:9; \\
    ~~~~~char~~c; \\
    ~~~~~short~t:9; \\
    ~~~~~short~u:9; \\
    ~~~~~char~~d; \\
    \}; \\
  \end{tabular}
  \quad
  \begin{tikzpicture}[baseline=(mid),x=1.3ex,y=3em]
    \path [bitchart box] (0, -2) rectangle (32, 1);
    \bitchartfields 0 9/s/, 18/j/, 24//, 32/c/;
    \bitchartbytes{0}{0,3}
    \bitchartsep
    \begin{scope}[shift={(0,-1)}]
      \bitchartfields 32 41/t/, 48//, 57/u/, 64//;
      \bitchartbytes{4}{4,6}
      \bitchartsep
    \end{scope}
    \begin{scope}[shift={(0,-2)}]
      \bitchartfields  64 72/d/, 96//;
      \bitchartbytes{8}{8,9}
    \end{scope}
    \path (current bounding box.center) coordinate (mid) {};
    \path (current bounding box.north)
    node [above] {Word aligned, \texttt{sizeof} is 12};
  \end{tikzpicture}
  \fi
  \caption{Boundary alignment}
  \label{fig:balign}
\end{figure}

\begin{figure}
  \centering
  \ifSkipTikZ
\begin{verbatim}
                 Halfword aligned, sizeof is 2
struct {        +-------------+-------------+
    char  c;    | 0           | 1           |
    short s:8;  |      c      |      s      |
};              | 0         7 | 8        15 |
                +-------------+-------------+
\end{verbatim}
  \else
  \begin{tabular}{>{\texttt\bgroup}l<{\texttt\egroup}}
    struct \{ \\
    ~~~~~char~~c; \\
    ~~~~~short~s:8; \\
    \}; \\
  \end{tabular}
  \quad
  \begin{tikzpicture}[baseline=(mid),x=1.3ex,y=3em]
    \path [bitchart box] (0, 0) rectangle (16, 1);
    \bitchartfields 0 8/c/, 16/s/;
    \bitchartbytes{0}{0,1}
    \path (current bounding box.center) coordinate (mid) {};
    \path (current bounding box.north)
    node [above] {Halfword aligned, \texttt{sizeof} is 2};
  \end{tikzpicture}
  \fi
\caption{Storage unit sharing}
\label{fig:sushar}
\end{figure}

\begin{figure}
  \centering
  \ifSkipTikZ
\begin{verbatim}
                 Halfword aligned, sizeof is 2
                +-------------+-------------+
                | 0           | 1           |
                |      c      |     pad     |
union {         | 0         7 | 8        15 |
    char  c;    +-------------+-------------+
    short s:8;  +-------------+-------------+
};              | 0           | 1           |
                |      s      |     pad     |
                | 0         7 | 8        15 |
                +-------------+-------------+
\end{verbatim}
  \else
  \begin{tabular}{>{\texttt\bgroup}l<{\texttt\egroup}}
    union \{ \\
    ~~~~~char~~c; \\
    ~~~~~short~s:8; \\
    \}; \\
  \end{tabular}
  \quad
  \begin{tikzpicture}[baseline=(mid),x=1.3ex,y=3em]
    \path [bitchart box] (0, 0) rectangle (16, 1);
    \bitchartfields 0 8/c/, 16//;
    \bitchartbytes{0}{0,1}
    \begin{scope}[shift={(0,-1.2)}]
      \path [bitchart box] (0, 0) rectangle (16, 1);
      \bitchartfields 0 8/s/, 16//;
      \bitchartbytes{0}{0,1}
    \end{scope}
    \path (current bounding box.south west) coordinate (sw) {}
    (current bounding box.north west) coordinate (nw) {};
    \path (current bounding box.center) coordinate (mid) {};
    \path (current bounding box.north)
    node [above] {Halfword aligned, \texttt{sizeof} is 2};
    \path (nw) +(-1em,0) coordinate (nw) {};
    \draw [decorate,decoration=brace, thick] (sw) + (-1em,0) -- (nw);
  \end{tikzpicture}
  \fi
  \caption{Union allocation}
  \label{fig:unalloc}
\end{figure}

\begin{figure}
  \centering
  \ifSkipTikZ
\begin{verbatim}
                Byte aligned, sizeof is 9
               +-------------+-----------------------------------------+
               | 0           | 1                                       |
               |      c      |                     :0                  |
struct {       | 0         7 | 8                                    31 |
    char  c;   |-------------+-------------+--------------+------------|
    int   :0;  | 4           | 5           | 6            | 7          |
    char  d;   |      d      |     pad     |       :9     |    pad     |
    short :9;  | 32       39 | 40       47 | 48        55 | 56      63 |
    char  e;   |-------------+-------------+--------------+------------+
};             | 8           |
               |      e      |
               | 64       71 |
               +-------------+
\end{verbatim}
  \else
  \begin{tabular}{>{\texttt\bgroup}l<{\texttt\egroup}}
    struct \{ \\
    ~~~~~char~~c; \\
    ~~~~~int~~~:0; \\
    ~~~~~char~~d; \\
    ~~~~~short~:9; \\
    ~~~~~char~~e; \\
    \}; \\
  \end{tabular}
  \quad
  \begin{tikzpicture}[baseline=(mid),x=1.3ex,y=3em]
    \path [bitchart box] (0, -2) -| (8, -1) -| (32, 1) -- (0, 1) --cycle;
    \bitchartfields 0 8/c/, 32/:0/;
    \bitchartbytes{0}{0,1}
    \bitchartsep
    \begin{scope}[shift={(0,-1)}]
      \bitchartfields 32 40/d/, 48//, 56/:9/, 64//;
      \bitchartbytes{4}{4,...,7}
      \draw (0,0) -- (8,0);
    \end{scope}
    \begin{scope}[shift={(0,-2)}]
      \bitchartfields  64 72/e/;
      \bitchartbytes{8}{8}
    \end{scope}
    \path (current bounding box.center) coordinate (mid) {};
    \path (current bounding box.north)
    node [above] {Byte aligned, \texttt{sizeof} is 9};
  \end{tikzpicture}
  \fi
  \caption{Unnamed bit-fields}
  \label{fig:unnbitf}
\end{figure}

\subsubsection{Vector Types}
\index{vector type}
Vector types are used for SIMD\index{SIMD} (single-instruction,
multiple-data) programming.  They are not part of the C language, but
defined by a language extension, such as the ``vector extensions''
described in the respective section in the GCC manual~\cite{gnu-vec}.

A vector\index{vector} holds multiple values (``elements'') of a given
base type (``element type'').  Valid element types include the scalar
types shown in \cref{tab:scalar}, except for pointer types and the Boolean
type \texttt{\_Bool}.  The number of elements in a vector must be a power
of two.  Each allowed combination of base type and number of elements
forms a distinct vector type.  A single-element vector type is not
compatible with its base type.

The size of a vector type is the size of the base type multiplied by the
number of elements.  Vectors with a size of 1, 2, 4, or $\ge 8$ bytes are
aligned to a 1-, 2-, 4-, or 8-byte boundary, respectively.

\section{Function-Calling Sequence}
\index{function call}
This section discusses the standard function-calling sequence,
including stack frame layout, register usage, and parameter passing.

\subsection{Registers}
\index{registers}
The ABI makes the assumption that the processor has 16 general registers,
\texttt{r0} through \texttt{r15}, and 16 floating-point registers,
\texttt{f0} through \texttt{f15}.  \ARCH{} processors {\ifzseries have
  these registers; each register is 64 bits wide.\else have 16 general
  registers; newer models have 16 IEEE floating-point registers but older
  systems have only four non-IEEE floating-point registers.  On these
  older machines Linux emulates 16 IEEE registers within the kernel.  The
  width of the general registers is 32 bits, and the width of the
  floating-point registers is 64 bits.\fi}%

\index{vector registers}
Optionally, \ARCH{} processors may have a vector facility installed, which
extends the 64-bit floating-point registers to 128-bit vector registers,
\texttt{v0} through \texttt{v15}, and provides 16 additional 128-bit
vector registers, \texttt{v16} through \texttt{v31}.

In addition, the processor state includes 32-bit access registers
\texttt{a0} through \texttt{a15}, a 2-bit \index{condition code}condition
code \texttt{cc}, a 4-bit \index{program mask}program mask \texttt{pm},
and a 32-bit \index{FPC}floating-point control register \texttt{fpc}.

\subsubsection{Register Preservation Rules}
\begin{table}
  \centering
  \begin{DIFnomarkup}
  \begin{threeparttable}
    \begin{tabularx}{\textwidth}{lXl}
      \toprule
      Register name & Role(s) & Call effect\tnote{\dagger} \\
      \midrule
      \texttt{r0}, \texttt{r1} & -- & Volatile \\
      \texttt{r2}{\ifzseries\else , \texttt{r3}\fi}&
      Argument / return value & Volatile \\
      {\ifzseries\texttt{r3}, \fi}\texttt{r4}, \texttt{r5} &
      Arguments & Volatile \\
      \texttt{r6} & Argument & Saved \\
      \texttt{r7}…\texttt{r11} & -- & Saved \\
      \texttt{r12} & (Commonly used as GOT pointer) & Saved \\
      \texttt{r13} & (Commonly used as literal pool pointer) & Saved \\
      \texttt{r14} & Return address & Volatile \\
      \texttt{r15} & Stack pointer & Saved \\
      \texttt{f0} & Argument / return value & Volatile \\
      \ifzseries %----------------------------------------
      \texttt{f2}, \texttt{f4}, \texttt{f6} & Arguments & Volatile \\
      \texttt{f1}, \texttt{f3}, \texttt{f5}, \texttt{f7} & -- & Volatile \\
      \texttt{f8}…\texttt{f15} & -- & Saved \\
      \texttt{v0}…\texttt{v7} & (Extend \texttt{f0}…\texttt{f7}) & Volatile \\
      \texttt{v8}…\texttt{v15} & (Extend \texttt{f8}…\texttt{f15}) &
      Volatile\tnote{\dagger\dagger} \\
      \else %---------------------------------------------
      \texttt{f2} & Argument & Volatile \\
      \texttt{f4}, \texttt{f6} & -- & Saved \\
      \texttt{f1}, \texttt{f3}, \texttt{f5}, \texttt{f7}…\texttt{f15} &
      -- & Volatile \\
      \texttt{v0}…\texttt{v3}, \texttt{v5}, \texttt{v7}…\texttt{v15} &
      (Extend respective FP registers) & Volatile \\
      \texttt{v4}, \texttt{v6} & (Extend \texttt{f4}, \texttt{f6}) &
      Volatile\tnote{\dagger\dagger} \\
      \fi %-----------------------------------------------
      \texttt{v16}…\texttt{v23} & -- & Volatile \\
      \texttt{v24} & Argument / return value & Volatile \\
      \texttt{v25}…\texttt{v31} & Arguments & Volatile \\
      \texttt{cc} & Condition code & Volatile \\
      \texttt{pm} & Program mask & Cleared \\
      \texttt{a0}{\ifzseries , \texttt{a1}\fi} &
      Reserved for system use & Reserved \\
      \texttt{a\ifzseries 2\else 1\fi}…\texttt{a15} & -- & Volatile \\
      \bottomrule
    \end{tabularx}
    \medskip
    \begin{tablenotes}
    \item [\dagger] Volatile: These registers' values are not preserved across
      function calls.
    \item Saved: These registers' values are preserved across function
      calls.
    \item Cleared: These registers must be 0 before entering/leaving a
      function.
    \item Reserved: These registers must not be modified by
      ABI-compliant functions.
    \item [\dagger\dagger] Except for bytes 0--7, which are aliased to
      {\ifzseries \texttt{f8}…\texttt{f15}\else \texttt{f4},
        \texttt{f6}\fi}.
    \end{tablenotes}
  \end{threeparttable}
  \end{DIFnomarkup}
  \caption{Register usage across function calls}
  \label{tab:regsacrosscall}
\end{table}

\Cref{tab:regsacrosscall} summarizes the roles of registers and their
persistence\index{registers!across function call} across function calls.
Registers marked as ``saved'' are also referred to as
``nonvolatile''\index{nonvolatile}; they ``belong'' to the calling
function and must retain their values over the function call.  A called
function modifying these registers must restore their original values
before returning.  By contrast, ``volatile''\index{volatile} registers
need not be restored.  To preserve such a register's value across the
function call, the caller must take care of saving and restoring the value
by itself.  ``Reserved'' registers are reserved for system use and must
not be modified at all.

Using these definitions, the registers are categorized as follows:
\begin{itemize}
\item Registers \texttt{r6} through \texttt{r13}, \texttt{r15},
  {\ifzseries and \texttt{f8} through \texttt{f15}\else \texttt{f4}
    and \texttt{f6}\fi} are nonvolatile.
\item Access register{\ifzseries s\fi} \texttt{a0} {\ifzseries and
    \texttt{a1} are\else is\fi} reserved.
\item The left halves of vector registers {\ifzseries \texttt{v8} through
    \texttt{v15}\else \texttt{v4} and \texttt{v6}\fi} are nonvolatile,
  since they are aliased to {\ifzseries \texttt{f8} through
    \texttt{f15}\else \texttt{f4} and \texttt{f6}\fi}.  The right halves
  are volatile.
\item The program mask \texttt{pm} must be zero before entering and before
  leaving a function.
\item All other registers are volatile.
\item Furthermore the values in registers \texttt{r0} and \texttt{r1}
  may be altered by the interface code in cross-module calls, so a
  function cannot depend on the values in these registers having the
  same values that were placed in them by the caller.
  % FIXME: What about the FPC?  -> See the FPSCR in the POWER ABI.
\end{itemize}

\subsubsection{Register Roles}
The roles\index{registers!roles} mentioned in \cref{tab:regsacrosscall}
have the following meaning:
\begin{description}
\item[Argument:] \index{argument register} When calling a function, such a
  register may hold {\ifzseries\else (part of) \fi}%
  an argument to that function, according to the parameter-passing rules
  defined in \cref{parameterpassing}.
\item[Return value:] \index{return value!register} When a called function
  returns, such a register may hold {\ifzseries\else (part of)\fi}%
  the return value of that function, according to the rules defined in
  \cref{retvalues}.
\item[GOT pointer:] \index{GOT pointer} Global Offset Table pointer.  In a
  position-independent module, such a register may point to the start of
  that module's GOT, described in \cref{globaloffsettable}.  If
  instructions like ``Load Relative'' can be used, no GOT pointer may be
  needed.
\item[Literal pool pointer:] \index{literal pool pointer} Some constant
  local data objects (``literals'') can be encoded in the instructions
  themselves, using immediate values.  Others are typically grouped into
  pools of such literals, in which case a register may be set up as a base
  pointer to such a pool.
\item[Return address:] \index{return address register} When entering a
  function, this register, \texttt{r14}, contains the address of the
  instruction that the function must return to.  Except at function entry,
  no special role is assigned to \texttt{r14}.
\item[Stack pointer:] \index{stack pointer} This register, \texttt{r15},
  always points to the lowest allocated valid stack frame.  It shall
  maintain an 8-byte alignment.  A function may decrement \texttt{r15} to
  allocate a new stack frame or to enlarge the current one.  Before
  returning, \texttt{r15} must be restored to its original value.  For
  more information about stack frames, see \cref{stackframe}.
\end{description}

\subsubsection{Registers And Signal Handling}
Signals can interrupt processes.  Functions called during signal
handling have no unusual restrictions on their use of registers.
Moreover, if a signal-handling function returns, the process will
resume its original execution path with all registers restored to
their original values.  Thus programs and compilers may freely use all
registers listed above, except those reserved for system use, without
the danger of signal handlers inadvertently changing their values.

\subsubsection{Register Usage in Inline Assemblies}
\index{inline assembly!register usage}

With these calling conventions, the following usage of the registers
for inline assemblies is recommended:
\begin{itemize}
\item General registers \texttt{r0} and \texttt{r1} should be used
  internally whenever possible.
\item General registers \texttt{r2} to \texttt{r5} should be second
  choice.
\item General registers \texttt{r12} to \texttt{r15} should only be
  used for their standard function.
\end{itemize}

\subsection{The Stack Frame}
\label{stackframe}
\index{stack frame}

A function will be passed a frame on the runtime stack by the function
which called it, and may allocate a new stack frame.  A new stack frame is
required if the called function will in turn call further functions (which
must be passed the address of the new frame).  The stack grows downward
from high addresses.  Each stack frame is aligned on an 8-byte boundary.
General register \texttt{r15} holds the stack pointer and always
points to the first byte of the lowest allocated stack frame.
\Cref{fig:stackframe} shows the stack frame organization.

\begin{figure}
  \centering
  \ifSkipTikZ
\begin{verbatim}
         |          ...             |
         |                          |
         |   Previous stack frame   |
+------> ============================ High address
|        | Local and spill variable |
|        | area of calling function |
|        +--------------------------+
|        | Parameter area passed to |
|        |     called function      |
| SP+160 +--------------------------+
|        |  Register save area for  |
|        |   called function use    |
|   SP+8 +--------------------------+
+--------|  Back chain (optional)   |
   SP -> ============================ Low address
\end{verbatim}
  \else
  \pgfsetlayers{background,main}
  \begin{tikzpicture}
    \matrix [inner sep=0pt,
    nodes={text width=12em, text badly centered, inner sep=1ex}] (m)
    {
      \node (prev)     {\stackit[c]{\ldots\\\strut\\ Previous stack frame}}; \\
      \node (spill)    {Local and spill variable area of calling function}; \\
      \node (param)    {Parameter area passed to called function}; \\
      \node (save)     {Register save area for called function use}; \\
      \node (back)     {Back chain (optional)}; \\
    };
    \draw \foreach \Node in {prev, spill, param, save} {
      (\Node.south -| m.west) -- (\Node.south -| m.east)
    };
    \draw [very thick, shorten <=-1ex, shorten >=-1ex] (prev.south -| m.west)
    -- (prev.south -| m.east) node [right=1em] {High address};
    \path
    (m.south east) node [right=1em] {Low address}
    (param.south -| m.west) node [left]
    {\footnotesize\texttt{SP+\STACKSIZE{}}}
    (save.south -| m.west) node [left]
    {\footnotesize\texttt{SP+\NBYTES}}
    (m.south west) node [left=2em] (sp) {\texttt{SP}};
    \draw [very thick, shorten <=-1ex, shorten >=-1ex]
    (m.south west) -- (m.south east);
    \draw [->, shorten >=1.5ex] (sp) -- (m.south west);
    \draw [rounded corners, shorten >=1.5ex, ->]
    (back -| m.west)  -- +(-4em,0) |- (prev.south -| m.west);
    \begin{pgfonlayer}{background}
      \path [memory layout] (m.north west) |- (m.south east) --
      (m.north east);
    \end{pgfonlayer}
  \end{tikzpicture}
  \fi
  \caption[Standard stack frame]{Standard stack frame.  \texttt{SP}
    denotes the value of \texttt{r15} upon entering the called function.}
  \label{fig:stackframe}
\end{figure}

\subsubsection{Back-Chain Slot}
\index{back chain}
The first {\ifzseries double\fi}word of a calling function's stack frame
is preserved across function calls.  It may be used for maintaining a back
chain for stack unwinding, in which case it must hold the address of the
previously allocated stack frame (toward higher addresses), or zero
(\texttt{NULL}) if there is none.

Maintenance of the stack back chain is optional.  If a function chooses to
maintain the back chain, it should also store the values of \texttt{r14}
and \texttt{r15} at function entry into the register save area, using
their standard save slots as shown in \cref{fig:regsave}.

\subsubsection{Register Save Area}
\index{register save area}
The first \STACKSIZE{} bytes of a calling function's stack frame,
excluding the initial {\ifzseries double\fi}word, are referred to as the
register save area.  This area must be allocated by the caller and may be
used by the called function in any way.  For example, if the called
function is going to modify any nonvolatile registers, it may use the
register save area for saving these registers' original values first.  It
is customary to assign a standard save slot to each register, as shown in
\cref{fig:regsave}.

\begin{figure}
  \centering
  \ifSkipTikZ
\begin{verbatim}
160 +-------------------------------+
    | f6                            |
    | f4   Floating-point argument  |
    | f2     register save area     |
    | f0                            |
128 +-------------------------------+
    | r15                           |
    |  .                            |
    |  .   Other register save area |
    |  .                            |
    | r7                            |
 56 +-------------------------------+
    | r6                            |
    |  .                            |
    |  .   Argument register save   |
    |  .            area            |
    | r2                            |
 16 +-------------------------------+
    |              Unused           |
  8 +-------------------------------+
\end{verbatim}
  \else
  \begin{tikzpicture}
    \matrix [memory layout, inner sep=0pt, nodes={inner sep=1ex},
    description/.style={text width=12em, text badly centered},
    cells={anchor=center}] (m) {
      \node {\texttt{\stackit{f6\\f4\\f2\\f0}}}; &
      \node [description] {Floating-point argument register save area};
      \\
      \coordinate (A);\\
      \path (0,1.5ex) node [above] (r15) {\texttt{r15}};
      \path (0,-1.5ex) node [below] (r7) {\texttt{r7}}; &
      \node [description] {Other register save area};
      \\
      \coordinate (B);\\
      \path (0,1.5ex) node [above] (r6) {\texttt{r6}};
      \path (0,-1.5ex) node [below] (r2) {\texttt{r2}}; &
      \node [description] {Argument register save area};
      \\
      \coordinate (C);\\
      &
      \node [description] {Unused};
      \\
    };
    \draw [<->] (r7) -- (r15);
    \draw [<->] (r2) -- (r6);
    \foreach \Node in {A, B, C} {
      \draw (\Node -| m.west) -- (\Node -| m.east);
    };
    \ifzseries
    \foreach \where/\offs in {m.north/160, A/128, B/56, C/16, m.south/8} {
      \path (\where -| m.west) +(-1ex,0)
      node [left] {\texttt{\offs}};
    }
    \else
    \foreach \where/\offs in {m.north/96, A/64, B/28, C/8, m.south/4} {
      \path (\where -| m.west) +(-1ex,0)
      node [left] {\texttt{\offs}};
    }
    \fi
  \end{tikzpicture}
  \fi
  \caption[Register save area usage example]
  {Register save area usage example.  The slots for \texttt{r2} through
    \texttt{r5} and for the floating-point argument registers are used
    when the called function receives varying arguments.}
  \label{fig:regsave}
\end{figure}

\subsubsection{Parameter Area}
\index{parameter area}
The parameter area shall be allocated by a calling function if some
parameters cannot be passed in registers, but must be passed on the stack
instead (see \cref{parameterpassing}).  This area starts at byte offset
\STACKSIZE{} of the calling function's stack frame and consists of as many
\NBYTES{}-byte parameter slots as needed.  The calling function cannot
rely on the contents of these slots to be preserved across the function
call.

\subsubsection{Stack Frame Allocation}
\index{stack frame!allocation}
A function may allocate a new stack frame by decrementing the stack
pointer by the size of the new frame.  The stack pointer must be restored
prior to return.  By restoring the stack pointer, the allocated stack
frame is deallocated and may not be accessed after that.

A new stack frame is required if the function calls further functions.
Then the stack frame must at least contain the back chain slot, the
register save area, and the parameter area (if needed).  The remaining
space in the stack frame is called the ``local-variable area.''  It
immediately follows the parameter area and can have arbitrary size,
provided that it contains any padding necessary to make the entire frame a
multiple of 8 bytes in length.

If a function does not call any other functions and does not require more
stack space than available in the register save area, it need not
establish a stack frame.

\subsection{Parameter Passing}
\label{parameterpassing}
\index{parameter passing}
Arguments to called functions are passed in registers.  Since all
computations must be performed in registers, memory traffic can be
eliminated if the caller can compute arguments into registers and pass
them in the same registers to the called function, where the called
function can then use these arguments for further computation in the
same registers.  The number of registers implemented in a processor
architecture naturally limits the number of arguments that can be
passed in this manner.

This ABI defines that the following registers shall be used for parameter
passing:\index{registers!parameter passing}
\begin{itemize}
\item General registers \texttt{r2} to \texttt{r5} (volatile)
\item General register \texttt{r6} (nonvolatile)
\item Floating-point registers {\ifzseries\texttt{f0}, \texttt{f2},
    \texttt{f4} and \texttt{f6}\else \texttt{f0} and \texttt{f2}\fi}
  (volatile)
\item Vector registers \texttt{v24} to \texttt{v31} (volatile)
\end{itemize}

If needed, more arguments are passed in the parameter area, which starts
\STACKSIZE{} bytes above the stack pointer (see \cref{fig:prmarea}).

\begin{figure}
  \centering
  \ifSkipTikZ
\begin{verbatim}
                              High Address
      |        ...         |
      |                    | \
 +16  |  Parameter slot 3  |  |
      +--------------------+  |
  +8  |  Parameter slot 2  |   > Parameter area
      +--------------------+  |
 160  |  Parameter slot 1  |  |
      +--------------------+ /
      |                    |
      | Register save area |
   8  |                    |
      +--------------------+
      |   Back chain slot  |
      +--------------------+  Low Address
\end{verbatim}
  \else
  \pgfsetlayers{background,main}
  \begin{tikzpicture}
    \matrix [inner sep=0pt, nodes={inner sep=1ex}] (m) {
     \node (A0) {\stackit[c]{\ldots\\\strut\\ Parameter slot 3}};\\
      \coordinate (A) {}; \\
      \node {Parameter slot 2};\\
      \coordinate (B) {}; \\
      \node {Parameter slot 1};\\
      \coordinate (C) {}; \\
      \node [minimum height=4em] {Register save area};\\
      \coordinate (D) {}; \\
      \node {Back chain slot};\\
    };
    \foreach \Node in {A, ..., D} {
      \draw (\Node -| m.west) -- (\Node -| m.east);
    };
    \ifzseries
    \foreach \Node/\offs in {A/+16, B/+8, C/160, D/8} {
      \path (\Node -| m.west) + (-1ex,0)
      node [above left] {\texttt{\offs}};
    };
    \else
    \foreach \Node/\offs in {A/+8, B/+4, C/96, D/4} {
      \path (\Node -| m.west) + (-1ex,0)
      node [above left] {\texttt{\offs}};
    };
    \fi
    \path (m.south east) +(1ex,0) node [above right] {Low Address};
    \path (m.north east) +(1ex,0) node [below right] {High Address};
    \path (C -| m.east) ++ (1ex,0) coordinate (C right) {};
    \draw [decorate,decoration=brace, thick]
    (A0 -| m.east) + (1ex,0) -- (C right)
    node [midway, right=1ex] {Parameter area};
    \begin{pgfonlayer}{background}
      \path [memory layout] (m.north west) |- (m.south east) --
      (m.north east);
    \end{pgfonlayer}
  \end{tikzpicture}
  \fi
  \caption{Parameter area}
  \label{fig:prmarea}
\end{figure}

The following algorithm\index{parameter passing!algorithm} specifies where
argument data is passed for the C language.  For this purpose, consider
the arguments as ordered from left (first argument) to right, although the
order of evaluation of the arguments is unspecified.  In this algorithm
\texttt{fr} contains the number of the next available floating-point
register, \texttt{gr} contains the number of the next available general
register, and \texttt{starg} is the address of the next available stack
argument word.

\begin{description}
\item[\jumplabel{initialize}:] Allocate a sufficiently large parameter
  area for the arguments that will be passed according to the
  \jumplabel{more} and \jumplabel{more\_vec} descriptions that follow.
  Set $\mbox{\texttt{fr}}=0$, $\mbox{\texttt{gr}}=2$,
  $\mbox{\texttt{vr}}=24$, and \texttt{starg} to the address of the
  parameter area.
\item[\jumplabel{return\_parameter}:] If the called function's return
  value is not passed in a register (according to \cref{retvalues}), then
  allocate a return value buffer, store its address in \texttt{r2}, and
  set $\mbox{\texttt{gr}}=3$.
\item[\jumplabel{scan}:] If there are no more arguments, terminate.
  Otherwise, select one of the following depending on the type of the next
  argument:
  \begin{description}
  \item[\jumplabel{double\_or\_float}:] A double\_or\_float is
    one of the following:
    \begin{itemize}
    \item A \texttt{float} or \texttt{\_Decimal32}.
    \item A \texttt{double} or \texttt{\_Decimal64}.
    \item A structure equivalent to one of the above.  A structure is
      equivalent to a type $T$ if and only if it has exactly one member,
      which is either of type $T$ itself or a structure equivalent to
      type~$T$.
    \end{itemize}
    If $\mbox{\texttt{fr}}>{\ifzseries 6\else 2\fi}$, that is, if there are no
    more floating-point registers available for parameter passing, go to
    \jumplabel{more}.  Otherwise, load the argument value into
    floating-point register \texttt{fr}, set \texttt{fr} to
    $\mbox{\texttt{fr}}+2$, and go to \jumplabel{scan}.
  \item[\jumplabel{vector\_arg}:] A vector\_arg has one of the following
    types:
    \begin{itemize}
    \item Any vector type whose size is 16 bytes or less.
    \item A structure equivalent to such a vector type, where
      ``equivalent'' has the same meaning as for double\_or\_float.
    \end{itemize}
    If the argument is part of the varying arguments (see \cref{varargs}),
    or if $\mbox{\texttt{vr}}=\mbox{\emph{nil}}$, go to
    \jumplabel{more\_vec}.  Otherwise, load the value left-justified into
    vector register \texttt{vr}, set \texttt{vr} to the next entry in the
    list
    \[ 24, 26, 28, 30, 25, 27, 29, 31, \mbox{\emph{nil}} \]
    and go to \jumplabel{scan}.
    \ifzseries\else
  \item[\jumplabel{double\_arg}:] A double\_arg is one of type
    \texttt{long long}, or is a struct or a union of size 8 bytes which is
    not a double\_or\_float.%
    \\
    If $\mbox{\texttt{gr}}>5$ set \texttt{gr} to 7 and go to
    \jumplabel{more}.  Else load the argument's lower-addressed word into
    \texttt{gr} and the higher-addressed word into $\mbox{\texttt{gr}}+1$,
    set \texttt{gr} to $\mbox{\texttt{gr}}+2$, and go to \jumplabel{scan}.
    \fi
  \item[\jumplabel{simple\_arg}:] A simple\_arg is one of the following:
    \begin{itemize}
    \item One of the simple integer types no more than \NBITS{} bits wide.
      This includes \texttt{signed} \texttt{char}, \texttt{short},
      \texttt{int}, \texttt{long},{\ifzseries{} \texttt{long}
        \texttt{long},\fi} their unsigned counterparts, \texttt{\_Bool},
      and any \texttt{enum} type.  If such an argument is shorter than
      \NBITS{} bits, replace it by a full \NBITS{}-bit integer
      representing the same number, using sign or zero extension, as
      appropriate.
    \item Any pointer type.
    \item A struct or a union of 1, 2, {\ifzseries 4, or 8\else or 4\fi}
      bytes that is not a double\_or\_float (see above).  If such a
      struct or union is strictly smaller than \NBYTES{} bytes, extend it
      to \NBYTES{} bytes by adding padding bytes with unspecified contents
      on the left.
    \item A struct or union of any other size, a complex type, an
      \texttt{\_\_int128}, a \texttt{long} \texttt{double}, a
      \texttt{\_Decimal128}, or a vector whose size exceeds 16 bytes.
      Replace such an argument by a pointer to the object, or to a copy
      where necessary to enforce call-by-value semantics.  Only if the
      caller can ascertain that the object is ``constant'' can it pass a
      pointer to the object itself.
    \end{itemize}
    If $\mbox{\texttt{gr}}>6$, go to \jumplabel{more}.  Otherwise load the
    argument value (now \NBITS{} bits wide) into general register
    \texttt{gr}, set \texttt{gr} to $\mbox{\texttt{gr}}+1$, and go to
    \jumplabel{scan}.
  \end{description}
\item[\jumplabel{more}:] The argument cannot be passed in registers; it
  will be passed in the parameter area of the caller's stack frame
  instead.  After having applied the replacement rules previously
  explained as appropriate, the argument now has a size of {\ifzseries 8
    bytes, except when its type is equivalent to \texttt{float} or
    \texttt{\_Decimal32}, in which case it has 4 bytes.\else 4 or 8
    bytes.\fi}%
  \\
  Copy the argument value {\ifzseries right-aligned into the 8-byte
    parameter slot at the current stack position \texttt{starg}, leaving
    the skipped bytes (if any) at unspecified values.\else to the current
    stack position \texttt{starg}.\fi} Increment \texttt{starg} by
  {\ifzseries 8\else the argument size\fi}, then go to \jumplabel{scan}.
\item[\jumplabel{more\_vec}:] The argument cannot be passed in vector
  registers, but will be passed in the parameter area.  Copy its value to
  the current stack position \texttt{starg}, increment \texttt{starg} by
  the argument size, align \texttt{starg} to the next \NBYTES{}-byte
  boundary, and go to \jumplabel{scan}.
\end{description}

As an example, assume the declarations and the function call shown in
\cref{lst:prmpass}.  The corresponding register allocation and
storage would be as shown in \cref{tab:prmpass}.

\ifzseries\else In this example \texttt{r6} is unused as the \texttt{long}
\texttt{long} variable \texttt{ll} will not fit into a single register.\fi

\begin{table}
  \centering
  \begin{lstlisting}[style=embed, label=lst:prmpass,
    caption={Parameter-passing example}]
typedef float __attribute__((vector_size(8)) v2f_t;

int i, j, k, l;
long long ll;
double f, g, h;
v2f_t v1, v2;
int m;
x = func(i, j, g, k, l, ll, f, h, m, v1, v2);
  \end{lstlisting}
  \begin{DIFnomarkup}
  \par\medskip
  \newdimen\mycolwidth\mycolwidth=.22\hsize
  \begin{tabular}[t]{%
      >{\texttt\bgroup}l<{\egroup :~}>{\texttt\bgroup}l<{\egroup}
      >{\texttt\bgroup}l<{\egroup :~}>{\texttt\bgroup}l<{\egroup}
      >{\texttt\bgroup}l<{\egroup :~}>{\texttt\bgroup}l<{\egroup}
      r<{:~}>{\texttt\bgroup}l<{\egroup}}
    \toprule
    \multicolumn{2}{>{\raggedright}p{\mycolwidth}}{General registers} &
    \multicolumn{2}{>{\raggedright}p{\mycolwidth}}{Floating-point registers} &
    \multicolumn{2}{>{\raggedright}p{\mycolwidth}}{Vector registers} &
    \multicolumn{2}{>{\raggedright}p{\mycolwidth}}{Stack frame offset} \\
    \midrule
    \ifzseries
    r2 & i  & f0    & g & v24   & v1 & 160   & m  \\
    r3 & j  & f2    & f & v26   & v2 & \omit &    \\
    r4 & k  & f4    & h & \omit &    & \omit &    \\
    r5 & l  & \omit &   & \omit &    & \omit &    \\
    r6 & ll & \omit &   & \omit &    & \omit &    \\
    \else
    r2 & i  & f0    & g & v24   & v1 & 96    & ll \\
    r3 & j  & f2    & f & v26   & v2 & 104   & h  \\
    r4 & k  & \omit &   & \omit &    & 112   & m  \\
    r5 & l  & \omit &   & \omit &    & \omit &    \\
    r6 & -- & \omit &   & \omit &    & \omit &    \\
    \fi
    \bottomrule
  \end{tabular}
  \end{DIFnomarkup}
  \caption{Parameter-passing example: register allocation}
  \label{tab:prmpass}
\end{table}

\subsection{Variable Argument Lists}
\label{varargs}
\index{variable argument list}
If a C function declaration has a parameter type list that terminates with
an ellipsis ``\texttt{...}\,,'' a call to that function can have varying
numbers and types of arguments corresponding to the ellipsis.  Except for
vector arguments of 16 bytes or less, these varying arguments are passed
to the called function as if the ellipsis were replaced with a parameter
type list of the actual arguments.  Varying vector arguments are always
passed in the parameter area.

\begin{lstlisting}[style=float,caption={\texttt{va\_list}
    declaration example},label={lst:valist}]
typedef struct __va_list_tag {
    long __gpr;
    long __fpr;
    void *__overflow_arg_area;
    void *__reg_save_area;
} va_list[1];
\end{lstlisting}

The called function can store the varying arguments in a variable of type
\texttt{va\_list}, defined in \texttt{<stdarg.h>}.  Such a variable
represents the list of remaining arguments to be processed and can be
passed down to further functions.  The \ABINAME{} ABI defines
\texttt{va\_list} to be equivalent to a structure with four {\ifzseries
  double\fi}word members, or to an array whose single element is such a
structure, like the declaration shown in \cref{lst:valist}.  The
declaration as an array reduces copying of the structure when used as an
argument.  The structure members have the following meaning:

\begin{description}
\item[\texttt{\_\_gpr}] holds the number (0 to 5) of general argument
  registers that have already been processed.
\item[\texttt{\_\_fpr}] holds the number (0 to {\ifzseries 4\else 2\fi})
  of floating-point argument registers that have already been processed.
\item[\texttt{\_\_overflow\_arg\_area}] points to the first ``overflow
  argument'' (passed via the parameter area) that has not been processed
  yet.
\item[\texttt{\_\_reg\_save\_area}] points to the start of a
  \STACKSIZE{}-byte memory region that contains the saved values of all
  argument registers, with the general registers (\texttt{r2} to
  \texttt{r6}) starting at offset {\ifzseries 16\else 8\fi} and the
  floating-point registers ({\ifzseries\texttt{f0}, \texttt{f2},
    \texttt{f4}, and \texttt{f6}\else \texttt{f0} and \texttt{f2}\fi})
  starting at offset {\ifzseries 128\else 64\fi}.  These offsets
  correspond to the layout shown in \cref{fig:regsave}.  The argument
  registers that have already been processed do not actually need to be
  saved in their slots.
\end{description}

\paragraph{Note:}
Since \texttt{va\_list} may be defined as an array, a variable of this
type cannot be copied by a simple C assignment.  The standard C header
\texttt{<stdarg.h>} defines the macro \texttt{va\_copy} for this purpose
instead.  Any C code that intends to be portable across platforms should
use this macro for copying a \texttt{va\_list} variable.

\subsection{Return Values}
\label{retvalues}
\index{return value!passing}
\index{registers!return value passing}
A function must pass its return value either in general register
\texttt{r2},{\ifzseries\else{} in the register pair
  \texttt{r2}/\texttt{r3},\fi} in floating-point register \texttt{f0}, in
vector register \texttt{v24}, or in a return value buffer allocated by the
caller, depending on the return value type:

\begin{itemize}
\item A value of type \texttt{double} or \texttt{\_Decimal64} is returned
  in \texttt{f0}.
\item A value of type \texttt{float} or \texttt{\_Decimal32} is returned
  in the left half of \texttt{f0} and encoded in short BFP format or short
  DFP format, respectively.  The right half of \texttt{f0} is unspecified.
\item Any integer type with \NBITS{} or fewer bits, including
  \texttt{\_Bool}, as well as any \texttt{enum} type, is returned in
  \texttt{r2}.  The return value is zero- or sign-extended to \NBITS{}
  bits, as appropriate.
\item A pointer to any type is returned in \texttt{r2}.
  \ifzseries\else
\item A value of type \texttt{long} \texttt{long} or \texttt{unsigned}
  \texttt{long} \texttt{long} is returned with the lower addressed half in
  \texttt{r2} and the higher in \texttt{r3}.\fi
\item A vector of 16 or fewer bytes is returned left-aligned in
  \texttt{v24}.  The padding bits' values are unspecified.
\item Any other type, such as \texttt{long} \texttt{double},
  \texttt{\_Decimal128}, \texttt{\_\_int128}, a complex type, a structure,
  a union, or a vector larger than 16 bytes, is returned in a return value
  buffer allocated by the caller.  This buffer's address is treated like a
  ``hidden argument'' and passed by the caller in \texttt{r2}.
\end{itemize}

\section{Operating System Interface}
This section describes various interfaces with the operating system that
are specific to the \ABINAME{} ABI\@.

\subsection{Signal Context}
\index{signal context}
A signal handler that was installed with \texttt{sigaction} using the
\texttt{SA\_SIGINFO} flag receives three arguments, as follows:
\begin{center}
  \lstinline@void handler(int sig, siginfo_t *info, void *ucontext);@
\end{center}
The second argument \texttt{info} is a pointer to a structure containing
additional signal information, including the number \texttt{si\_code} that
indicates why the signal \texttt{sig} was sent.

The third argument \texttt{ucontext} points to a \texttt{ucontext\_t}
structure on the stack where signal-related context information has been
saved by the operating system.  It contains the processing context to be
restored when resuming the interrupted program, including the
architecture-dependent register state.  Although most signal handlers will
ignore this information, some may access it for debugging purposes such as
printing the registers, or when their logic depends on that state.

\Cref{lst:ucontext} shows the declaration of \texttt{ucontext\_t} on
systems implementing the \ABINAME{} ABI.

\begin{lstlisting}[style=float,caption={[The \texttt{ucontext\_t}
    structure]The \texttt{ucontext\_t} structure.  The size of
    \texttt{uc\_sigmask} may vary, and additional information may be
    stored after it.},label={lst:ucontext}]
typedef struct {
    unsigned long      mask;       /* PSW mask */
    unsigned long      addr;       /* PSW address */
} __psw_t;

typedef union {
    double             d;
    float              f;
} fpreg_t;

typedef struct {
    unsigned int       fpc;        /* floating-point control register */
    fpreg_t            fprs[16];   /* floating-point registers */
} fpregset_t;

typedef struct {
    __psw_t            psw;
    unsigned long      gregs[16];  /* general registers */
    unsigned int       aregs[16];  /* access registers */
    fpregset_t         fpregs;
} mcontext_t;

typedef struct {
    void              *ss_sp;
    int                ss_flags;
    size_t             ss_size;
} stack_t;

typedef ... sigset_t;               /* opaque type */

struct ucontext_t {
    unsigned long      uc_flags;
    struct ucontext_t *uc_link;
    stack_t            uc_stack;
    mcontext_t         uc_mcontext; /* machine-specific context */
    sigset_t           uc_sigmask;  /* blocked signals */
};
\end{lstlisting}

\subsection{Exception Interface}
\label{exceptionint}
\index{exception}
When the CPU detects an exceptional condition while a process is executing
instructions, an \index{interruption}interruption may occur, transferring
control to the operating system.  The operating system then handles the
interruption either in a manner transparent to the application, or by
delivering a signal.

If such an exception and its corresponding interruption are immediately
caused by the execution of an instruction, the exception is called
``synchronous''.  Program interruptions generally fall into this category.
They may give rise to \texttt{SIGILL}, \texttt{SIGSEGV}, \texttt{SIGBUS},
\texttt{SIGTRAP}, or \texttt{SIGFPE}\@.  If one of these signals is
generated due to an exception when the signal is blocked, the behavior is
undefined.

When a signal handler other than for \texttt{SIGSEGV} or \texttt{SIGBUS}
gets control after a synchronous exception, the \texttt{si\_addr} field in
the signal handler's \texttt{siginfo\_t} argument points to the
instruction that caused the exception, while the \index{PSW address!after
  signal}PSW address in the signal context points to the next instruction.

In the case of \texttt{SIGSEGV} or \texttt{SIGBUS}, the PSW address points
to the faulting instruction instead, whereas \texttt{si\_addr} points to
the address of the memory access causing the fault, possibly rounded down
to a page boundary.

The correspondence between the causes of program interruptions and the
resulting signals\index{signal!from exception} is shown in
\cref{tab:exceptions}.

\begin{table}
  \centering
  \begin{DIFnomarkup}
  \begin{threeparttable}
    \begin{tabular}{llll}
      \toprule
      \ARCH{} exception
      & Signal & \texttt{si\_code} \\
      \midrule
      Addressing & \multirow{10}{*}{\texttt{SIGILL}} & \texttt{ILL\_ILLADR} \\
      Data, general-operand & & \texttt{ILL\_ILLOPN} \\
      Execute & & \texttt{ILL\_ILLOPN} \\
      Operand & & \texttt{ILL\_ILLOPN} \\
      Operation, no breakpoint\tnote{\dagger} & & \texttt{ILL\_ILLOPC} \\
      Privileged-operation & & \texttt{ILL\_PRVOPC} \\
      Special-operation & & \texttt{ILL\_ILLOPN} \\
      Space-switch & & \texttt{ILL\_PRVOPC} \\
      Specification & & \texttt{ILL\_ILLOPN} \\
      Transaction-constraint & & \texttt{ILL\_ILLOPN} \\
      \midrule
      Operation, breakpoint\tnote{\dagger} & \texttt{SIGTRAP}
               & \texttt{TRAP\_BRKPT} \\
      \midrule
      Data, (simulated) IEEE invalid operation
      & \multirow{18}{*}{\texttt{SIGFPE}} & \texttt{FPE\_FLTINV} \\
      Data, (simulated) IEEE division by zero & & \texttt{FPE\_FLTDIV} \\
      Data, any (simulated) IEEE overflow & & \texttt{FPE\_FLTOVF} \\
      Data, any (simulated) IEEE underflow & & \texttt{FPE\_FLTUND} \\
      Data, any (simulated) IEEE inexact\tnote{\ddagger} &
               & \texttt{FPE\_FLTRES} \\
      Data, neither IEEE nor general-operand & & \texttt{SI\_USER} \\
      Fixed-point/decimal divide & & \texttt{FPE\_INTDIV} \\
      Fixed-point/decimal overflow & & \texttt{FPE\_INTOVF} \\
      HFP divide & & \texttt{FPE\_FLTDIV} \\
      HFP exp\@. overflow & & \texttt{FPE\_FLTOVF} \\
      HFP exp\@. underflow & & \texttt{FPE\_FLTUND} \\
      HFP square root & & \texttt{FPE\_FLTINV} \\
      HFP significance & & \texttt{FPE\_FLTRES} \\
      Vector-processing, invalid operation & & \texttt{FPE\_FLTINV} \\
      Vector-processing, division by zero & & \texttt{FPE\_FLTDIV} \\
      Vector-processing, overflow & & \texttt{FPE\_FLTOVF} \\
      Vector-processing, underflow & & \texttt{FPE\_FLTUND} \\
      Vector-processing, inexact & & \texttt{FPE\_FLTRES} \\
      \midrule
      Protection
      & \multirow{2}{*}{\texttt{SIGSEGV}} & \texttt{SEGV\_ACCERR} \\
      Any translation\tnote{*} & & \texttt{SEGV\_MAPERR} \\
      \midrule
      Any translation\tnote{*} & \texttt{SIGBUS} & \texttt{BUS\_ADDRERR} \\
      \bottomrule
    \end{tabular}
    \medskip
    \begin{tablenotes}
    \item [\dagger] A breakpoint\index{breakpoint} is recognized when a
      \texttt{ptrace} target executes the special illegal instruction
      \texttt{0x0001}.
    \item [\ddagger] Except if an overflow or underflow condition is
      indicated as well.
    \item [*] For a translation exception the operating system may yield
      SIGSEGV or SIGBUS, or it may handle the fault without a signal.
    \end{tablenotes}
  \end{threeparttable}
  \end{DIFnomarkup}
  \caption[Exceptions and signals]{Exceptions and signals.
    \texttt{si\_code} refers to the respective field in
    \texttt{siginfo\_t}.}
  \label{tab:exceptions}
\end{table}

\subsection{Virtual Address Space}
\index{address space}
Processes execute in a \ADDRBITS{}-bit virtual address
space.  Memory management translates virtual addresses to physical
addresses, hiding physical addressing and letting a process run
anywhere in the system's real memory.  Processes typically begin with
three logical segments, commonly called ``text,'' ``data,'' and
``stack.''  An object file may contain more segments (for example, for
debugger use), and a process can also create additional segments for
itself with system services.

\paragraph{Note:}
\index{virtual address}
The term ``virtual address'' as used in this document refers to a
\ADDRBITS{}-bit address generated by a program, as
contrasted with the physical address to which it is mapped.

\subsection{Page Size}
\index{memory page}
\index{page size}
Memory is organized into pages, which are the system's smallest units
of memory allocation.  The hardware page size for \ARCHarch{}
is 4096 bytes.

\subsection{Virtual Address Assignments}
Processes have {\ifzseries a 42, 53, or 64\else the full 31\fi}-bit
address space available to them{\ifzseries, depending on the Linux
  kernel level\fi}.

\Cref{fig:vac} shows the virtual address configuration on \theARCH{}.
The segments with different properties are typically
grouped in different areas of the address space.  The loadable segments
may begin at zero (\texttt{0}); the exact addresses depend on the
executable file format (see \cref{chobjfiles,chprogload}).  The process's
stack resides at the end of the virtual memory and grows downwards.
Processes can control the amount of virtual memory allotted for stack
space, as described below.

\begin{figure}
  \centering
  \ifSkipTikZ
\begin{verbatim}
0x3ffffffffff+----------------------------+ End of memory
             |                            |
             |           Stack            |
             |                            |
             +----------------------------+
             |                            |
             |      Dynamic segments      |
  Anonymous  |                            |
mapping base +----------------------------+
             |                            |
             |            Heap            |
             |                            |
             +----------------------------+
             |                            |
             |      Executable file       |
             |                            |
Program base +----------------------------+
             |                            |
             |         Unmapped           |
             |                            |
0x00000000   +----------------------------+ Beginning of memory
\end{verbatim}
  \else
  \begin{tikzpicture}
    \matrix [memory layout,nodes={minimum height=3em}] (m) {
      \node (stack)  {Stack}; \\
      \node (dynseg) {Dynamic segments}; \\
      \node (heap)   {Heap}; \\
      \node (exec)   {Executable file}; \\
      \node (unmap)  {Unmapped}; \\
    };
    \draw (stack.south -| m.west) -- (stack.south -| m.east);
    \draw (dynseg.south -| m.west)
    node [left=1em, text width=6em, align=right] {Anonymous mapping base}
    -- (dynseg.south -| m.east);
    \draw (heap.south -| m.west) -- (heap.south -| m.east);
    \draw (exec.south -| m.west)
    node [left=1em, text width=6em, align=right] {Program base}
    -- (exec.south -| m.east);
    \path (m.south west) node [left=1em] {\texttt{0}}
    (m.north west) node [left=1em]
    {\texttt{\ifzseries 0x3ffffffffff\else 0x7fffffff\fi}}
    (m.south east) node [right=1em] {Beginning of memory}
    (m.north east) node [right=1em] {End of memory};
  \end{tikzpicture}
  \fi
  \caption{{\ifzseries 42-bit virtual\else Virtual\fi} address
    configuration}
  \label{fig:vac}
\end{figure}

\paragraph{Note:}
Although application programs may begin at virtual address 0, they
conventionally begin above \texttt{0x1000} (4$\,$Kbytes), leaving the
initial 4$\,$Kbytes with an invalid address mapping.  Processes that
reference this invalid memory (for example by de-referencing a
null pointer) generate a translation exception as described in
\cref{exceptionint}.

Although applications may control their memory assignments, the
typical arrangement follows \cref{fig:vac}.

\subsection{Managing the Process Stack}
\Cref{procinit} describes the initial stack contents.
Stack addresses can change from one system to the next---even from one
process execution to the next on a single system.  A program,
therefore, should not depend on finding its stack at a particular
virtual address.

A tunable configuration parameter controls the system maximum stack size.
A process can also use \texttt{setrlimit} to set its own maximum stack
size, up to the system limit.  The stack segment is both readable and
writable.

\subsection{Coding Guidelines}
Operating system facilities, such as \texttt{mmap}, allow a process to
establish address mappings in two ways.  Firstly, the program can let
the system choose an address.  Secondly, the program can request the
system to use an address the program supplies.  The second alternative
can cause application portability problems because the requested
address might not always be available.  Differences in virtual address
space can be particularly troublesome between different architectures,
but the same problems can arise within a single architecture.

Processes' address spaces typically have three segments that can
change size from one execution to the next: the stack (through
\texttt{setrlimit}); the data segment (through \texttt{malloc}); and
the dynamic segment area (through \texttt{mmap}).  Changes in one area
may affect the virtual addresses available for another.  Consequently
an address that is available in one process execution might not be
available in the next.  Thus a program that used \texttt{mmap} to
request a mapping at a specific address could appear to work in some
environments and fail in others.  For this reason programs that want
to establish a mapping in their address space should let the system
choose the address.

Despite these warnings about requesting specific addresses, the
facility can be used properly.  For example, a multiprocess
application might map several files into the address space of each
process and build relative pointers among the files' data.  This could
be done by having each process ask for a certain amount of memory at
an address chosen by the system.  After each process received its own
private address from the system it would map the desired files into
memory at specific addresses within the original area.  This
collection of mappings could be at different addresses in each process
but their relative positions would be fixed.  Without the ability to
ask for specific addresses, the application could not build shared
data structures because the relative positions for files in each
process would be unpredictable.

\subsection{Processor Execution Modes}
Two execution modes exist in \ARCHarch{}: problem (user) state and
supervisor state.  Processes run in problem state (the less privileged).
The operating system kernel runs in supervisor state.  A program executes
a ``Supervisor Call'' (\texttt{SVC}) instruction to change execution
modes.

Note that the ABI does not define the implementation of individual
system calls.  Instead programs should use the system libraries.
Programs with embedded \texttt{SVC} instructions do not conform
to the ABI.

\section{Process Initialization}
\label{procinit}
\index{process initialization}
\index{initialization!process}
This section describes the machine state that \texttt{exec} creates
for ``infant'' processes, including argument passing, register usage,
and stack frame layout.  Programming language systems use this initial
program state to establish a standard environment for their
application programs.  For example, a C program begins executing at a
function named \texttt{main}, conventionally declared as follows:
\begin{center}
  \lstinline@extern int main (int argc, char *argv[ ], char *envp[ ]);@
\end{center}

Its parameters are passed from the C programming language system when
invoking \texttt{main}.  They are:
\begin{description}
\item[\texttt{argc}] a non-negative argument count
\item[\texttt{argv}] an array of argument strings, with
  \begin{center}
    \lstinline@argv[argc] == NULL@
  \end{center}
\item[\texttt{envp}] an array of environment strings, also terminated by a
  null pointer
\end{description}

Although this section does not describe C program initialization, it
gives the information necessary to implement the call to \texttt{main}
or to the entry point for a program in any other language.

\subsection{Registers}
\index{registers!process startup}
\index{process initialization!registers}
When a process is first entered (from an \texttt{exec} system call),
the contents of registers other than those listed below are
unspecified.  Consequently, a program that requires registers to have
specific values must set them explicitly during process
initialization.  It should not rely on the operating system to set all
registers to 0.  Following are the registers whose contents are
specified:
\begin{description}
\item[\texttt{r15}] The initial stack pointer, aligned to an 8-byte
  boundary and pointing to a stack location that contains the
  argument count (see \cref{processstack} for further
  information about the initial stack layout).
\item[\texttt{fpc}] The floating-point control register contains 0,
  specifying ``round to nearest'' mode and the disabling of
  floating-point exceptions.
\end{description}

\subsection{Process Stack}
\label{processstack}
\index{process initialization!stack}
Every process has a stack, but the system defines no fixed stack
address.  Furthermore, a program's stack address can change from one
system to another---even from one process invocation to another.
Thus the process initialization code must use the stack address in
general register \texttt{r15}.  Data in the stack segment at
addresses below the stack pointer contain undefined values.

When a process receives control, its stack holds the arguments,
environment, and auxiliary vector (see \cref{auxvector}) from
\texttt{exec}.  Argument strings, environment strings, and the auxiliary
information appear in no specific order within the information block; the
system makes no guarantees about their relative arrangement.  The system
may also leave an unspecified amount of memory between the \texttt{NULL}
auxiliary vector entry and the beginning of the information block.  A
sample initial stack is shown in \cref{fig:inistack}.

\begin{figure}
  \centering
  \ifSkipTikZ
\begin{verbatim}
      +--------------------------------+  Top of Stack
      |  Information block, including  |
      |  argument and environment      |
      |  strings and auxiliary         |
      |  information (size varies)     |
      +--------------------------------+
      |        Unspecified             |
      +--------------------------------+
      | AT_NULL auxiliary vector entry |
      +--------------------------------+
      |       Auxiliary vector         |
      |       (4-word entries)         |
      +--------------------------------+
      |       Zero doubleword          |
      +--------------------------------+
      |       Environment pointers     |
      |          (2-word each)         |
      +--------------------------------+
      |       Zero doubleword          |
      +--------------------------------+
      |        Argument pointers       |
      |          (2-word each)         |
      +--------------------------------+
      |   Argument count doubleword    |
%r15  +--------------------------------+  Low Address
\end{verbatim}
  \else
  \begin{tikzpicture}
    \matrix [memory layout,inner sep=0pt,
    nodes={text width=16em, text badly centered, inner sep=1ex}] (m) {
      \node (A) {Information block, including argument and
        environment strings and auxiliary information (size varies)}; \\
      \node (B) {Unspecified}; \\
      \node (C) {\texttt{AT\_NULL} auxiliary vector entry}; \\
      \node (D) {Auxiliary vector ({\ifzseries 4\else 2\fi}-word entries)}; \\
      \node (E) {Zero {\ifzseries double\fi}word}; \\
      \node (F) {Environment pointers ({\ifzseries 2\else 1\fi}-word each)}; \\
      \node (G) {Zero {\ifzseries double\fi}word}; \\
      \node (H) {Argument pointers ({\ifzseries 2\else 1\fi}-word each)}; \\
      \node (I) {Argument count {\ifzseries double\fi}word}; \\
    };
    \foreach \Node in {A,...,H} {
      \draw (\Node.south -| m.west) -- (\Node.south -| m.east);
    }
    \path (m.south west) node [left=1em] (r15) {\texttt{r15}}
    (m.south east) node [right=1em] {Low address}
    (m.north east) node [right=1em] {Top of stack};
    \draw [->, shorten >=1pt] (r15) -- (m.south west);
  \end{tikzpicture}
  \fi
  \caption{Initial process stack}
\label{fig:inistack}
\end{figure}

\subsection{Auxiliary Vector}
\label{auxvector}
\index{auxiliary vector}
\index{process initialization!auxiliary vector}
Whereas the argument and environment vectors transmit information from
one application program to another, the auxiliary vector conveys
information from the operating system to the program.  This vector is
an array of structures, which are defined in \cref{lst:auxstruct}.

\begin{lstlisting}[style=float,label=lst:auxstruct,
  caption=Auxiliary vector structure,escapechar=@]
typedef struct {
    @\ifzseries long\else int\fi@ a_type;
    union {
        long a_val;
        void *a_ptr;
        void (*a_fcn)();
    } a_un;
} auxv_t;
\end{lstlisting}

The structures are interpreted according to the \texttt{a\_type}
member, as shown in \cref{tab:auxtypes}.

\begin{table}
  \centering
  \begin{DIFnomarkup}
  \begin{tabular}{lrl!{\qquad}lrl}
    \toprule
    Name & Value & \texttt{a\_un}
    & Name & Value & \texttt{a\_un} \\
    \midrule
    \texttt{AT\_NULL} & 0 & ignored
    & \texttt{AT\_UID} & 11 & \texttt{a\_val} \\
    \texttt{AT\_IGNORE} & 1 & ignored
    & \texttt{AT\_EUID} & 12 & \texttt{a\_val} \\
    \texttt{AT\_EXECFD} & 2 & \texttt{a\_val}
    & \texttt{AT\_GID} & 13 & \texttt{a\_val} \\
    \texttt{AT\_PHDR} & 3 & \texttt{a\_ptr}
    & \texttt{AT\_EGID} & 14 & \texttt{a\_val} \\
    \texttt{AT\_PHENT} & 4 & \texttt{a\_val}
    & \texttt{AT\_PLATFORM} & 15 & \texttt{a\_ptr} \\
    \texttt{AT\_PHNUM} & 5 & \texttt{a\_val}
    & \texttt{AT\_HWCAP} & 16 & \texttt{a\_val} \\
    \texttt{AT\_PAGESZ} & 6 & \texttt{a\_val}
    & \texttt{AT\_CLKTCK} & 17 & \texttt{a\_val} \\
    \texttt{AT\_BASE} & 7 & \texttt{a\_ptr}
    & \texttt{AT\_SECURE} & 23 & \texttt{a\_val} \\
    \texttt{AT\_FLAGS} & 8 & \texttt{a\_val}
    & \texttt{AT\_RANDOM} & 25 & \texttt{a\_ptr} \\
    \texttt{AT\_ENTRY} & 9 & \texttt{a\_ptr}
    & \texttt{AT\_EXECFN} & 31 & \texttt{a\_ptr} \\
    \texttt{AT\_NOTELF} & 10 & \texttt{a\_val}
    & \texttt{AT\_SYSINFO\_EHDR} & 33 & \texttt{a\_ptr} \\
    \bottomrule
  \end{tabular}
  \end{DIFnomarkup}
  \caption{Auxiliary vector types, \texttt{a\_type}}
  \label{tab:auxtypes}
\end{table}

\begin{table}
  \begin{DIFnomarkup}
  \begin{tabularx}{\textwidth}{lr>{\raggedright\arraybackslash}X}
    \toprule
    Name & Value & Description \\
    \midrule
    \texttt{HWCAP\_S390\_ZARCH} & \texttt{0x2}
    & Running in z/Architecture mode \\
    \texttt{HWCAP\_S390\_STFLE} & \texttt{0x4}
    & Store-facility-list-extended facility installed \\
    \texttt{HWCAP\_S390\_MSA} & \texttt{0x8}
    & Message-security assist available \\
    \texttt{HWCAP\_S390\_LDISP} & \texttt{0x10}
    & Long-displacement facility installed \\
    \texttt{HWCAP\_S390\_EIMM} & \texttt{0x20}
    & Extended-immediate facility installed \\
    \texttt{HWCAP\_S390\_DFP} & \texttt{0x40}
    & Decimal floating-point facility and perform floating-point
    facility (PFPO) installed \\
    \texttt{HWCAP\_S390\_HPAGE} & \texttt{0x80}
    & Huge page support available \\
    \texttt{HWCAP\_S390\_ETF3EH} & \texttt{0x100}
    & Extended-translation facility 3 and ETF3-enhancement
    facility installed \\
    \texttt{HWCAP\_S390\_TE} & \texttt{0x400}
    & Transactional-execution facility installed \\
    \texttt{HWCAP\_S390\_VXRS} & \texttt{0x0800}
    & Vector facility installed \\
    \texttt{HWCAP\_S390\_VXRS\_BCD} & \texttt{0x1000}
    & Vector packed-decimal facility installed \\
    \texttt{HWCAP\_S390\_VXRS\_EXT} & \texttt{0x2000}
    & Vector-enhancements facility 1 installed \\
    \texttt{HWCAP\_S390\_GS} & \texttt{0x4000}
    & Guarded-storage facility installed \\
    \texttt{HWCAP\_S390\_VXRS\_EXT2} & \texttt{0x8000}
    & Vector-enhancements facility 2 installed \\
    \texttt{HWCAP\_S390\_VXRS\_PDE} & \texttt{0x10000}
    & Vector-packed-decimal enhancement facility installed \\
    \texttt{HWCAP\_S390\_DFLT} & \texttt{0x40000}
    & Deflate-conversion facility installed \\
    \bottomrule
  \end{tabularx}
  \end{DIFnomarkup}
  \caption{Hardware capabilities}
  \label{tab:hwcap}
\end{table}

\texttt{a\_type} auxiliary vector types are described in the
following:
\begin{description}
\item[\texttt{AT\_NULL}] The auxiliary vector has no fixed length, so
  an entry of this type is used to denote the end of the vector.  The
  corresponding value of \texttt{a\_un} is undefined.
\item[\texttt{AT\_IGNORE}] This type indicates the entry has no
  meaning.  The corresponding value of \texttt{a\_un} is undefined.
\item[\texttt{AT\_EXECFD}] \texttt{exec} may pass control to an interpreter
  program.  When this happens, the system places either an entry of
  type \texttt{AT\_EXECFD} or one of type \texttt{AT\_PHDR} in the
  auxiliary vector.  The \texttt{a\_val} field in the
  \texttt{AT\_EXECFD} entry contains a file descriptor for the
  application program's object file.
\item[\texttt{AT\_PHDR}] Under some conditions, the system creates the
  memory image of the application program before passing control to an
  interpreter program.  When this happens, the \texttt{a\_ptr} field of
  the \texttt{AT\_PHDR} entry tells the interpreter where to find the
  program header table in the memory image.  If the \texttt{AT\_PHDR}
  entry is present, entries of types \texttt{AT\_PHENT},
  \texttt{AT\_PHNUM}, and \texttt{AT\_ENTRY} must also be present.  See
  \cref{chprogload} for more information about the program header
  table.
\item[\texttt{AT\_PHENT}] The \texttt{a\_val} field of this entry
  holds the size, in bytes, of one entry in the program header table
  at which the \texttt{AT\_PHDR} entry points.
\item[\texttt{AT\_PHNUM}] The \texttt{a\_val} field of this entry
  holds the number of entries in the program header table at which the
  \texttt{AT\_PHDR} entry points.
\item[\texttt{AT\_PAGESZ}] If present, this entry's \texttt{a\_val}
  field gives the system page size in bytes.  The same information is
  also available through \texttt{sysconf}.
\item[\texttt{AT\_BASE}] The \texttt{a\_ptr} member of this entry
  holds the base address at which the interpreter program was loaded
  into memory.
\item[\texttt{AT\_FLAGS}] If present, the \texttt{a\_val} field of
  this entry holds 1-bit flags.  Undefined bits are set to zero.
\item[\texttt{AT\_ENTRY}] The \texttt{a\_ptr} field of this entry
  holds the entry point of the application program to which the
  interpreter program should transfer control.
\item[\texttt{AT\_NOTELF}] The \texttt{a\_val} field of this entry is
  non-zero if the program is in another format than ELF, for example
  in the old COFF format.
\item[\texttt{AT\_UID}] The \texttt{a\_ptr} field of this entry holds
  the real user id of the process.
\item[\texttt{AT\_EUID}] The \texttt{a\_ptr} field of this entry holds
  the effective user id of the process.
\item[\texttt{AT\_GID}] The \texttt{a\_ptr} field of this entry holds
  the real group id of the process.
\item[\texttt{AT\_EGID}] The \texttt{a\_ptr} field of this entry holds
  the effective group id of the process.
\item[\texttt{AT\_PLATFORM}] The \texttt{a\_ptr} field of this entry holds
  the address of a string that identifies the platform the program runs
  on.
\item[\texttt{AT\_HWCAP}] The \texttt{a\_val} field of this entry holds a
  bit map of hardware capabilities\index{hardware capabilities} hints.
  \Cref{tab:hwcap} lists some of the assigned bits and their meaning.
\item[\texttt{AT\_CLKTCK}] The \texttt{a\_val} field of this entry holds
  the number of clock ticks per second.  The function \texttt{times()},
  which measures execution time, reports all times in clock ticks.  The
  number of clock ticks per second is also available through
  \texttt{sysconf}.
\item[\texttt{AT\_SECURE}] The \texttt{a\_val} field of this entry holds a
  Boolean that indicates whether the program shall be locked into a secure
  environment, such as when access rights have been upgraded by executing
  a setuid/setgid executable.
\item[\texttt{AT\_RANDOM}] The \texttt{a\_ptr} field of this entry holds
  the address of 16 random bytes.
\item[\texttt{AT\_EXECFN}] The \texttt{a\_ptr} field of this entry holds
  the address of a string that contains the executable's file name.
\item[\texttt{AT\_SYSINFO\_EHDR}] The \texttt{a\_ptr} field of this entry
  holds the address at which the system-supplied dynamic shared object
  (DSO), specifically its ELF header, is mapped in the program's virtual
  address space.
\end{description}

Other auxiliary vector types are reserved.  No flags are currently
defined for \texttt{AT\_FLAGS} on \ABINAME{}.

\section{Coding Examples}
\label{codingexamples}
This section describes example code sequences for fundamental
operations such as calling functions, accessing static objects, and
transferring control from one part of a program to another.  Previous
sections discussed how a program may use the machine or the operating
system, and they specified what a program may and may not assume about
the execution environment.  Unlike previous material, the information
in this section illustrates how operations \emph{may} be done,
not how they \emph{must} be done.

As before, examples use the ISO C language.  Other programming
languages may use the same conventions displayed below, but failure to
do so does not prevent a program from conforming to the ABI\@.  Two main
object code models are available:
\begin{description}
\item[Absolute code:] Instructions can hold absolute addresses under
  this model.  To execute properly, the program must be loaded at a
  specific virtual address, making the program's absolute addresses
  coincide with the process's virtual addresses.
\item[Position-independent code:] Instructions under this model hold
  relative addresses, not absolute addresses.  Consequently, the code
  is not tied to a specific load address, allowing it to execute
  properly at various positions in virtual memory.
\end{description}

The following sections describe the differences between these models.
When different, code sequences for the models appear together for
easier comparison.

\paragraph{Note:}
The examples below show code fragments with various simplifications.
They are intended to explain addressing modes, not to show optimal
code sequences or to reproduce compiler output.

\subsection{Code Model Overview}
When the system creates a process image, the executable file
portion of the process has fixed addresses and the system chooses
shared object library virtual addresses to avoid conflicts with other
segments in the process.  To maximize text sharing, shared objects
conventionally use position-independent code, in which instructions
contain no absolute addresses.  Shared object text segments can be
loaded at various virtual addresses without having to change the
segment images.  Thus multiple processes can share a single shared
object text segment, even if the segment resides at a different
virtual address in each process.

Position-independent code relies on two techniques:
\begin{itemize}
\item Control transfer instructions hold addresses relative to the
  Current Instruction Address (CIA), or use registers that hold the
  transfer address.  A CIA-relative branch computes its destination
  address in terms of the CIA, not relative to any absolute address.
\item When the program requires an absolute address, it computes the
  desired value.  Instead of embedding absolute addresses in
  instructions (in the text segment), the compiler generates code to
  calculate an absolute address (in a register or in the stack or data
  segment) during execution.
\end{itemize}

Because \ARCHarch{}
provides CIA-relative branch instructions and also branch instructions
using registers that hold the transfer address, compilers can satisfy
the first condition easily.

A Global Offset Table (GOT) provides information for address
calculation.  Position-independent object files (executable and shared
object files) have a table in their data segment that holds
addresses.  When the system creates the memory image for an object
file, the table entries are relocated to reflect the absolute virtual
address as assigned for an individual process.  Because data segments
are private for each process, the table entries can change---unlike
those of text segments, which multiple processes share.

Two position-independent models give programs a choice between more
efficient code with some size restrictions and less efficient code
without those restrictions.  Because of the processor architecture, a
GOT with no more than {\ifzseries 512\else 1024\fi} entries (4096
bytes) is more efficient than a larger one.  Programs that need more
entries must use the larger, more general code.  In the following
sections, the term ``small model position-independent code'' is used
to refer to code that assumes the smaller GOT, and ``large model
position-independent code'' is used to refer to the general code.

\subsection{Function Prologue and Epilogue}
This section describes the prologue and epilogue code of functions.  A
function's prologue establishes a stack frame, if necessary, and may
save any nonvolatile registers it uses.  A function's epilogue generally
restores registers that were saved in the prologue code, restores the
previous stack frame, and returns to the caller.

\subsubsection{Prologue}
The prologue of a function has to save the state of the calling
function and set up the base register for the code of the function
body.  The following is in general done by the function
prologue:
\begin{itemize}
\item Save all registers used within the function which the calling
  function assumes to be nonvolatile.
\item Set up the base register for the literal pool, if needed.
\item Allocate stack space by decrementing the stack pointer.
\item Set up the dynamic chain by storing the old stack pointer value
  at stack location zero if the ``back chain'' is implemented.
\item Set up the GOT pointer if the compiler is generating
  position-independent code.

  (Usually the GOT pointer is loaded into a nonvolatile register.  This
  may be omitted if the function makes no external data references.  If
  external data references are only made within conditional code, loading
  the GOT pointer may be deferred until it is known to be needed.)
\item Set up the frame pointer if the function allocates stack space
  dynamically (with \texttt{alloca}).
\end{itemize}

The compiler tries to do as little as possible of the above; the ideal
case is to do nothing at all (for a leaf function without symbolic
references).

\ifzseries
\begin{lstlisting}[language=simpleasm,style=float,label=lst:prolcode,
  caption={[Prologue and epilogue example]{Prologue and epilogue example.
      This example stores the optional backchain.}}]
          .section .rodata
          .align  2
.LC0:     .string "hello, world!"

          .text
          .align  8
          .globl  main
          .type   main, @function
main:
                                        # Prologue
          stmg    %r14,%r15,112(%r15)   # Save caller's registers
          lgr     %r1,%r15              # Load stack pointer into r1
          aghi    %r15,-160             # Allocate new stack frame
          stg     %r1,0(%r15)           # Store back chain
                                        # Prologue end
          larl    %r2,.LC0
          brasl   %r14,puts
          lghi    %r2,0
                                        # Epilogue
          lmg     %r14,%r15,272(%r15)   # Restore registers
          br      %r14                  # Branch back to caller
                                        # Epilogue end
\end{lstlisting}
\else
\begin{lstlisting}[language=simpleasm,style=float,label=lst:prolcode,
  caption=Prologue and epilogue example]
          .string "hello, world\n"
          .align  4
          .globl  main
          .type   main,@function
main:
                                       # Prologue
          STM     11,15,44(15)         # Save callers registers
          BRAS    13,.LTN0_0           # Set up literal pool
                                       #   and branch over
.LT0_0:
.LC21:
          .long   .LC18
.LC22:
          .long   printf
.LTN0_0:
          LR      1,15                 # Load stack pointer in GPR 1
          AHI     15,-96               # Allocate stack space
          ST      1,0(15)              # Save backchain
                                       # Prologue end
          L       2,.LC21-.LT0_0(13)
          L       1,.LC22-.LT0_0(13)
          BASR    14,1
          SLR     2,2
                                       # Epilogue
          L       4,152(15)            # Load return address
          LM      11,15,140(15)        # Restore registers
          BR      4                    # Branch back to caller
                                       # Epilogue end
\end{lstlisting}
\fi

\subsubsection{Epilogue}
The epilogue of a function restores the registers saved in the prologue
(which include the stack pointer) and branches to the return address.

The small program in \cref{lst:prolcode} shows a simple example of a
function prologue and epilogue.

\subsection{Profiling}
\index{profiling}
This section shows a way of providing profiling (entry counting) for
\ABINAME{} applications.  An ABI-conforming system is not required to
provide profiling; however, if it does, this is one possible (not
required) implementation.

If a function is to be profiled, it has to call the \texttt{\_mcount}
routine before the function prologue.  This routine has a special linkage.
Its return address is passed in \texttt{r14} as usual.  However, instead
of register arguments it receives the caller's return address in the first
slot of the register save area, which is located \NBYTES{} bytes above the
current stack pointer.  And it preserves more registers than a normal
function, treating all the usual argument registers as nonvolatile as
well.  Since \texttt{\_mcount} gets invoked before the caller's prologue,
no additional frame needs to be allocated for it.  It may overwrite the
caller's register save area, except for the first slot, which it will
preserve.

\Cref{lst:profcode} shows an example of a function prologue preceded by a
call to \texttt{\_mcount}.

\ifzseries
\begin{lstlisting}[language=simpleasm,style=float,label=lst:profcode,
  caption=Code for profiling]
          stg     %r14,8(%r15)          # Pass r14 in first regsave slot
          brasl   %r14,_mcount          # Branch to _mcount
          lg      %r14,8(%r15)          # Restore r14
          stmg    %r7,%r15,56(%r15)     # Save caller's registers
          aghi    %r15,-160             # Allocate new frame
          ...
\end{lstlisting}
\else
\begin{lstlisting}[language=simpleasm,style=float,label=lst:profcode,
  caption=Code for profiling]
          STM     7,15,28(15)          # Save callers registers
          BRAS    13,.LTN0_0           # Jump to function prologue
.LT0_0:
.LC3:     .long   _mcount              # Literal pool entry for _mcount
.LC4:     .long   .LP0                 # Literal pool entry
                                       #   for profile counter
.LTN0_0:
          LR      1,15                 # Stack pointer
          AHI     15,-96               # Allocate new
          ST      1,0(15)              # Save backchain
          LR      11,15                # Local stack pointer
          .data
          .align 4
.LP0:     .long   0                    # Profile counter
          .text
                                       # Function profiler
          ST    14,4(15)               # Preserve r14
          L     14,.LC3-.LT0_0(13)     # Load address of _mcount
          L     1,.LC4-.LT0_0(13)      # Load address of profile counter
          BASR  14,14                  # Branch to _mcount
          L     14,4(15)               # Restore r14
\end{lstlisting}
\fi

\subsection{Data Objects}
This section describes only objects with static storage duration.  It
excludes stack-resident objects because programs always compute their
virtual addresses relative to the stack or frame pointers.

% TODO: The use of literal pool entries for relative symbols is outdated.
% Better describe the modern approach here.
Because \ARCH{} instructions cannot hold \ADDRBITS{}-bit addresses
directly, a program has to build an address in a register and access
memory through that register.  In order to do so, a function may contain a
literal pool that holds the addresses of data objects used by the
function.  Then \texttt{r13} is typically set up in the function prologue
to point to the start of this literal pool.

Position-independent code cannot contain absolute addresses.  In order
to access a local symbol, the literal pool contains the (signed) offset
of the symbol relative to the start of the pool.  Combining the offset
loaded from the literal pool with the address in \texttt{r13} gives the
absolute address of the local symbol.  In the case of a global symbol
the address of the symbol has to be loaded from the Global Offset
Table.  The offset in the GOT can either be contained in the
instruction itself or in the literal pool.

\Crefrange{tab:addresses}{tab:largegot} show sample assembly
language equivalents to C language code for absolute and
position-independent compilations.  It is assumed that all shared
objects are compiled as position-independent and only executable
modules may have absolute addresses.  The
function prologue is not shown, and it is assumed that it has loaded the
address of the literal pool in \texttt{r13}.

\begin{table}
  \centering
  \begin{DIFnomarkup}
  \begin{tabular}{p{0.35\textwidth}p{0.60\textwidth}}
    \toprule
    C & \ARCH{} machine instructions (Assembler) \\
    \midrule
\begin{lstlisting}[style=short]
extern int src;
extern int dst;
extern int *ptr;
dst = src;
ptr = &dst;
\end{lstlisting}
    &
\ifzseries
\begin{lstlisting}[style=short,language=simpleasm]
larl  %r1,src
larl  %r2,dst
larl  %r3,ptr
mvc   0(4,%r2),0(%r1)   # dst = src
stg   %r2,0(%r3)        # ptr = &dst
\end{lstlisting}
\else
\begin{lstlisting}[style=short,language=simpleasm]
           # Literal pool
.LT0:
.LC1:      .long dst
.LC2:      .long src
           # Code
           L     2,.LC1-.LT0(13)
           L     1,.LC2-.LT0(13)
           MVC   0(4,2),0(1)
           # Literal pool
.LT0:
.LC1:      .long ptr
.LC2:      .long dst
           # Code
           L     1,.LC1-.LT0(13)
           MVC   0(4,1),.LC2-.LT0(13)
           # Literal pool
.LT0:
.LC1:      .long ptr
.LC2:      .long src
           # Code
           L     1,.LC1-.LT0(13)
           L     2,.LC2-.LT0(13)
           L     3,0(1)
           MVC 0(4,3),0(2)
\end{lstlisting}
\fi \\
    \bottomrule
  \end{tabular}
  \end{DIFnomarkup}
  \caption{Absolute addressing}
  \label{tab:addresses}
\end{table}

\begin{table}
  \centering
  \begin{DIFnomarkup}
  \begin{tabular}{p{0.35\textwidth}p{0.60\textwidth}}
    \toprule
    C & \ARCH{} machine instructions (Assembler) \\
    \midrule
\begin{lstlisting}[style=short]
extern int src;
extern int dst;
extern int *ptr;
dst = src;
ptr = &dst;
*ptr = src;
\end{lstlisting}
    &
\ifzseries
\begin{lstlisting}[style=short,language=simpleasm]
larl  %r12,_GLOBAL_OFFSET_TABLE_
lg    %r1,dst@GOT12(%r12)
lg    %r2,src@GOT12(%r12)
lgf   %r3,0(%r2)
st    %r3,0(%r1)
larl  %r12,_GLOBAL_OFFSET_TABLE_
lg    %r1,ptr@GOT12(%r12)
lg    %r2,dst@GOT12(%r12)
stg   %r2,0(%r1)
larl  %r12,_GLOBAL_OFFSET_TABLE_
lg    %r2,ptr@GOT12(%r12)
lg    %r1,0(%r2)
lg    %r2,src@GOT12(%r12)
lgf   %r3,0(%r2)
st    %r3,0(%r1)
\end{lstlisting}
\else
\begin{lstlisting}[style=short,language=simpleasm]
           # Literal pool
.LT0:
.LC1:      .long _GLOBAL_OFFSET_TABLE_-.LT0
           # Code
           L     12,.LC1-.LT0(13)
           LA    12,0(12,13)
           L     2,dst@GOT(12)
           L     1,src@GOT(12)
           MVC   0(4,2),0(1)
           # Literal pool
.LT0:
.LC1:      .long _GLOBAL_OFFSET_TABLE_-.LT0
           # Code
           L     12,.LC1-.LT0(13)
           LA    12,0(12,13)
           L     1,ptr@GOT(12)
           L     2,dst@GOT(12)
           ST    2,0(1)
           # Literal pool
.LT0:
.LC1:      .long _GLOBAL_OFFSET_TABLE_-.LT0
           # Code
           L     12,.LC1-.LT0(13)
           LA    12,0(12,13)
           L     1,ptr@GOT(12)
           L     2,src@GOT(12)
           L     3,0(1)
           MVC 0(4,3),0(2)
\end{lstlisting}
\fi \\
    \bottomrule
  \end{tabular}
  \end{DIFnomarkup}
  \caption{Small model position-independent addressing}
\end{table}

\begin{table}
  \centering
  \begin{DIFnomarkup}
  \begin{tabular}{p{0.35\textwidth}p{0.60\textwidth}}
    \toprule
    C & \ARCH{} Assembler \\
    \midrule
\begin{lstlisting}[style=short]
extern int src;
extern int dst;
extern int *ptr;
dst = src;
ptr = &dst;
*ptr = src;
\end{lstlisting}
    &
\ifzseries
\begin{lstlisting}[style=short,language=simpleasm]
larl  %r2,dst@GOT
lg    %r2,0(%r2)
larl  %r3,src@GOT
lg    %r3,0(%r3)
mvc   0(4,%r2),0(%r3)
larl  %r2,ptr@GOT
lg    %r2,0(%r2)
larl  %r3,dst@GOT
lg    %r3,0(%r3)
stg   %r3,0(%r2)
larl  %r2,ptr@GOT
lg    %r2,0(%r2)
larl  %r3,src@GOT
lg    %r3,0(%r3)
mvc   0(4,%r3),0(%r2)
\end{lstlisting}
\else
\begin{lstlisting}[style=short,language=simpleasm]
           # Literal pool
.LT0:
.LC1:      .long dst@GOT
.LC2:      .long src@GOT
.LC3:      .long _GLOBAL_OFFSET_TABLE_-.LT0
           # Code
           L     12,.LC3-.LT0(13)
           LA    12,0(12,13)
           L     2,.LC1-.LT0(13)
           L     1,.LC2-.LT0(13)
           L     2,0(2,12)
           L     1,0(1,12)
           MVC   0(4,2),0(1)
           # Literal pool
.LT0:
.LC1:      .long ptr@GOT
.LC2:      .long dst@GOT
.LC3:      .long _GLOBAL_OFFSET_TABLE_-.LT0
           # Code
           L     12,.LC3-.LT0(13)
           LA    12,0(12,13)
           L     2,.LC1-.LT0(13)
           L     1,.LC2-.LT0(13)
           L     2,0(2,12)
           L     1,0(1,12)
           ST    1,0(2)
           # Literal pool
.LT0:
.LC1:      .long ptr@GOT
.LC2:      .long src@GOT
.LC3:      .long _GLOBAL_OFFSET_TABLE_-.LT0
           # Code
           L   12,.LC1-.LT0(13)
           LA  12,0(12,13)
           L   1,.LC1-.LT0(13)
           L   2,.LC2-.LT0(13)
           L   1,0(1,12)
           L   2,0(2,12)
           L   3,0(1)
           MVC 0(4,3),0(2)
\end{lstlisting}
\fi \\
    \bottomrule
  \end{tabular}
  \end{DIFnomarkup}
  \caption{Large model position-independent addressing}
  \label{tab:largegot}
\end{table}

\subsection{Function Calls}
Programs can use the \ARCH{} {\ifzseries\texttt{BRASL}\else
  \texttt{BRAS}\fi} instruction to make direct function calls.
A {\ifzseries \texttt{BRASL}\else \texttt{BRAS}\fi} instruction has a
self-relative branch displacement that can reach {\ifzseries
  4$\,$GBytes\else 64$\,$Kbytes\fi} in either direction.  {\ifzseries
  To call functions beyond this limit (inter-module calls),\else Hence
  the use of the \texttt{BRAS} instruction is limited to very rare
  cases.  The usual method of calling a function is to\fi} load the
address in a register and use the \texttt{BASR} instruction for the
call.  Register \texttt{r14} is used as the first operand of \texttt{BASR}
to hold the return address as shown in \cref{tab:fncalldirect}.

The called function may be in the same module (executable or shared
object) as the caller, or it may be in a different module.  In the
former case, if the called function is not in a shared object, the
linkage editor resolves the symbol.  In all other cases the linkage
editor cannot directly resolve the symbol.  Instead the linkage editor
generates ``glue'' code and resolves the symbol to point to the glue
code.  The dynamic linker will provide the real address of the
function in the Global Offset Table.  The glue code loads this address
and branches to the function itself.  See
\cref{procedurelinkagetable} for more details.

\begin{table}
  \centering
  \begin{DIFnomarkup}
  \begin{tabular}{p{0.35\textwidth}p{0.60\textwidth}}
    \toprule
    C & \ARCH{} machine instructions (Assembler) \\
    \midrule
\begin{lstlisting}[style=short]
extern void func();
extern void (*ptr)();
ptr = func;
func();
(*ptr) ();
\end{lstlisting}
    &
\ifzseries
\begin{lstlisting}[style=short,language=simpleasm]
larl  %r1,ptr
larl  %r2,func
stg   %r2,0(%r1)
brasl %r14,func
larl  %r1,ptr
lg    %r1,0(%r1)
basr  %r14,%r1
\end{lstlisting}
\else
\begin{lstlisting}[style=short,language=simpleasm]
           # Literal pool
.LT0:
.LC1:      .long ptr
.LC2:      .long func
           # Code
           L     1,.LC1-.LT0(13)
           MVC   0(4,1),.LC2-.LT0(13)
           # Literal pool
.LT0:
.LC1:      .long func
           # Code
           L     1,.LC1-.LT0(13)
           BASR  14,1
           # Literal pool
.LT0:
.LC1:      .long ptr
           # Code
           L     1,.LC1-.LT0(13)
           L     1,0(1)
           BASR  14,1
\end{lstlisting}
\fi \\
    \bottomrule
  \end{tabular}
  \end{DIFnomarkup}
  \caption{Absolute {\ifzseries\else direct\fi} function call}
  \label{tab:fncalldirect}
\end{table}

\begin{table}
  \centering
  \begin{DIFnomarkup}
  \begin{tabular}{p{0.35\textwidth}p{0.60\textwidth}}
    \toprule
    C & \ARCH{} machine instructions (Assembler) \\
    \midrule
\begin{lstlisting}[style=short]
extern void func();
extern void (*ptr)();
ptr = func;
func();
(*ptr) ();
\end{lstlisting}
    &
\ifzseries
\begin{lstlisting}[style=short,language=simpleasm]
larl  %r12,_GLOBAL_OFFSET_TABLE_
lg    %r1,ptr@GOT12(%r12)
lg    %r2,func@GOT12(%r12)
stg   %r2,0(%r1)
brasl %r14,func@PLT
larl  %r12,_GLOBAL_OFFSET_TABLE_
lg    %r1,ptr@GOT12(%r12)
lg    %r1,0(%r1)
basr  %r14,%r1
\end{lstlisting}
\else
\begin{lstlisting}[style=short,language=simpleasm]
           # Literal pool
.LT0:
.LC1:      .long _GLOBAL_OFFSET_TABLE_-.LT0
           # Code
           L     12,.LC1-.LT0(13)
           LA    12,0(12,13)
           L     1,ptr@GOT(12)
           L     2,func@GOT(12)
           ST    2,0(1)
           # Literal pool
.LT0:
.LC1:      .long _GLOBAL_OFFSET_TABLE_-.LT0
.LC2:      .long func@PLT-.LT0
           # Code
           L     12,.LC1-.LT0(13)
           LA    12,0(12,13)
           L     1,.LC2-.LT0(13)
           BAS   14,0(1,13)
           # Literal pool
.LT0:
.LC1:      .long _GLOBAL_OFFSET_TABLE_-.LT0
           # Code
           L     12,.LC1-.LT0(13)
           LA    12,0(12,13)
           L     1,ptr@GOT(12)
           L     2,0(1)
           BASR  14,2
\end{lstlisting}
\fi \\
    \bottomrule
  \end{tabular}
  \end{DIFnomarkup}
  \caption{Small model position-independent {\ifzseries\else direct\fi}
    function call}
  \label{tab:fnsmalldirect}
\end{table}

\begin{table}
  \centering
  \begin{DIFnomarkup}
  \begin{tabular}{p{0.35\textwidth}p{0.60\textwidth}}
    \toprule
    C & \ARCH{} machine instructions (Assembler) \\
    \midrule
\begin{lstlisting}[style=short]
extern void func();
extern void (*ptr)();
ptr = func;
func();
(*ptr) ();
\end{lstlisting}
    &
\ifzseries
\begin{lstlisting}[style=short,language=simpleasm]
larl  %r2,ptr@GOT
lg    %r2,0(%r2)
larl  %r3,func@GOT
lg    %r3,0(%r3)
stg   %r3,0(%r2)
brasl %r14,func@PLT
larl  %r2,ptr@GOT
lg    %r2,0(%r2)
lg    %r2,0(%r2)
basr  %r14,%r2
\end{lstlisting}
\else
\begin{lstlisting}[style=short,language=simpleasm]
           # Literal pool
.LT0:
.LC1:      .long ptr@GOT
.LC2:      .long func@GOT
.LC3:      .long _GLOBAL_OFFSET_TABLE_-.LT0
           # Code
           L     12,.LC3-.LT0(13)
           LA    12,0(12,13)
           L     2,.LC1-.LT0(13)
           L     1,.LC2-.LT0(13)
           L     2,0(2,12)
           L     1,0(1,12)
           ST    1,0(2)
           # Literal pool
.LT0:
.LC1:      .long _GLOBAL_OFFSET_TABLE_-.LT0
.LC2:      .long func@PLT-.LT0
           # Code
           L     12,.LC1-.LT0(13)
           LA    12,0(12,13)
           L     1,.LC2-.LT0(13)
           BAS   14,0(1,13)
           # Literal pool
.LT0:
.LC1:      .long ptr@GOT
.LC2:      .long _GLOBAL_OFFSET_TABLE_-.LT0
           # Code
           L     12,.LC2-.LT0(13)
           LA    12,0(12,13)
           L     1,.LC1-.LT0(13)
           L     1,0(1,12)
           L     2,0(1)
           BASR  14,2
\end{lstlisting}
\fi \\
    \bottomrule
  \end{tabular}
  \end{DIFnomarkup}
  \caption{Large model position-independent {\ifzseries\else direct
      \fi}function call}
  \label{tab:fnlargedirect}
\end{table}

\ifzseries\else
\begin{table}
  \centering
  \begin{DIFnomarkup}
  \begin{tabular}{p{0.35\textwidth}p{0.60\textwidth}}
    \toprule
    C & \ARCH{} machine instructions (Assembler) \\
    \midrule
\begin{lstlisting}[style=short]
extern void func();
extern void (*ptr)();
ptr = func;
func();
(*ptr) ();
\end{lstlisting}
    &
\begin{lstlisting}[style=short,language=simpleasm]
          # Literal pool
.LT0:
.LC1:     .long ptr
.LC2:     .long func
          # Code
          L     1,.LC1-.LT0(13)
          MVC   0(4,1),.LC2-.LT0(13)
          # Literal pool
.LT0:
.LC1:     .long ptr
          # Code
          L     1,.LC1-.LT0(13)
          L     1,0(1)
          BASR  14,1
\end{lstlisting} \\
  \end{tabular}
  \end{DIFnomarkup}
  \caption{Absolute indirect function call}
  \label{tab:fncallabsindirect}
\end{table}
\fi

\ifzseries\else
\begin{table}
  \centering
  \begin{DIFnomarkup}
  \begin{tabular}{p{0.35\textwidth}p{0.60\textwidth}}
    \toprule
    C & \ARCH{} machine instructions (Assembler) \\
    \midrule
\begin{lstlisting}[style=short]
extern void func();
extern void (*ptr)();
ptr = func;
func();
(*ptr) ();
\end{lstlisting}
    &
\begin{lstlisting}[style=short,language=simpleasm]
           # Literal pool
.LT0:
.LC1:      .long _GLOBAL_OFFSET_TABLE_-.LT0
           # Code
           L     12,.LC2-.LT0(13)
           LA    12,0(12,13)
           L     1,ptr@GOT(12)
           L     2,func@GOT(12)
           ST    2,0(1)
           # Literal pool
.LT0:
.LC1:      .long _GLOBAL_OFFSET_TABLE_-.LT0
           # Code
           L     12,.LC1-.LT0(13)
           LA    12,0(12,13)
           L     1,ptr@GOT(12)
           L     2,0(1)
           BASR  14,2
\end{lstlisting} \\
  \end{tabular}
  \end{DIFnomarkup}
  \caption{Small model position-independent indirect function call}
  \label{tab:fncallpicsmall}
\end{table}
\fi

\ifzseries\else
\begin{table}
  \centering
  \begin{DIFnomarkup}
  \begin{tabular}{p{0.35\textwidth}p{0.60\textwidth}}
    \toprule
    C & \ARCH{} machine instructions (Assembler) \\
    \midrule
\begin{lstlisting}[style=short]
extern void func();
extern void (*ptr)();
ptr = func;
func();
(*ptr) ();
\end{lstlisting}
    &
\begin{lstlisting}[style=short,language=simpleasm]
           # Literal pool
.LT0:
.LC1:      .long ptr@GOT
.LC2:      .long func@GOT
.LC3:      .long _GLOBAL_OFFSET_TABLE_-.LT0
           # Code
           L     12,.LC3-.LT0(13)
           LA    12,0(12,13)
           L     2,.LC1-.LT0(13)
           L     1,.LC2-.LT0(13)
           L     2,0(2,12)
           L     1,0(1,12)
           ST    1,0(2)
           # Literal pool
.LT0:
.LC1:      .long ptr@GOT
.LC2:      .long _GLOBAL_OFFSET_TABLE_-.LT0
           # Code
           L     12,.LC2-.LT0(13)
           LA    12,0(12,13)
           L     1,.LC1-.LT0(13)
           L     1,0(1,12)
           L     2,0(1)
           BASR  14,2
\end{lstlisting} \\
  \end{tabular}
  \end{DIFnomarkup}
  \caption{Large model position-independent indirect function call}
  \label{tab:fncallpiclarge}
\end{table}
\fi

\subsection{Branching}
Programs use branch instructions to control their execution flow.
{\ifzseries \ARCH{}\else The \ARCH{} architecture\fi} has a
variety of branch instructions.  The most commonly used of these
performs a self-relative jump with a 128-Kbyte range (up to 64 Kbytes
in either direction).  {\ifzseries For large functions, another
  self-relative jump is available with a range of 4$\,$Gbytes (up to
  2$\,$Gbytes in either direction).\fi}%

\begin{table}
  \centering
  \begin{DIFnomarkup}
  \begin{tabular}{p{0.35\textwidth}p{0.60\textwidth}}
    \toprule
    C & \ARCH{} machine instructions (Assembler) \\
    \midrule
    \ifzseries
\begin{lstlisting}[style=short]
label:
        ...
        goto label;
        ...
        ...
        ...
farlabel:
        ...
        ...
        ...
        goto farlabel;
\end{lstlisting}
    &
\begin{lstlisting}[style=short,language=simpleasm]
.L01:
           ...
           j    .L01
           ...
           ...
           ...
.L02:
           ...
           ...
           ...
           jg   .L02
\end{lstlisting} \\
    \else
\begin{lstlisting}[style=short]
label:
        ...
        goto label;
\end{lstlisting}
    &
\begin{lstlisting}[style=short,language=simpleasm]
.L01:
           ...
           BRC 15,.L01
\end{lstlisting}
\fi \\
    \bottomrule
  \end{tabular}
  \end{DIFnomarkup}
  \caption{Branch instruction}
  \label{tab:branchinsn}
\end{table}

C language switch statements provide multi-way selection.  When the case
labels of a switch statement satisfy grouping constraints, the compiler
implements the selection with an address table.  The examples shown in
\cref{tab:absswitch,tab:indswitch} use several simplifying conventions to
hide irrelevant details:
\begin{enumerate}
\item The selection expression resides in \texttt{r2}.
\item The case label constants begin at zero.
\item The case labels, the default, and the address table use assembly
  names \texttt{.Lcasei}, \texttt{.Ldef}, and \texttt{.Ltab} respectively.
\end{enumerate}

\begin{table}
  \centering
  \begin{DIFnomarkup}
  \begin{tabular}{p{0.3\textwidth}p{0.65\textwidth}}
    \toprule
    C & \ARCH{} machine instructions (Assembler) \\
    \midrule
\begin{lstlisting}[style=short]
switch(j)
  {
  case 0:
    /* ... */
  case 1:
    /* ... */
  case 3:
    /* ... */
  default:
  }
\end{lstlisting}
    &
\ifzseries
\begin{lstlisting}[style=short,language=simpleasm]
           lghi  %r1,%r3
           clgr  %r2,%r1
           brc   2,.Ldef
           sllg  %r2,%r2,3
           larl  %r1,.Ltab
           lg    %r3,0(%r1,%r2)
           br    %r3
.Ltab:     .quad .Lcase0
           .quad .Lcase1
           .quad .Ldef
           .quad .Lcase3
\end{lstlisting}
\else
\begin{lstlisting}[style=short,language=simpleasm]
           # Literal pool
.LT0:
.LC1:      .long .Ltab
           # Code
           LHI    1,3
           CLR    2,1
           BRC    2,.Ldef
           SLL    2,2
           A      2,.LC1-.LT0(13)
           L      1,0(2)
           BR     1
.Ltab:     .long .Lcase0
           .long .Lcase1
           .long .Ldef
           .long .Lcase3
\end{lstlisting}
\fi \\
    \bottomrule
  \end{tabular}
  \end{DIFnomarkup}
  \caption{Absolute switch code}
  \label{tab:absswitch}
\end{table}

\begin{table}
  \centering
  \begin{DIFnomarkup}
  \begin{tabular}{p{0.3\textwidth}p{0.65\textwidth}}
    \toprule
    C & \ARCH{} machine instructions (Assembler) \\
    \midrule
\begin{lstlisting}[style=short]
switch(j)
  {
  case 0:
    /* ... */
  case 1:
    /* ... */
  case 3:
    /* ... */
  default:
  }
\end{lstlisting}
    &
\ifzseries
\begin{lstlisting}[style=short,language=simpleasm]
            # Literal pool
.LT0:
            # Code
            lghi  %r1,3
            clgr  %r2,%r1
            brc   2,.Ldef
            sllg  %r2,%r2,3
            larl  %r1,.Ltab
            lg    %r3,0(%r1,%r2)
            agr   %r3,%r13
            br    %r3
.Ltab:      .quad .Lcase0-.LT0
            .quad .Lcase1-.LT0
            .quad .Ldef-.LT0
            .quad .Lcase3-.LT0
\end{lstlisting}
\else
\begin{lstlisting}[style=short,language=simpleasm]
           # Literal pool
.LT0:
.LC1:      .long .Ltab-.LT0
           # Code
           LHI   1,3
           CLR   2,1
           BRC   2,.Ldef
           SLL   2,2
           L     1,.LC1-.LT0(13)
           LA    1,0(1,13)
           L     2,0(1,2)
           LA    2,0(2,13)
           BR    2
.Ltab:     .long .Lcase0-.LT0
           .long .Lcase1-.LT0
           .long .Ldef-.LT0
           .long .Lcase3-.LT0
\end{lstlisting}
\fi \\
    \bottomrule
  \end{tabular}
  \end{DIFnomarkup}
  \caption{Position-independent switch code, all models}
  \label{tab:indswitch}
\end{table}

\subsection{Dynamic Stack Space Allocation}
\label{dynamicstack}
The GNU C compiler, and most recent compilers, support dynamic stack
space allocation via \texttt{alloca}.

\Cref{fig:dynstackalloc} shows the stack frame before and after dynamic
stack allocation.  The local variables area is used for storage of
function data, such as local variables, whose sizes are known to the
compiler.  This area is allocated at function entry and does not
change in size or position during the function's activation.

The parameter area holds ``overflow'' arguments passed in calls to
other functions.  (See the \jumplabel{more} label in
\cref{parameterpassing}.)  Its size is also known to the compiler and can
be allocated along with the fixed frame area at function entry.  However,
the standard calling sequence requires that the parameter area begins
at a fixed offset (\STACKSIZE{}) from the stack pointer, so this area must
move when dynamic stack allocation occurs.

Data in the parameter area are naturally addressed at
constant offsets from the stack pointer.  However, in the presence of
dynamic stack allocation, the offsets from the stack pointer to the
data in the local-variable area are not constant.  To provide
addressability, a frame pointer is established to locate the local
variables area consistently throughout the function's
activation.

Dynamic stack allocation is accomplished by ``opening'' the stack
just above the parameter area.  The following steps show the
process in detail:

\begin{enumerate}
\item After a new stack frame is acquired, and before the
first dynamic space allocation, a new register, the frame pointer or
FP, is set to the value of the stack pointer.  The frame pointer is
used for references to the function's local, non-static variables.  The
frame pointer does not change during the execution of a function, even
though the stack pointer may change as a result of dynamic
allocation.

\item The amount of dynamic space to be allocated is rounded
up to a multiple of 8 bytes, so that 8-byte stack alignment is
maintained.

\item The stack pointer is decreased by the rounded byte
count, and the address of the previous stack frame (the back chain)
may be stored at the word addressed by the new stack pointer.  The back
chain is not necessary to restore from this allocation at the end of
the function since the frame pointer can be used to restore the stack
pointer.

\end{enumerate}

\Cref{fig:dynstackalloc} is a snapshot of the stack layout after the
prologue code has dynamically extended the stack frame.

\begin{figure}
  \centering
  \ifSkipTikZ
\begin{verbatim}
          Before Dynamic               After Dynamic
         Stack Allocation             Stack Allocation

       |  Previous stack  |         |  Previous stack  |
       |      frame       |         |      frame       |
       +------------------+         +------------------+
       |    Back chain    |         |    Back chain    |
       |    (optional)    |         |    (optional)    |
  .--> ====================    .--> ====================
 /     | Local and spill  |   /     | Local and spill  |
|      | variable area of |  |      | variable area of |
|      | calling function |  |      | calling function |
|160+n +------------------+  |      +------------------+ <---  FP+160+n
|      |  Parameter area  |  |      |                  |     ^
|      | passed to called |  |      |                  |     :
|      |    functions     |  |      |                  |     :
|  160 +------------------+  |      |                  |     :
|      |  Register save   |  |      |                  |     :
|      | area for called  |  |      |                  |     :
|      |    functions     |  |      |                  |     :
 \   8 +------------------+  |      |     Dynamic      |     :
  `----+    Back chain    |  |      | Allocation Area  |     :
       |    (optional)    |  |      |                  |     :
SP-> 0 ====================- | - - -|                  | <-- FP
                             |      |                  |
                             |      |                  |
                             |      |                  |
                             |      |                  |
                             |      |                  |
                             |160+n +------------------+
                             |      |  Parameter area  |
                             |      | passed to called |
                             |      |    functions     |
                             |  160 +------------------+
                             |      |  Register save   |
                             |      | area for called  |
                             |      |    functions     |
                              \   8 +------------------+
                               `----+    Back chain    |
                                    |    (optional)    |
                             SP-> 0 ====================
\end{verbatim}
  \else
  \pgfsetlayers{background,main}
  \begin{tikzpicture}
    \matrix [memory layout,inner sep=0pt,
    nodes={text width=10em, align=center, inner sep=1ex}] (ma)
    {
      \node (prev)     {Previous stack frame}; \\
      \node (prevback) {Back chain (optional)}; \\
      \node (aspill)    {Local and spill variable area of calling function}; \\
      \node [minimum height=9em] (dynalloc) {\vfill Dynamic Allocation Area\par\vfill}; \\
      \node [inactive layout] (param) {Parameter area passed to called functions}; \\
      \node [inactive layout] (save)  {Register save area for called functions}; \\
      \node (back)     {Back chain (optional)}; \\
    };
    \draw \foreach \Node in {prev, aspill, param, dynalloc, save} {
      (\Node.south -| ma.west) -- (\Node.south -| ma.east)
    };
    \begin{scope}[very thick, shorten <=-1ex, shorten >=-1ex]
      \draw (prevback.south -| ma.west) -- (prevback.south -| ma.east);
      \draw (ma.south west) -- (ma.south east);
    \end{scope}
    \path [every node/.style={left=1pt,fill=white,font={\footnotesize}}]
    (dynalloc.south -| ma.west) node {\texttt{~SP+\STACKSIZE{}+n}}
    (param.south -| ma.west) node {\texttt{SP+\STACKSIZE{}}}
    (save.south -| ma.west) node {\texttt{SP+{\ifzseries 16\else 8\fi}}};
    \path (ma.south west) node [left=2em] (sp) {\texttt{SP}};
    \draw [->, shorten >=1.5ex] (sp) -- (ma.south west);
    \draw [rounded corners, shorten >=1.5ex, ->]
    (back -| ma.west)  -- (back -| current bounding box.west)
    |- (prevback.south -| ma.west);
    % ---
    \path (current bounding box.north west) +(-2em,0)
    node [matrix, memory layout, inner sep=0pt, below left,
    nodes={text width=10em, align=center, inner sep=1ex}] (mb)
    {
      \node (prev)     {Previous stack frame}; \\
      \node (prevback) {Back chain (optional)}; \\
      \node (spill)    {Local and spill variable area of calling function}; \\
      \node [inactive layout] (param) {Parameter area passed to called functions}; \\
      \node [inactive layout] (save)  {Register save area for called functions}; \\
      \node (back)     {Back chain (optional)}; \\
    };
    \draw \foreach \Node in {prev, spill, param, save} {
      (\Node.south -| mb.west) -- (\Node.south -| mb.east)
    };
    \begin{scope}[very thick, shorten <=-1ex, shorten >=-1ex]
      \draw (prevback.south -| mb.west) -- (prevback.south -| mb.east);
      \draw (mb.south west) -- (mb.south east);
    \end{scope}
    \path [every node/.style={left=1pt,font={\footnotesize}}]
    (spill.south -| mb.west) node {\texttt{~SP+\STACKSIZE{}+n}}
    (param.south -| mb.west) node {\texttt{SP+\STACKSIZE{}}}
    (save.south -| mb.west) node {\texttt{SP+{\ifzseries 16\else 8\fi}}};
    \path (mb.south west) node [left=2em] (sp) {\texttt{SP}};
    \draw [->, shorten >=1.5ex] (sp) -- (mb.south west);
    \draw [rounded corners, shorten >=1.5ex, ->]
    (back -| mb.west)  -- (back -| current bounding box.west)
    |- (prevback.south -| mb.west);
    % ---
    \path [every node/.style={right=2em}]
    (back.south -| ma.east) node (fp) {\texttt{FP}}
    (aspill.south -| ma.east) node [font={\footnotesize}] (newfp)
    {\texttt{FP+\STACKSIZE{}+n}};
    \draw [->] (fp) -- (fp -| ma.east);
    \draw [->] (newfp) -- (newfp -| ma.east);
    \draw [loosely dotted, ->] (fp) -- (fp |- newfp.south);
    \begin{pgfonlayer}{background}
      \begin{scope}[dashed, thin]
        \draw (newfp -| ma.east) -- (spill.south -| mb.east);
        \draw (fp -| ma.east) -- (mb.south east);
      \end{scope}
    \end{pgfonlayer}
    % ---
    \path [every node/.style={above=1ex}]
    (ma.north) node {After Dynamic Stack Allocation}
    (mb.north) node {Before Dynamic Stack Allocation};
  \end{tikzpicture}
  \fi
  \caption{Dynamic stack space allocation}
  \label{fig:dynstackalloc}
\end{figure}

The above process can be repeated as many times as desired
within a single function activation.  When it is time to return, the
stack pointer is set to the value of the back chain, thereby removing
all dynamically allocated stack space along with the rest of the stack
frame.  Naturally, a program must not reference the dynamically
allocated stack area after it has been freed.

Even in the presence of signals, the above dynamic allocation
scheme is ``safe.'' If a signal interrupts allocation, one of three
things can happen:
\begin{itemize}
\item The signal handler can return.  The process then resumes the
  dynamic allocation from the point of interruption.
\item The signal handler can execute a non-local goto or a jump.  This
  resets the process to a new context in a previous stack frame,
  automatically discarding the dynamic allocation.
\item The process can terminate.
\end{itemize}

Regardless of when the signal arrives during dynamic allocation, the
result is a consistent (though possibly dead) process.

\section{DWARF Definition}
\index{DWARF}
This section defines the ``Debug With Attributed Record Format''
(DWARF) debugging format for \ARCH{} processors.  The
\ABINAME{} ABI does not define a debug format.  However, all systems that
do implement DWARF shall use the following definitions.

DWARF is a specification developed for symbolic source-level
debugging.  The debugging information format does not favor the design
of any compiler or debugger.

The DWARF definition requires some machine-specific definitions.  The
register number mapping\index{DWARF!register
  numbers}\index{registers!DWARF numbers} is specified for the \ARCH{}
processors in \cref{tab:dwarfreg}.

\begin{table}
  \centering
  \begin{DIFnomarkup}
  \begin{threeparttable}
    \begin{tabular}[t]{rl!{\qquad}rl}
      \toprule
      DWARF  & \ARCH{}  & DWARF  & \ARCH{} \\
      number & register & number & register \\
      \midrule
      0--15 & \texttt{r0}--\texttt{r15}
                        & 65 & PSW address \\
      16 & \texttt{f0} / \texttt{v0}
                        & 66 & \emph{reserved} (z/OS) \\
      17 & \texttt{f2} / \texttt{v2}
                        & 67 & \emph{reserved} (z/OS) \\
      18 & \texttt{f4} / \texttt{v4}
                        & 68 & \texttt{v16} \\
      19 & \texttt{f6} / \texttt{v6}
                        & 69 & \texttt{v18} \\
      20 & \texttt{f1} / \texttt{v1}
                        & 70 & \texttt{v20} \\
      21 & \texttt{f3} / \texttt{v3}
                        & 71 & \texttt{v22} \\
      22 & \texttt{f5} / \texttt{v5}
                        & 72 & \texttt{v17} \\
      23 & \texttt{f7} / \texttt{v7}
                        & 73 & \texttt{v19} \\
      24 & \texttt{f8} / \texttt{v8}
                        & 74 & \texttt{v21} \\
      25 & \texttt{f10} / \texttt{v10}
                        & 75 & \texttt{v23} \\
      26 & \texttt{f12} / \texttt{v12}
                        & 76 & \texttt{v24} \\
      27 & \texttt{f14} / \texttt{v14}
                        & 77 & \texttt{v26} \\
      28 & \texttt{f9} / \texttt{v9}
                        & 78 & \texttt{v28} \\
      29 & \texttt{f11} / \texttt{v11}
                        & 79 & \texttt{v30} \\
      30 & \texttt{f13} / \texttt{v13}
                        & 80 & \texttt{v25} \\
      31 & \texttt{f15} / \texttt{v15}
                        & 81 & \texttt{v27} \\
      32--47 & \texttt{cr0}--\texttt{cr15}\tnote{\dagger}
                        & 82 & \texttt{v29} \\
      48--63 & \texttt{a0}--\texttt{a15}
                        & 83 & \texttt{v31} \\
      64 & PSW mask \\
      \bottomrule
    \end{tabular}
    \medskip
    \begin{tablenotes}
    \item [\dagger] Control registers cannot be referenced by user-space
      applications.  They are reserved for use by operating system code.
    \end{tablenotes}
  \end{threeparttable}
  \end{DIFnomarkup}
  \caption{DWARF register number mapping}
  \label{tab:dwarfreg}
\end{table}

% FIXME: Should document the CFA and its offset.

For the placement of a piece within a composite location description, as
defined by the byte piece operation \texttt{DW\_OP\_piece} or the bit
piece operation \texttt{DW\_OP\_bit\_piece}, the following applies:
\begin{itemize}
\item Pieces of a floating-point or vector register are taken from the
  left.  This means that a bit piece with offset \(t\) and size \(n\)
  consists of the register's bits numbered from \(t\) to \(t+n-1\),
  according to big-endian bit numbering.  And a byte piece of a
  floating-point or vector register of size \(n\) consists of the
  register's \(n\) leftmost bytes.
\item For any other register, pieces are taken from the right.  This means
  that a bit piece with offset \(t\) and size \(n\) consists of the bits
  numbered from \(w-t-n\) to \(w-t-1\), where \(w\) is the register's bit
  width.  And a byte piece of size \(n\) consists of the register's \(n\)
  rightmost bytes.
\end{itemize}

Whenever interpreting a register as a given type, such as when using the
register value operation \texttt{DW\_OP\_regval\_type} or the register
location description \texttt{DW\_OP\_regx}, the resulting value consists
of the same bits as the bit piece starting at offset zero and having the
size of the given type.

\chapter{Object files}
\label{chobjfiles}
\index{ELF}
\index{object file}
This section describes the Executable and Linking Format (ELF).

\section{ELF Header}
\subsection{Machine Information}
For file identification in \texttt{e\_ident} the \ARCH{} processor
family requires the values shown in \cref{tab:eident}.

\begin{table}
  \centering
  \begin{DIFnomarkup}
  \begin{tabular}{lll}
    \toprule
    Position & Value & Comments \\
    \midrule
    \texttt{e\_ident[EI\_CLASS]} & \texttt{ELFCLASS\NBITS{}} &
    For all \NBITS{}$\,$bit implementations \\
    \texttt{e\_ident[EI\_DATA]} & \texttt{ELFDATA\NBITS{}MSB} &
    For all Big-Endian implementations \\
    \bottomrule
  \end{tabular}
  \end{DIFnomarkup}
  \caption{Machine-specific ELF identification fields}
  \label{tab:eident}
\end{table}

The ELF header's \texttt{e\_flags} field holds bit flags associated
with the file.  Since the \ARCH{} processor family defines no flags,
this member contains zero.

Processor identification resides in the ELF header's
\texttt{e\_machine} field and must have the value 22, defined as the
name \texttt{EM\_S390}.

\section{Sections}
\subsection{Special Sections}
Various sections hold program and control information.  The following
sections, whose types and attributes are listed in \cref{tab:sections},
are used by the system:
\begin{description}
\item[\texttt{.got}] \index{got section@\texttt{.got} section} This
  section holds the Global Offset Table, or GOT\@.  See
  \cref{codingexamples,globaloffsettable} for more information.
\item[\texttt{.plt}] \index{plt section@\texttt{.plt} section} This
  section holds the Procedure Linkage Table, or PLT.  See
  \cref{procedurelinkagetable} for more information.
\end{description}

\begin{table}
  \centering
  \begin{DIFnomarkup}
  \begin{tabular}{lll}
    \toprule
    Name & Type & Attributes \\
    \midrule
    \texttt{.got} & \texttt{SHT\_PROGBITS} &
    \texttt{SHF\_ALLOC + SHF\_WRITE} \\
    \texttt{.plt} & \texttt{SHT\_PROGBITS} &
    \texttt{SHF\_ALLOC + SHF\_WRITE + SHF\_EXECINSTR} \\
    \bottomrule
  \end{tabular}
  \end{DIFnomarkup}
  \caption{Special sections}
  \label{tab:sections}
\end{table}

\section{Symbol Table}
\index{symbol table}
\subsection{Symbol Values}
\label{symbolvalues}
A symbol table entry's \texttt{st\_value} field is the symbol value.  If
that value represents a section offset or a virtual address, it must be
halfword aligned.  This enables use of CIA-relative addressing
instructions such as \texttt{LARL}.

If an executable file contains a reference to a function defined in
one of its associated shared objects, the symbol table section for the
file will contain an entry for that symbol.  The \texttt{st\_shndx}
field of that symbol table entry contains \texttt{SHN\_UNDEF}.  This
informs the dynamic linker that the symbol definition for that
function is not contained in the executable file itself.  If that
symbol has been allocated a Procedure Linkage Table entry in the
executable file, and the \texttt{st\_value} field for that symbol
table entry is nonzero, the value is the virtual address of the first
instruction of that PLT entry.  Otherwise the \texttt{st\_value} field
contains zero.  This PLT entry address is used by the dynamic linker
in resolving references to the address of the function.  See
\cref{functionaddresses} for details.

\section{Relocation}
\index{relocation}
\subsection{Relocation Types}
Relocation entries describe how to alter the instruction and data
relocation fields listed below.  \Cref{fig:relocfields} illustrates the
affected bits of each field type.

\begin{figure}
  \centering
  \ifSkipTikZ
\begin{verbatim}
+-------+-------+-------+-------+-------+-------+-------+-------+
|0      |1      |2      |3      |4      |5      |6      |7      |
|                            quad64                             |
|0                                                            63|
+---------------------------------------------------------------+

+-------+-------+-------+-------+
|0      |1      |2      |3      |
|             word32            |
|0                            31|
+-------------------------------+

+-------+-------+-------+-------+
|0      |1      |2      |3      |
|              pc32             |
|0                            31|
+-------------------------------+

+-------+-------+-------+-------+
|0      |1      |2      |3      |
|       |      pc24             |
|0     7|8                    31|
+-------+-----------------------+

+----+--+-------+-------+-------+
|0   |  |1      |2      |3      |
|    |      mid20       |       |
|0  3|4               23|24   31|
+----+------------------+-------+

+-------+-------+
|0      |1      |
|     half16    |
|0            15|
+---------------+

+-------+-------+
|0      |1      |
|     pc16      |
|0            15|
+---------------+

+---+---+-------+
|0  |   |1      |
|   |   low12   |
|0 3|4        15|
+---+-----------+

+---+---+-------+
|0  |   |1      |
|   |    pc12   |
|0 3|4        15|
+---+-----------+

+-------+
|0      |
| byte8 |
|0     7|
+-------+
\end{verbatim}
  \else
  \begin{tikzpicture}[x=1.2ex,y=3em]
    \ifzseries
    \foreach \fields/\nbytes/\shifted in {
      {0 64/quad64/}/8/{0,0},
      {0 32/word32/}/4/{0,-1.3},
      {0 32/pc32/}/4/{0,-2.6},
      {0 8/ /,32/pc24/}/4/{0,-3.9},
      {0 4/ /,24/mid20/, 32/ /}/4/{0,-5.2},
      {0 16/half16/}/2/{0,-6.5},
      {0 16/pc16/}/2/{44,-1.3},
      {0 4/ /,16/low12/}/2/{44,-2.6},
      {0 4/ /,16/pc12/}/2/{44,-3.9},
      {0 8/\vbox{\hbox{byte8}\vskip 1em }/}/1/{44,-5.2}}
    {
      \begin{scope}[shift={(\shifted)}]
        \pgfmathsetmacro{\nbits}{8*\nbytes}
        \path [bitchart box] (0, 0) rectangle (\nbits, 1);
        \begin{scope}[every path/.style={draw,shorten >=2em}]
          \pgfmathsetmacro{\bytesm}{\nbytes-1}
          \bitchartbytes{0}{0,...,\bytesm}
        \end{scope}
        \begin{scope}[bitfield label/.style={shift={(0,-0.2)}}]
          \csname bitchartfields\expandafter\endcsname\fields;
        \end{scope}
      \end{scope}
    }
    \else
    \foreach \fields/\nbytes/\shifted in {
      {0 32/word32/}/4/{0,0},
      {0 16/half16/}/2/{40,0},
      {0 16/pc16/}/2/{0,-1.3},
      {0 4/ /,16/low12/}/2/{24,-1.3},
      {0 8/\vbox{\hbox{byte8}\vskip 1em }/}/1/{48,-1.3}}
    {
      \begin{scope}[shift={(\shifted)}]
        \pgfmathsetmacro{\nbits}{8*\nbytes}
        \path [bitchart box] (0, 0) rectangle (\nbits, 1);
        \begin{scope}[every path/.style={draw,shorten >=2em}]
          \pgfmathsetmacro{\bytesm}{\nbytes-1}
          \bitchartbytes{0}{0,...,\bytesm}
        \end{scope}
        \begin{scope}[bitfield label/.style={shift={(0,-0.2)}}]
          \csname bitchartfields\expandafter\endcsname\fields;
        \end{scope}
      \end{scope}
    }
    \fi
  \end{tikzpicture}
  \fi
  \caption[Relocation fields]{Relocation fields.  Bit numbers appear in
    the lower box corners; byte numbers appear in the upper left box
    corners.}
  \label{fig:relocfields}
\end{figure}

\begin{description}
\ifzseries
\item[\texttt{quad64}] This specifies a 64-bit field occupying 8
  bytes, the alignment of which is 4 bytes unless otherwise specified.
\fi
\item[\texttt{word32}] This specifies a 32-bit field occupying 4 bytes, the
  alignment of which is 4 bytes unless otherwise specified.
\ifzseries
\item[\texttt{pc32}] This specifies a 32-bit field occupying 4 bytes
  with 2-byte alignment.  The signed value in this field is shifted to
  the left by 1 before it is used as a program counter relative
  displacement (for example, the immediate field of a ``Load Address
  Relative Long'' instruction).
\item[\texttt{pc24}] This specifies a 24-bit field contained within 4
  consecutive bytes with 2-byte alignment.  The signed value in this field
  is shifted to the left by 1 before it is used as a program counter
  relative displacement (for example, the third immediate field of a
  ``Branch Prediction Relative Preload'' instruction).
\item[\texttt{mid20}] This specifies a 20-bit field contained within 4
  consecutive bytes with 2-byte alignment.  The 20-bit signed value is the
  ``long displacement'' of a memory reference.
\fi
\item[\texttt{half16}] This specifies a 16-bit field occupying 2 bytes with
  2-byte alignment (for example, the immediate field of an ``Add
  Halfword Immediate'' instruction).
\item[\texttt{pc16}] This specifies a 16-bit field occupying 2 bytes
  with 2-byte alignment.  The signed value in this field is shifted to
  the left by 1 before it is used as a program counter relative
  displacement (for example, the immediate field of an ``Branch
  Relative'' instruction).
\item[\texttt{low12}] This specifies a 12-bit field contained within a
  halfword with 2-byte alignment.  The 12 bit unsigned value is the
  displacement of a memory reference.
\item[\texttt{pc12}] This specifies a 12-bit field contained within a
  halfword with 2-byte alignment.  The signed value in this field is
  shifted to the left by 1 before it is used as a program counter relative
  displacement (for example, the second immediate field of a ``Branch
  Prediction Relative Preload'' instruction).
\item[\texttt{byte8}] This specifies an 8-bit field with 1-byte
  alignment.
\end{description}

Calculations in \cref{tab:relocations} assume the actions are
transforming a relocatable file into either an executable or a shared
object file.  Conceptually, the linkage editor merges one or more
relocatable files to form the output.  It first determines how to
combine and locate the input files, next it updates the symbol values,
and then it performs relocations.

Relocations applied to executable or shared object files are similar
and accomplish the same result.  The following notations are used in
\cref{tab:relocations}:

\begin{description}
\item[$A$] Represents the addend used to compute the value of the
  relocatable field.
\item[$B$] Represents the base address at which a shared object has
  been loaded into memory during execution.  Generally, a shared object
  file is built with a 0 base virtual address, but the execution
  address will be different.
\item[$G$] Represents the section offset or address of the Global
  Offset Table.  See \cref{codingexamples,globaloffsettable}
  for more information.
\item[$L$] Represents the section offset or address of the Procedure
  Linkage Table entry for a symbol.  A PLT entry redirects a function
  call to the proper destination.  The linkage editor builds the
  initial PLT\@.  See \cref{procedurelinkagetable} for more
  information.
\item[$O$] Represents the offset into the GOT at which the address of
  the relocation entry's symbol will reside during execution.  See
  \cref{codingexamples,globaloffsettable} for more
  information.
\item[$P$] Represents the place (section offset or address) of the
  storage unit being relocated (computed using \texttt{r\_offset}).
\item[$R$] Represents the offset of the symbol within the section in
  which the symbol is defined (its section-relative address).
\item[$S$] Represents the value of the symbol whose index resides in
  the relocation entry.
\item[$T$] Similar to $O$, except that the address that is stored may be
  the address of the PLT entry for the symbol.
\end{description}

Relocation entries apply to bytes, halfwords, {\ifzseries words, or
  doublewords\else or words\fi}.  In either case, the \texttt{r\_offset}
value designates the offset or virtual address of the first byte of the
affected storage unit.  The relocation type specifies which bits to change
and how to calculate their values.  The \ARCH{} family uses only the
\texttt{Elf\NBITS{}\_Rela} relocation entries with explicit addends.  For
the relocation entries, the \texttt{r\_addend} field serves as the
relocation addend.  In all cases, the offset, addend, and the computed
result use the byte order specified in the ELF header.

The following general rules apply to the interpretation of the relocation
types in \cref{tab:relocations}:

\begin{itemize}
\item ``$+$'' and ``$-$'' denote \NBITS{}-bit modulus addition and
  subtraction, respectively.  ``$>>$'' denotes arithmetic right-shifting
  (shifting with sign copying) of the value of the left operand by the
  number of bits given by the right operand.
\item Reference in a calculation to the value $G$, $O$, or $T$ implicitly
  creates a GOT entry for the indicated symbol, and a reference to $L$
  implicitly creates a PLT entry.
\item A computed value must be suited for the relocation field it is used
  for.  In particular:
  \begin{description}
  \item[\texttt{half16}:] The upper {\ifzseries 48\else 16\fi} bits must
    be all ones or all zeroes.
  \item[\texttt{pc16}:] The upper {\ifzseries 47\else 15\fi} bits must be
    all ones or all zeroes and the lowest bit must be zero.
    %
    {\ifzseries{} \item[\texttt{pc32}:] The upper 31 bits must be all ones
      or all zeroes and the lowest bit must be zero.\fi}
    %
  \item[\texttt{low12}:] The upper {\ifzseries 52\else 20\fi} bits must
    all be zero.
  \item[\texttt{byte8}:] The upper {\ifzseries 56\else 24\fi} bits must
    all be zero.
  \end{description}
\end{itemize}

\begin{DIFnomarkup}
\begin{longtable}{lrll}
  \caption{Relocation types\label{tab:relocations}}\\[\medskipamount]
  \toprule
  Name & Value & Field & Calculation \\
  \midrule
  \endfirsthead
  \caption[]{Relocation types \emph{-- continued}}\\[\medskipamount]
  \toprule
  Name & Value & Field & Calculation \\
  \midrule
  \endhead
  \bottomrule
  \endfoot
  \texttt{R\_390\_NONE} & 0 & \emph{none} & \emph{none} \\
  \texttt{R\_390\_8} & 1 & \texttt{byte8} & $S + A$ \\
  \texttt{R\_390\_12} & 2 & \texttt{low12} & $S + A$ \\
  \texttt{R\_390\_16} & 3 & \texttt{half16} & $S + A$ \\
  \texttt{R\_390\_32} & 4 & \texttt{word32} & $S + A$ \\
  \texttt{R\_390\_PC32} & 5 & \texttt{word32} & $S + A - P$ \\
  \texttt{R\_390\_GOT12} & 6 & \texttt{low12} & $O + A$ \\
  \texttt{R\_390\_GOT32} & 7 & \texttt{word32} & $O + A$ \\
  \texttt{R\_390\_PLT32} & 8 & \texttt{word32} & $L + A$ \\
  \texttt{R\_390\_COPY}\textsuperscript{ \dagger} & 9 & \emph{none} & \\
  \texttt{R\_390\_GLOB\_DAT}\textsuperscript{ \dagger} & 10 & \texttt{quad64} & $S + A$ \\
  \texttt{R\_390\_JMP\_SLOT}\textsuperscript{ \dagger} & 11 & \emph{none} & \\
  \texttt{R\_390\_RELATIVE}\textsuperscript{ \dagger} & 12 & \texttt{quad64} & $B + A$ \\
  \texttt{R\_390\_GOTOFF32} & 13 & \texttt{word32} & $S + A - G$ \\
  \texttt{R\_390\_GOTPC} & 14 & \texttt{quad64} & $G + A - P$ \\
  \texttt{R\_390\_GOT16} & 15 & \texttt{half16} & $O + A$ \\
  \texttt{R\_390\_PC16} & 16 & \texttt{half16} & $S + A - P$ \\
  \texttt{R\_390\_PC16DBL} & 17 & \texttt{pc16} & $(S + A - P) >> 1$ \\
  \texttt{R\_390\_PLT16DBL} & 18 & \texttt{pc16} & $(L + A - P) >> 1$ \\
  \ifzseries
  \texttt{R\_390\_PC32DBL} & 19 & \texttt{pc32} & $(S + A - P) >> 1$ \\
  \texttt{R\_390\_PLT32DBL} & 20 & \texttt{pc32} & $(L + A - P) >> 1$ \\
  \texttt{R\_390\_GOTPCDBL} & 21 & \texttt{pc32} & $(G + A - P) >> 1$ \\
  \texttt{R\_390\_64} & 22 & \texttt{quad64} & $S + A$ \\
  \texttt{R\_390\_PC64} & 23 & \texttt{quad64} & $S + A - P$ \\
  \texttt{R\_390\_GOT64} & 24 & \texttt{quad64} & $O + A$ \\
  \texttt{R\_390\_PLT64} & 25 & \texttt{quad64} & $L + A - P$ \\
  \texttt{R\_390\_GOTENT} & 26 & \texttt{pc32} & $(G + O + A - P) >> 1$ \\
  \texttt{R\_390\_GOTOFF16} & 27 & \texttt{half16} & $S + A - G$ \\
  \texttt{R\_390\_GOTOFF64} & 28  & \texttt{quad64} & $S + A - G$\\
  \texttt{R\_390\_GOTPLT12} & 29 & \texttt{low12} & $T + A$ \\
  \texttt{R\_390\_GOTPLT16} & 30 & \texttt{half16} & $T + A$ \\
  \texttt{R\_390\_GOTPLT32} & 31 & \texttt{word32} & $T + A - P$ \\
  \texttt{R\_390\_GOTPLT64} & 32 & \texttt{quad64} & $T + A$ \\
  \texttt{R\_390\_GOTPLTENT} & 33 & \texttt{pc32} & $(G + T + A - P) >> 1$ \\
  \texttt{R\_390\_PLTOFF16} & 34 & \texttt{half16} & $L - G + A$ \\
  \texttt{R\_390\_PLTOFF32} & 35 & \texttt{word32} & $L - G + A$ \\
  \texttt{R\_390\_PLTOFF64} & 36 & \texttt{quad64} & $L - G + A$ \\
  \texttt{R\_390\_TLS\_LOAD}\textsuperscript{ \dagger} & 37 & \emph{none} & \\
  \texttt{R\_390\_TLS\_GDCALL}\textsuperscript{ \dagger} & 38 & \emph{none} & \\
  \texttt{R\_390\_TLS\_LDCALL}\textsuperscript{ \dagger} & 39 & \emph{none} & \\
  \texttt{R\_390\_TLS\_GD64}\textsuperscript{ \dagger} & 41 & \texttt{quad64} & \\
  \texttt{R\_390\_TLS\_GOTIE12}\textsuperscript{ \dagger} & 42 & \texttt{low12} & \\
  \texttt{R\_390\_TLS\_GOTIE64}\textsuperscript{ \dagger} & 44 & \texttt{quad64} & \\
  \texttt{R\_390\_TLS\_LDM64}\textsuperscript{ \dagger} & 46 & \texttt{quad64} & \\
  \texttt{R\_390\_TLS\_IE64}\textsuperscript{ \dagger} & 48 & \texttt{quad64} & \\
  \texttt{R\_390\_TLS\_IEENT}\textsuperscript{ \dagger} & 49 & \texttt{pc32} & \\
  \texttt{R\_390\_TLS\_LE64}\textsuperscript{ \dagger} & 51 & \texttt{quad64} & \\
  \texttt{R\_390\_TLS\_LDO64}\textsuperscript{ \dagger} & 53 & \texttt{quad64} & \\
  \texttt{R\_390\_TLS\_DTPMOD}\textsuperscript{ \dagger} & 54 & \texttt{quad64} & \\
  \texttt{R\_390\_TLS\_DTPOFF}\textsuperscript{ \dagger} & 55 & \texttt{quad64} & \\
  \texttt{R\_390\_TLS\_TPOFF}\textsuperscript{ \dagger} & 56 & \texttt{quad64} & \\
  \texttt{R\_390\_20} & 57 & \texttt{mid20} & $S + A$ \\
  \texttt{R\_390\_GOT20} & 58 & \texttt{mid20} & $O + A$ \\
  \texttt{R\_390\_GOTPLT20} & 59 & \texttt{mid20} & $T + A$ \\
  \texttt{R\_390\_TLS\_GOTIE20}\textsuperscript{ \dagger} & 60 & \texttt{mid20} & \\
  \texttt{R\_390\_IRELATIVE}\textsuperscript{ \dagger} & 61 & \texttt{quad64} & $\char`*(B + A)()$ \\
  \texttt{R\_390\_PC12DBL} & 62 & \texttt{pc12} & $(S + A - P) >> 1$ \\
  \texttt{R\_390\_PLT12DBL} & 63 & \texttt{pc12} & $(L + A - P) >> 1$ \\
  \texttt{R\_390\_PC24DBL} & 64 & \texttt{pc24} & $(S + A - P) >> 1$ \\
  \texttt{R\_390\_PLT24DBL} & 65 & \texttt{pc24} & $(L + A - P) >> 1$ \\
  \fi
\end{longtable}
\end{DIFnomarkup}

The relocation types marked with ``\dagger'' in \cref{tab:relocations} are
handled specially:

\begin{description}
\item[\texttt{R\_390\_COPY}] The linkage editor creates this relocation
  type for dynamic linking.  Its offset member refers to a location in a
  writable segment.  The symbol table index specifies a symbol that should
  exist both in the current object file and in a shared object.  During
  execution, the dynamic linker copies data associated with the shared
  object's symbol to the location specified by the offset.
\item[\texttt{R\_390\_GLOB\_DAT}] This relocation type resembles
  \texttt{R\_390\_\NBITS{}}, except that it sets a Global Offset Table
  entry to the address of the specified symbol.  This special relocation
  type allows one to determine the correspondence between symbols and GOT
  entries.
\item[\texttt{R\_390\_JMP\_SLOT}] The linkage editor creates this
  relocation type for dynamic linking.  Its offset member gives the
  location of a Global Offset Table entry.  The dynamic linker modifies
  the GOT entry to transfer control to the designated symbol's address
  (see \cref{procedurelinkagetable}).
\item[\texttt{R\_390\_RELATIVE}] The linkage editor creates this
  relocation type for dynamic linking.  Its offset member gives a location
  within a shared object that contains a value representing a virtual
  address.  The dynamic linker computes the virtual address by adding the
  shared object's base address to the addend.  Relocation entries for this
  type must specify 0 for the symbol table index.
\item[\texttt{R\_390\_IRELATIVE}] The linkage editor creates this
  relocation type for dynamic linking.  The dynamic linker computes an
  address as for the \texttt{R\_390\_RELATIVE} relocation and then invokes
  the function residing at that address, passing the value of
  \texttt{AT\_HWCAP} from the auxiliary vector as its single argument (see
  \cref{auxvector}).  The return value resulting from that invocation is
  written into the location described by the offset.  Such a function is
  also known as an ``IFUNC resolver'' and has the following signature:
  \begin{center}
    \lstinline@void *f (unsigned long hwcap);@
  \end{center}
\item[\texttt{R\_390\_TLS\_\char`*}] These relocation types are used for
  thread-local storage handling.  They are described in
  \cite{tlshandling}.
\end{description}

\chapter{Program Loading and Dynamic Linking}
\label{chprogload}
This section describes how the Executable and Linking Format (ELF) is
used in the construction and execution of programs.

\section{Program Loading}
\index{program loading}
As the system creates or augments a process image, it logically copies
a file's segment to a virtual memory segment.  When---and if---the
system physically reads the file depends on the program's execution
behavior, on the system load, and so on.  A process does not require a
physical page until it references the logical page during execution,
and processes commonly leave many pages unreferenced.  Therefore, if
physical reads can be delayed they can frequently be dispensed with,
improving system performance.  To obtain this efficiency in practice,
executable and shared object files must have segment images of which
the offsets and virtual addresses are congruent modulo the page size.

Virtual addresses and file offsets for the \ARCH{} processor family
segments are congruent modulo {\ifzseries the system page size\else
  4$\,$Kbytes\fi}.  The value of the \texttt{p\_align} field of each
program header in a shared object file must be {\ifzseries a multiple
  of the system page size\else \texttt{0x1000} (4$\,$Kbytes)\fi}.
\Cref{fig:execfile} is an example of an executable file assuming an
executable program linked with a base address of {\ifzseries
  \texttt{0x80000000} (2$\,$Gbytes)\else \texttt{0x00400000}
  (4$\,$Mbytes)\fi}.

\begin{figure}
  \centering
  \ifSkipTikZ
\begin{verbatim}
File Offset                               Virtual Address
         0 +----------------------------+ 0x80000000
           |         ELF header         |
           |    Program header table    |
           |     Other information      |
           |                            |
           |         Text segment       |
           |           . . .            |
           |        0x1bf58 bytes       |
           |                            | 0x8001bfff
           +----------------------------+
   0x1bf58 |                            | 0x8001cf58
           |        Data segment        |
           |           . . .            |
           |        0x17c4 bytes        |
           |                            | 0x8001d72b
           +----------------------------+
   0x1d71c |     Other Information      |
           +----------------------------+
\end{verbatim}
  \else
  \begin{tikzpicture}
    \matrix [memory layout,inner sep=0pt,
    nodes={align=center, inner sep=0.7ex}] (m) {
      \node (A1) {ELF header}; \\
      \node (A2) {Program header table}; \\
      \node (A3) {Other information}; \\
      \node (A4) {\stackit[c]{\noalign{\medskip}
          Text segment\\ \ldots\\\noalign{\medskip}}}; \\
      \node (A5) {\texttt{0x1bf58} bytes}; \\
      \node (B1) {\stackit[c]{Data segment\\ \ldots\\
          \noalign{\medskip}}}; \\
      \node (B2) {\texttt{0x17c4} bytes}; \\
      \node (C1) {Other information\strut}; \\
    };
    \foreach \Node in {A5, B2} {
      \draw (\Node.south -| m.west) -- (\Node.south -| m.east);
    }
    \path (m.north west) + (-1ex,0.5ex) node [above left]
    (foffs) {File Offset};
    \path (m.north east) + (1ex,0.5ex) node [above right]
    (vaddr) {Virtual Address};
    \foreach \Node/\text in {A1/0, B1/0x1bf58, C1/0x1d71c} {
      \path (\Node.north west -| foffs.east) node [below left]
      {\texttt{\text}};
    }
    \ifzseries
    \foreach \Node/\text in {A1/0x80000000, B1/0x8001cf58} {
      \path (\Node.north east -| vaddr.west) node [below right]
      {\texttt{\text}};
    }
    \foreach \Node/\text in {A5/0x8001bfff, B2/0x8001e71b} {
      \path (\Node.south east -| vaddr.west) node [above right]
      {\texttt{\text}};
    }
    \else
    \foreach \Node/\text in {A1/0x00400000, B1/0x0041cf58} {
      \path (\Node.north east -| vaddr.west) node [below right]
      {\texttt{\text}};
    }
    \foreach \Node/\text in {A5/0x0041bfff, B2/0x0041e71b} {
      \path (\Node.south east -| vaddr.west) node [above right]
      {\texttt{\text}};
    }
    \fi
  \end{tikzpicture}
  \fi
  \caption{Executable file example}
  \label{fig:execfile}
\end{figure}

\begin{table}
  \centering
  \begin{DIFnomarkup}
  \begin{tabular}{lll}
    \toprule
    Member & Text & Data \\
    \midrule
    \texttt{p\_type} & \texttt{PT\_LOAD} & \texttt{PT\_LOAD} \\
    \texttt{p\_offset} & \texttt{0x0} & \texttt{0x1bf58} \\
    \texttt{p\_vaddr} & \texttt{\ifzseries 0x80000000\else 0x400000\fi}
    & \texttt{\ifzseries 0x8001cf58\else 0x41cf58\fi} \\
    \texttt{p\_paddr} & unspecified & unspecified \\
    \texttt{p\_filesz} & \texttt{0x1bf58} & \texttt{0x17c4} \\
    \texttt{p\_memsz} & \texttt{0x1bf58} & \texttt{0x2578} \\
    \texttt{p\_flags} & \texttt{PF\_R+PF\_X} & \texttt{PF\_R+PF\_W} \\
    \texttt{p\_align} & \texttt{0x1000} & \texttt{0x1000} \\
    \bottomrule
  \end{tabular}
  \end{DIFnomarkup}
  \caption{Program header segments}
  \label{tab:phdr}
\end{table}

Although the file offsets and virtual addresses are congruent modulo
4$\,$Kbytes for both text and data, up to four file pages can hold
impure text or data (depending on page size and file system block
size).

\begin{itemize}
\item The first text page contains the ELF header, the program header
  table, and other information.
\item The last text page may hold a copy of the beginning of data.
\item The first data page may have a copy of the end of text.
\item The last data page may contain file information not relevant to
  the running process.
\end{itemize}

Logically, the system enforces memory permissions as if each segment
were complete and separate; segment addresses are adjusted to ensure
that each logical page in the address space has a single set of
permissions.  In the example in \cref{tab:phdr} the file region
holding the end of text and the beginning of data is mapped twice; at
one virtual address for text and at a different virtual address for
data.

The end of the data segment requires special handling for
uninitialized data, which the system defines to begin with zero
values.  Thus if the last data page of a file includes information
beyond the logical memory page, the extraneous data must be set to
zero by the loader, rather than to the unknown contents of the
executable file.  ``Impurities'' in the other three segments are not
logically part of the process image, and whether the system clears
them is unspecified.  The memory image for the program in
\cref{tab:phdr} is presented in \cref{fig:pimgseg}.

\begin{figure}
  \centering
  \ifSkipTikZ
\begin{verbatim}
Virtual Address                             Segment
  0x80000000 +----------------------------+
             |         ELF header         |
             |    Program header table    |
             |     Other information      |
             |                            |  Text
             |         Text segment       |
             |           . . .            |
             |        0x1bf58 bytes       |
             +----------------------------+
  0x8001bf58 |        Page padding        |
             |         0xa8 bytes         |
             +----------------------------+

             +----------------------------+
  0x8001c000 |          Padding           |
             |         0xf58 bytes        |
             +----------------------------+
  0x8001cf58 |                            |
             |        Data segment        |
             |           . . .            |  Data
             |        0x17c4 bytes        |
             |                            |
             +----------------------------+
  0x8001e71c |     Uninitialized data     |
             |        0xdb4 bytes         |
             +----------------------------+
  0x8001f4d0 |        Page padding        |
             |        0xb30 bytes         |
  0x8001ffff +----------------------------+
\end{verbatim}
  \else
  \pgfsetlayers{background,main}
  \begin{tikzpicture}
    \matrix [inner sep=0pt,
    nodes={align=center, inner sep=0.7ex}] (m) {
      \node (A1) {ELF header}; \\
      \node (A2) {Program header table}; \\
      \node (A3) {Other information}; \\
      \node (A4) {\stackit[c]{\noalign{\medskip} Text segment\\ \ldots}}; \\
      \node (A5) {\texttt{0x1bf58} bytes}; \\
      \node (B1) {Page padding}; \\
      \node (B2) {\texttt{0xa8} bytes}; \\
      \node (C1) {\strut}; \\
      \node (D1) {Padding}; \\
      \node (D2) {\texttt{0xf58} bytes}; \\
      \node (E1) {\stackit[c]{Data segment\\ \ldots}}; \\
      \node (E2) {\texttt{0x17c4} bytes}; \\
      \node (F1) {Uninitialized data}; \\
      \node (F2) {\texttt{0xdb4} bytes}; \\
      \node (G1) {Page padding}; \\
      \node (G2) {\texttt{0xb30} bytes}; \\
    };
    \begin{pgfonlayer}{background}
      \path [memory layout] (B2.south west -| m.west)
      rectangle (A1.north east -| m.east);
      \path [memory layout] (G2.south west -| m.west)
      rectangle (D1.north east -| m.east);
    \end{pgfonlayer}
    \foreach \Node in {A5, D2, E2, F2} {
      \draw (\Node.south -| m.west) -- (\Node.south -| m.east);
    }
    \path (m.north west) + (-1ex,0.5ex) node [above left]
    (vaddr) {Virtual Address};
    \path (m.north east) + (1ex,0.5ex) node [above right]
    (segm) {Segment};
    \ifzseries
    \foreach \Node/\text in {%
      A1/0x80000000,
      B1/0x8001bf58,
      D1/0x8001c000,
      E1/0x8001cf58,
      F1/0x8001e71c,
      G1/0x8001f4d0} {
      \path (\Node.north -| vaddr.east) node [below left] {\texttt{\text}};
    }
    \path (m.south -| vaddr.east) node [above left] {\texttt{0x8001ffff}};
    \else
    \foreach \Node/\text in {%
      A1/0x00400000,
      B1/0x0041bf58,
      D1/0x0041c000,
      E1/0x0041cf58,
      F1/0x0041e71c,
      G1/0x0041f4d0} {
      \path (\Node.north -| vaddr.east) node [below left] {\texttt{\text}};
    }
    \path (m.south -| vaddr.east) node [above left] {\texttt{0x0041ffff}};
    \fi
    \foreach \from/\to/\text in {%
      A1/B2/Text, D1/G2/Data
    } {
      \path (\from.north -| segm.west) -- node [pos=0.5,right] {\text}
      (\to.south -| segm.west);
    }
  \end{tikzpicture}
  \fi
\caption{Process image segments}
\label{fig:pimgseg}
\end{figure}

One aspect of segment loading differs between executable files and
shared objects.  Executable file segments may contain absolute code.
For the process to execute correctly, the segments must reside at the
virtual addresses assigned when building the executable file, with the
system using the \texttt{p\_vaddr} values unchanged as virtual
addresses.

On the other hand, shared object segments typically contain
position-independent code.  This allows a segment's virtual address to
change from one process to another, without invalidating execution
behavior.  Though the system chooses virtual addresses for individual
processes, it maintains the ``relative positions'' of the
segments.  Because position-independent code uses relative addressing
between segments, the difference between virtual addresses in memory
must match the difference between virtual addresses in the
file.  \Cref{tab:soseg} shows possible shared object virtual
address assignments for several processes, illustrating constant
relative positioning.  The table also illustrates the base address
computations.

\begin{table}
  \centering
  \begin{DIFnomarkup}
  \begin{tabular}{llll}
    \toprule
    Source & Text & Data & Base Address \\
    \midrule
    \ifzseries
    File & \texttt{0x00000000000} & \texttt{0x0000002a400} & \\
    Process 1 & \texttt{0x20000000000} & \texttt{0x2000002a400} &
    \texttt{0x20000000000} \\
    Process 2 & \texttt{0x20000010000} & \texttt{0x2000003a400} &
    \texttt{0x20000010000} \\
    Process 3 & \texttt{0x20000020000} & \texttt{0x2000004a400} &
    \texttt{0x20000020000} \\
    Process 4 & \texttt{0x20000030000} & \texttt{0x2000005a400} &
    \texttt{0x20000030000} \\
    \else
    File & \texttt{0x00000200} & \texttt{0x0002a400} & \\
    Process 1 & \texttt{0x40000000} & \texttt{0x4002a400} &
    \texttt{0x40000000} \\
    Process 2 & \texttt{0x40010000} & \texttt{0x4003a400} &
    \texttt{0x40010000} \\
    Process 3 & \texttt{0x40020000} & \texttt{0x4004a400} &
    \texttt{0x40020000} \\
    Process 4 & \texttt{0x40030000} & \texttt{0x4005a400} &
    \texttt{0x40030000} \\
    \fi
    \bottomrule
  \end{tabular}
  \end{DIFnomarkup}
  \caption{Shared object segment example\ifzseries{} for 42-bit
    address space\fi}
  \label{tab:soseg}
\end{table}

\section{Dynamic Linking}
\label{dynamiclinking}
\index{dynamic linking}
\subsection{Dynamic Section}
Dynamic section entries give information to the dynamic linker.  Some
of this information is processor-specific, including the
interpretation of some entries in the dynamic structure.

\begin{description}
\item[\texttt{DT\_PLTGOT}] The \texttt{d\_ptr} field of this entry gives
  the address of the first byte in the Global Offset Table.  See
  \cref{globaloffsettable} for more information.

\item[\texttt{DT\_JMPREL}] This entry is associated with a table of
  relocation entries for the PLT\@.  For \ABINAME{} this entry is
  mandatory both for executable and shared object files.  Moreover,
  the relocation table's entries must have a one-to-one correspondence
  with the PLT\@.  The table of \texttt{DT\_JMPREL} relocation entries
  is wholly contained within the \texttt{DT\_RELA} referenced table.
  See \cref{procedurelinkagetable} for more information.
\end{description}

\subsection{Global Offset Table}
\label{globaloffsettable}
\index{global offset table}
\index{GOT}
Position-independent code cannot, in general, contain absolute virtual
addresses.  Global Offset Tables hold absolute addresses in private
data, thus making the addresses available without compromising the
position-independence and sharability of a program's text.  A program
references its GOT using position-independent addressing and extracts
absolute values, thus redirecting position-independent references to
absolute locations.

When the dynamic linker creates memory segments for a loadable object
file, it processes the relocation entries, some of which will be of
type \texttt{R\_390\_GLOB\_DAT}, referring to the GOT\@.  The dynamic
linker determines the associated symbol values, calculates their
absolute addresses, and sets the GOT entries to the proper
values.  Although the absolute addresses are unknown when the linkage
editor builds an object file, the dynamic linker knows the addresses
of all memory segments and can thus calculate the absolute addresses
of the symbols contained therein.

A GOT entry provides direct access to the absolute address of a symbol
without compromising position-independence and sharability.  Because
the executable file and shared objects have separate GOTs, a symbol
may appear in several tables.  The dynamic linker processes all the
GOT relocations before giving control to any code in the process
image, thus ensuring the absolute addresses are available during
execution.

The dynamic linker may choose different memory segment addresses for
the same shared object in different programs; it may even choose
different library addresses for different executions of the same
program.  Nevertheless, memory segments do not change addresses once
the process image is established.  As long as a process exists, its
memory segments reside at fixed virtual addresses.

The format and interpretation of the Global Offset Table is processor
specific.  For \ABINAME{} the symbol \texttt{\_GLOBAL\_OFFSET\_TABLE\_}
may be used to access the table.  The symbol refers to the start of
the \texttt{.got} section.  Two words in the GOT are reserved:
\begin{itemize}
\item The word at \texttt{\_GLOBAL\_OFFSET\_TABLE\_[0]} is set by the
  linkage editor to hold the address of the dynamic structure,
  referenced with the symbol \texttt{\_DYNAMIC}.  This allows a
  program, such as the dynamic linker, to find its own dynamic
  structure without having yet processed its relocation entries.  This
  is especially important for the dynamic linker, because it must
  initialize itself without relying on other programs to relocate its
  memory image.
\item The word at \texttt{\_GLOBAL\_OFFSET\_TABLE\_[1]} is reserved
  for future use.
\end{itemize}

The Global Offset Table resides in the ELF \texttt{.got} section.

\subsection{Function Addresses}
\label{functionaddresses}
References to a function address from an executable file and from the
shared objects associated with the file must resolve to the same
value.  References from within shared objects will normally be resolved
(by the dynamic linker) to the virtual address of the function itself.
References from within the executable file to a function defined in a
shared object will normally be resolved (by the linkage editor) to the
address of the Procedure Linkage Table entry for that function within
the executable file.

To allow comparisons of function addresses to work as expected, if an
executable file references a function defined in a shared object, the
linkage editor will place the address of the PLT entry for that
function in its associated symbol table entry.  See
\cref{symbolvalues} for details.  The dynamic linker treats
such symbol table entries specially.  If the dynamic linker is
searching for a symbol and encounters a symbol table entry for that
symbol in the executable file, it normally follows these rules:

\begin{itemize}
\item If the \texttt{st\_shndx} field of the symbol table entry is not
  \texttt{SHN\_UNDEF}, the dynamic linker has found a definition for
  the symbol and uses its \texttt{st\_value} field as the symbol's
  address.
\item If the \texttt{st\_shndx} field is \texttt{SHN\_UNDEF} and the
  symbol is of type \texttt{STT\_FUNC} and the \texttt{st\_value}
  field is not zero, the dynamic linker recognizes this entry as
  special and uses the \texttt{st\_value} field as the symbol's
  address.
\item Otherwise, the dynamic linker considers the symbol to be
  undefined within the executable file and continues processing.
\end{itemize}

Some relocations are associated with PLT entries.  These entries are
used for direct function calls rather than for references to function
addresses.  These relocations are not treated specially as described
above because the dynamic linker must not redirect PLT entries to
point to themselves.

\subsection{Procedure Linkage Table}
\label{procedurelinkagetable}
\index{procedure linkage table}
\index{PLT}
Much as the Global Offset Table redirects position-independent address
calculations to absolute locations, the Procedure Linkage Table
redirects position-independent function calls to absolute
locations.  The linkage editor cannot resolve execution transfers (such
as function calls) from one executable or shared object to another, so
instead it arranges for the program to transfer control to entries in
the PLT\@.  The dynamic linker determines the absolute addresses of the
destinations and stores them in the GOT, from which they are loaded by
the PLT entry.  The dynamic linker can thus redirect the entries
without compromising the position-independence and sharability of the
program text.  Executable files and shared object files have separate
PLTs.

As mentioned above, a relocation table is associated with the PLT\@.
The \texttt{DT\_JMPREL} entry in the \texttt{\_DYNAMIC} array gives
the location of the first relocation entry.  The relocation table
entries match the PLT entries in a one-to-one correspondence
(relocation table entry 1 applies to PLT entry 1 and so on).  The
relocation type for each entry shall be
\texttt{R\_390\_JMP\_SLOT}.  The relocation offset shall specify the
address of the GOT entry containing the address of the function, and
the symbol table index shall reference the appropriate symbol.

To illustrate Procedure Linkage Tables, \cref{lst:pltex} shows how
the linkage editor might initialize the PLT when linking a shared
executable or shared object.

% FIXME below: the layout of the listing below is difficult to understand:
% Two different PLT entries in one listing; untypical assembler source
% style, registers specified as mere numbers, use of general mnemonics
% instead of extended ones, etc.
%
% Also:
%
% * "L" -> "LG".  This seems like a cut & paste error from the ESA/390
%   ABI.
%
% * Listing should better be split before second asterisk.
%
% * Last comment: "Offset into symbol table" -> "offset into .rela.plt".

\ifzseries
\begin{lstlisting}[style=float,language=simpleasm,
  caption=Procedure Linkage Table example,label=lst:pltex]
*                                  # PLT for executables (not
                                   #   position-independent)
PLT1      BASR  1,0                # Establish base
BASE1     L     1,AGOTENT-BASE1(1) # Load address of the GOT entry
          L     1,0(0,1)           # Load function address from the
                                   #   GOT to r1
          BCR   15,1               # Jump to address
RET1      BASR  1,0                # Return from GOT first time
                                   #   (lazy binding)
BASE2     L     1,ASYMOFF-BASE2(1) # Load offset in symbol table to r1
          BRC   15,-x              # Jump to start of PLT
          .word 0                  # Filler
AGOTENT   .long ?                  # Address of the GOT entry
ASYMOFF   .long ?                  # Offset into the symbol table
*                                  # PLT for shared objects
                                   #   (position-independent)
PLT1      LARL  1,<fn>@GOTENT      # Load address of GOT entry in r1
          LG    1,0(1)             # Load function address from the
                                   #   GOT to r1
          BCR   15,1               # Jump to address
RET1      BASR  1,0                # Return from GOT first time
                                   #   (lazy binding)
BASE2     LGF   1,ASYMOFF-BASE2(1) # Load offset in symbol table to r1
          BRCL  15,-x              # Jump to start of PLT
ASYMOFF   .long ?                  # Offset into symbol table
\end{lstlisting}
\else
\begin{lstlisting}[style=float,language=simpleasm,
  caption=Procedure Linkage Table example,label=lst:pltex]
*                                  # PLT for executables (not
                                   #   position-independent)
PLT1      BASR  1,0                # Establish base
BASE1     L     1,AGOTENT-BASE1(1) # Load address of the GOT entry
          L     1,0(0,1)           # Load function address from the
                                   #   GOT to r1
          BCR   15,1               # Jump to address
RET1      BASR  1,0                # Return from GOT first time
                                   #   (lazy binding)
BASE2     L     1,ASYMOFF-BASE2(1) # Load offset in symbol table to r1
          BRC   15,-x              # Jump to start of PLT
          .word 0                  # Filler
AGOTENT   .long ?                  # Address of the GOT entry
ASYMOFF   .long ?                  # Offset into the symbol table
*                                  # PLT for shared objects
                                   #   (position-independent)
PLT1      BASR  1,0                # Establish base
BASE1     L     1,AGOTOFF-BASE1(1) # Load offset into the GOT to r1
          L     1,(1,12)           # Load address from the GOT to r1
          BCR   15,1               # Jump to address
RET1      BASR  1,0                # Return from GOT first time
                                   #   (lazy binding)
BASE2     L     1,ASYMOFF-BASE2(1) # Load offset in symbol table to r1
          BRC   15,-x              # Jump to start of PLT
          .word 0                  # Filler
AGOTOFF   .long ?                  # Offset in the GOT
ASYMOFF   .long ?                  # Offset in the symbol table
\end{lstlisting}
\fi

As described below, the dynamic linker and the program cooperate to
resolve symbolic references through the PLT\@.  Again, the details
described below are for explanation only.  The precise execution-time
behavior of the dynamic linker is not specified.

\begin{enumerate}
\item The caller of a function in a different shared object transfers
  control to the start of the PLT entry associated with the function.
\item The first part of the PLT entry loads the address from the GOT
  entry associated with the function to be called.  Control is
  transferred to the code referenced by the address.  If the function
  has already been called at least once, or if lazy binding is not used,
  then the address found in the GOT is the address of the function.
\item If a function has never been called and lazy binding is used,
  the address in the GOT points to the second half of the PLT\@.
  The second half loads the offset in the symbol table associated with
  the called function.  Control is then transferred to the special
  first entry of the PLT\@.
\item This first entry of the PLT entry (see \cref{lst:plt0ex})
  calls the dynamic linker, giving it the offset into the symbol table
  % FIXME: symbol table -> .rela.plt
  and the address of a structure that identifies the location of the
  caller.
\item The dynamic linker finds the real address of the symbol.  It
  will store this address in the GOT entry of the function in the
  object code of the caller and it will then transfer control to the
  function.
\item Subsequent calls to the function from this object will find the
  resolved address in the first half of the PLT entry and will
  transfer control directly without invoking the dynamic linker.
\end{enumerate}

\ifzseries
% FIXME bad 32-bit operations, should be 64-bit
\begin{lstlisting}[style=float,language=simpleasm,label=lst:plt0ex,
  caption=Special first entry in Procedure Linkage Table]
*                               # PLT0 for static object (not
                                #   position-independent)
PLT0      ST    1,28(15)        # R1 has offset into symbol table
          BASR  1,0             # Establish base
BASE1     L     1,AGOT-BASE1(1) # Get address of GOT
          MVC   24(4,15),4(1)   # Move loader info to stack
          L     1,8(1)          # Get address of loader
          BR    1               # Jump to loader
          .word 0               # Filler
AGOT      .long got             # Address of GOT

                                # PLT0 for shared object
                                #   (position-independent)
PLT0      STG   1,56(15)        # R1 has offset into symbol table
          LARL  1,_GLOBAL_OFFSET_TABLE_
          MVC   48(8,15),8(1)   # move loader info (object struct
                                #   address) to stack
          LG    1,16(12)        # Entry address of loader in R1
          BCR   15,1            # Jump to loader
\end{lstlisting}
\else
\begin{lstlisting}[style=float,language=simpleasm,label=lst:plt0ex,
  caption=Special first entry in Procedure Linkage Table]
*                               # PLT0 for static object (not
                                #   position-independent)
PLT0      ST    1,28(15)        # R1 has offset into symbol table
          BASR  1,0             # Establish base
BASE1     L     1,AGOT-BASE1(1) # Get address of GOT
          MVC   24(4,15),4(1)   # Move loader info to stack
          L     1,8(1)          # Get address of loader
          BR    1               # Jump to loader
          .word 0               # Filler
AGOT      .long got             # Address of GOT

                                # PLT0 for shared object
                                #   (position-independent)
PLT0      ST    1,28(15)        # R1 has offset into symbol table
          L     1,4(12)         # Get loader info (object struct
                                #   address)
          ST    1,24(15)        # Store address
          L     1,8(12)         # Entry address of loader in R1
          BR    1               # Jump to loader
\end{lstlisting}
\fi

The \texttt{LD\_BIND\_NOW} environment variable can change dynamic
linking behavior.  If set to a nonempty string, the dynamic linker
resolves the function call binding at load time, before transferring
control to the program.  In other words, the dynamic linker processes
relocation entries of type \texttt{R\_390\_JMP\_SLOT} during process
initialization.  If \texttt{LD\_BIND\_NOW} is not set, the dynamic
linker evaluates PLT entries lazily, delaying symbol resolution and
relocation until the first execution of a table entry.

\paragraph{Note:}
Lazy binding generally improves overall application performance
because unused symbols do not incur the overhead of dynamic
linking.  Nevertheless, two situations make lazy binding undesirable
for some applications:
\begin{enumerate}
\item The initial reference to a shared object function takes longer
  than subsequent calls because the dynamic linker intercepts the call
  to resolve the symbol, and some applications cannot tolerate this
  unpredictability.
\item If an error occurs and the dynamic linker cannot resolve the
  symbol, the dynamic linker will terminate the program.  Under lazy
  binding, this might occur at arbitrary times.  Once again, some
  applications cannot tolerate this unpredictability.  By turning off
  lazy binding, the dynamic linker forces the failure to occur during
  process initialization, before the application receives control.
\end{enumerate}

\appendix
% fdl.tex 
% This file is a chapter.  It must be included in a larger document to work
% properly.

\chapter{GNU Free Documentation License}

Version 1.1, March 2000\\

 Copyright \copyright\ 2000  Free Software Foundation, Inc.\\
     51 Franklin St, Fifth Floor, Boston, MA  02110-1301  USA\\
 Everyone is permitted to copy and distribute verbatim copies
 of this license document, but changing it is not allowed.

\section*{Preamble}

The purpose of this License is to make a manual, textbook, or other
written document ``free'' in the sense of freedom: to assure everyone
the effective freedom to copy and redistribute it, with or without
modifying it, either commercially or noncommercially.  Secondarily,
this License preserves for the author and publisher a way to get
credit for their work, while not being considered responsible for
modifications made by others.

This License is a kind of ``copyleft'', which means that derivative
works of the document must themselves be free in the same sense.  It
complements the GNU General Public License, which is a copyleft
license designed for free software.

We have designed this License in order to use it for manuals for free
software, because free software needs free documentation: a free
program should come with manuals providing the same freedoms that the
software does.  But this License is not limited to software manuals;
it can be used for any textual work, regardless of subject matter or
whether it is published as a printed book.  We recommend this License
principally for works whose purpose is instruction or reference.

\section{Applicability and Definitions}

This License applies to any manual or other work that contains a
notice placed by the copyright holder saying it can be distributed
under the terms of this License.  The ``Document'', below, refers to any
such manual or work.  Any member of the public is a licensee, and is
addressed as ``you''.

A ``Modified Version'' of the Document means any work containing the
Document or a portion of it, either copied verbatim, or with
modifications and/or translated into another language.

A ``Secondary Section'' is a named appendix or a front-matter section of
the Document that deals exclusively with the relationship of the
publishers or authors of the Document to the Document's overall subject
(or to related matters) and contains nothing that could fall directly
within that overall subject.  (For example, if the Document is in part a
textbook of mathematics, a Secondary Section may not explain any
mathematics.)  The relationship could be a matter of historical
connection with the subject or with related matters, or of legal,
commercial, philosophical, ethical or political position regarding
them.

The ``Invariant Sections'' are certain Secondary Sections whose titles
are designated, as being those of Invariant Sections, in the notice
that says that the Document is released under this License.

The ``Cover Texts'' are certain short passages of text that are listed,
as Front-Cover Texts or Back-Cover Texts, in the notice that says that
the Document is released under this License.

A ``Transparent'' copy of the Document means a machine-readable copy,
represented in a format whose specification is available to the
general public, whose contents can be viewed and edited directly and
straightforwardly with generic text editors or (for images composed of
pixels) generic paint programs or (for drawings) some widely available
drawing editor, and that is suitable for input to text formatters or
for automatic translation to a variety of formats suitable for input
to text formatters.  A copy made in an otherwise Transparent file
format whose markup has been designed to thwart or discourage
subsequent modification by readers is not Transparent.  A copy that is
not ``Transparent'' is called ``Opaque''.

Examples of suitable formats for Transparent copies include plain
ASCII without markup, Texinfo input format, \LaTeX~input format, SGML
or XML using a publicly available DTD, and standard-conforming simple
HTML designed for human modification.  Opaque formats include
PostScript, PDF, proprietary formats that can be read and edited only
by proprietary word processors, SGML or XML for which the DTD and/or
processing tools are not generally available, and the
machine-generated HTML produced by some word processors for output
purposes only.

The ``Title Page'' means, for a printed book, the title page itself,
plus such following pages as are needed to hold, legibly, the material
this License requires to appear in the title page.  For works in
formats which do not have any title page as such, ``Title Page'' means
the text near the most prominent appearance of the work's title,
preceding the beginning of the body of the text.


\section{Verbatim Copying}

You may copy and distribute the Document in any medium, either
commercially or noncommercially, provided that this License, the
copyright notices, and the license notice saying this License applies
to the Document are reproduced in all copies, and that you add no other
conditions whatsoever to those of this License.  You may not use
technical measures to obstruct or control the reading or further
copying of the copies you make or distribute.  However, you may accept
compensation in exchange for copies.  If you distribute a large enough
number of copies you must also follow the conditions in section~3.

You may also lend copies, under the same conditions stated above, and
you may publicly display copies.


\section{Copying in Quantity}

If you publish printed copies of the Document numbering more than 100,
and the Document's license notice requires Cover Texts, you must enclose
the copies in covers that carry, clearly and legibly, all these Cover
Texts: Front-Cover Texts on the front cover, and Back-Cover Texts on
the back cover.  Both covers must also clearly and legibly identify
you as the publisher of these copies.  The front cover must present
the full title with all words of the title equally prominent and
visible.  You may add other material on the covers in addition.
Copying with changes limited to the covers, as long as they preserve
the title of the Document and satisfy these conditions, can be treated
as verbatim copying in other respects.

If the required texts for either cover are too voluminous to fit
legibly, you should put the first ones listed (as many as fit
reasonably) on the actual cover, and continue the rest onto adjacent
pages.

If you publish or distribute Opaque copies of the Document numbering
more than 100, you must either include a machine-readable Transparent
copy along with each Opaque copy, or state in or with each Opaque copy
a publicly-accessible computer-network location containing a complete
Transparent copy of the Document, free of added material, which the
general network-using public has access to download anonymously at no
charge using public-standard network protocols.  If you use the latter
option, you must take reasonably prudent steps, when you begin
distribution of Opaque copies in quantity, to ensure that this
Transparent copy will remain thus accessible at the stated location
until at least one year after the last time you distribute an Opaque
copy (directly or through your agents or retailers) of that edition to
the public.

It is requested, but not required, that you contact the authors of the
Document well before redistributing any large number of copies, to give
them a chance to provide you with an updated version of the Document.


\section{Modifications}

You may copy and distribute a Modified Version of the Document under
the conditions of sections 2 and 3 above, provided that you release
the Modified Version under precisely this License, with the Modified
Version filling the role of the Document, thus licensing distribution
and modification of the Modified Version to whoever possesses a copy
of it.  In addition, you must do these things in the Modified Version:

\begin{itemize}

\item Use in the Title Page (and on the covers, if any) a title distinct
   from that of the Document, and from those of previous versions
   (which should, if there were any, be listed in the History section
   of the Document).  You may use the same title as a previous version
   if the original publisher of that version gives permission.
\item List on the Title Page, as authors, one or more persons or entities
   responsible for authorship of the modifications in the Modified
   Version, together with at least five of the principal authors of the
   Document (all of its principal authors, if it has less than five).
\item State on the Title page the name of the publisher of the
   Modified Version, as the publisher.
\item Preserve all the copyright notices of the Document.
\item Add an appropriate copyright notice for your modifications
   adjacent to the other copyright notices.
\item Include, immediately after the copyright notices, a license notice
   giving the public permission to use the Modified Version under the
   terms of this License, in the form shown in the Addendum below.
\item Preserve in that license notice the full lists of Invariant Sections
   and required Cover Texts given in the Document's license notice.
\item Include an unaltered copy of this License.
\item Preserve the section entitled ``History'', and its title, and add to
   it an item stating at least the title, year, new authors, and
   publisher of the Modified Version as given on the Title Page.  If
   there is no section entitled ``History'' in the Document, create one
   stating the title, year, authors, and publisher of the Document as
   given on its Title Page, then add an item describing the Modified
   Version as stated in the previous sentence.
\item Preserve the network location, if any, given in the Document for
   public access to a Transparent copy of the Document, and likewise
   the network locations given in the Document for previous versions
   it was based on.  These may be placed in the ``History'' section.
   You may omit a network location for a work that was published at
   least four years before the Document itself, or if the original
   publisher of the version it refers to gives permission.
\item In any section entitled ``Acknowledgements'' or ``Dedications'',
   preserve the section's title, and preserve in the section all the
   substance and tone of each of the contributor acknowledgements
   and/or dedications given therein.
\item Preserve all the Invariant Sections of the Document,
   unaltered in their text and in their titles.  Section numbers
   or the equivalent are not considered part of the section titles.
\item Delete any section entitled ``Endorsements''.  Such a section
   may not be included in the Modified Version.
\item Do not retitle any existing section as ``Endorsements''
   or to conflict in title with any Invariant Section.

\end{itemize}

If the Modified Version includes new front-matter sections or
appendices that qualify as Secondary Sections and contain no material
copied from the Document, you may at your option designate some or all
of these sections as invariant.  To do this, add their titles to the
list of Invariant Sections in the Modified Version's license notice.
These titles must be distinct from any other section titles.

You may add a section entitled ``Endorsements'', provided it contains
nothing but endorsements of your Modified Version by various
parties -- for example, statements of peer review or that the text has
been approved by an organization as the authoritative definition of a
standard.

You may add a passage of up to five words as a Front-Cover Text, and a
passage of up to 25 words as a Back-Cover Text, to the end of the list
of Cover Texts in the Modified Version.  Only one passage of
Front-Cover Text and one of Back-Cover Text may be added by (or
through arrangements made by) any one entity.  If the Document already
includes a cover text for the same cover, previously added by you or
by arrangement made by the same entity you are acting on behalf of,
you may not add another; but you may replace the old one, on explicit
permission from the previous publisher that added the old one.

The author(s) and publisher(s) of the Document do not by this License
give permission to use their names for publicity for or to assert or
imply endorsement of any Modified Version.


\section{Combining Documents}

You may combine the Document with other documents released under this
License, under the terms defined in section~4 above for modified
versions, provided that you include in the combination all of the
Invariant Sections of all of the original documents, unmodified, and
list them all as Invariant Sections of your combined work in its
license notice.

The combined work need only contain one copy of this License, and
multiple identical Invariant Sections may be replaced with a single
copy.  If there are multiple Invariant Sections with the same name but
different contents, make the title of each such section unique by
adding at the end of it, in parentheses, the name of the original
author or publisher of that section if known, or else a unique number.
Make the same adjustment to the section titles in the list of
Invariant Sections in the license notice of the combined work.

In the combination, you must combine any sections entitled ``History''
in the various original documents, forming one section entitled
``History''; likewise combine any sections entitled ``Acknowledgements'',
and any sections entitled ``Dedications''.  You must delete all sections
entitled ``Endorsements.''


\section{Collections of Documents}

You may make a collection consisting of the Document and other documents
released under this License, and replace the individual copies of this
License in the various documents with a single copy that is included in
the collection, provided that you follow the rules of this License for
verbatim copying of each of the documents in all other respects.

You may extract a single document from such a collection, and distribute
it individually under this License, provided you insert a copy of this
License into the extracted document, and follow this License in all
other respects regarding verbatim copying of that document.



\section{Aggregation With Independent Works}

A compilation of the Document or its derivatives with other separate
and independent documents or works, in or on a volume of a storage or
distribution medium, does not as a whole count as a Modified Version
of the Document, provided no compilation copyright is claimed for the
compilation.  Such a compilation is called an ``aggregate'', and this
License does not apply to the other self-contained works thus compiled
with the Document, on account of their being thus compiled, if they
are not themselves derivative works of the Document.

If the Cover Text requirement of section~3 is applicable to these
copies of the Document, then if the Document is less than one quarter
of the entire aggregate, the Document's Cover Texts may be placed on
covers that surround only the Document within the aggregate.
Otherwise they must appear on covers around the whole aggregate.


\section{Translation}

Translation is considered a kind of modification, so you may
distribute translations of the Document under the terms of section~4.
Replacing Invariant Sections with translations requires special
permission from their copyright holders, but you may include
translations of some or all Invariant Sections in addition to the
original versions of these Invariant Sections.  You may include a
translation of this License provided that you also include the
original English version of this License.  In case of a disagreement
between the translation and the original English version of this
License, the original English version will prevail.


\section{Termination}

You may not copy, modify, sublicense, or distribute the Document except
as expressly provided for under this License.  Any other attempt to
copy, modify, sublicense or distribute the Document is void, and will
automatically terminate your rights under this License.  However,
parties who have received copies, or rights, from you under this
License will not have their licenses terminated so long as such
parties remain in full compliance.


\section{Future Revisions of This License}

The Free Software Foundation may publish new, revised versions
of the GNU Free Documentation License from time to time.  Such new
versions will be similar in spirit to the present version, but may
differ in detail to address new problems or concerns. See
http://www.gnu.org/copyleft/.

Each version of the License is given a distinguishing version number.
If the Document specifies that a particular numbered version of this
License "or any later version" applies to it, you have the option of
following the terms and conditions either of that specified version or
of any later version that has been published (not as a draft) by the
Free Software Foundation.  If the Document does not specify a version
number of this License, you may choose any version ever published (not
as a draft) by the Free Software Foundation.

\section*{ADDENDUM: How to use this License for your documents}

To use this License in a document you have written, include a copy of
the License in the document and put the following copyright and
license notices just after the title page:

\begin{quote}

      Copyright \copyright\ YEAR  YOUR NAME.
      Permission is granted to copy, distribute and/or modify this document
      under the terms of the GNU Free Documentation License, Version 1.1
      or any later version published by the Free Software Foundation;
      with the Invariant Sections being LIST THEIR TITLES, with the
      Front-Cover Texts being LIST, and with the Back-Cover Texts being LIST.
      A copy of the license is included in the section entitled ``GNU
      Free Documentation License''.

\end{quote}

If you have no Invariant Sections, write ``with no Invariant Sections''
instead of saying which ones are invariant.  If you have no
Front-Cover Texts, write ``no Front-Cover Texts'' instead of
``Front-Cover Texts being LIST''; likewise for Back-Cover Texts.

If your document contains nontrivial examples of program code, we
recommend releasing these examples in parallel under your choice of
free software license, such as the GNU General Public License,
to permit their use in free software.



\chapter{Notices}
% Legal notices for IBM-provided documentation about IBM products.  Should
% be included as a separate chapter.

This information was developed for products and services offered in the
US\@.  This material might be available from IBM in other languages.
However, you may be required to own a copy of the product or product
version in that language in order to access it.

IBM may not offer the products, services, or features discussed in this
document in other countries.  Consult your local IBM representative for
information on the products and services currently available in your area.
Any reference to an IBM product, program, or service is not intended to
state or imply that only that IBM product, program, or service may be
used.  Any functionally equivalent product, program, or service that does
not infringe any IBM intellectual property right may be used instead.
However, it is the user's responsibility to evaluate and verify the
operation of any non-IBM product, program, or service.

IBM may have patents or pending patent applications covering subject
matter described in this document.  The furnishing of this document does
not grant you any license to these patents.  You can send license
inquiries, in writing, to:
\begin{quote}
  IBM Director of Licensing\\
  IBM Corporation\\
  North Castle Drive, MD-NC119\\
  Armonk, NY 10504-1785\\
  US
\end{quote}

For license inquiries regarding double-byte character set (DBCS)
information, contact the IBM Intellectual Property Department in your
country or send inquiries, in writing, to:
\begin{quote}
  Intellectual Property Licensing\\
  Legal and Intellectual Property Law\\
  IBM Japan Ltd.\\
  19-21, Nihonbashi-Hakozakicho, Chuo-ku\\
  Tokyo 103-8510, Japan
\end{quote}

INTERNATIONAL BUSINESS MACHINES CORPORATION PROVIDES THIS PUBLICATION ``AS
IS'' WITHOUT WARRANTY OF ANY KIND, EITHER EXPRESS OR IMPLIED, INCLUDING,
BUT NOT LIMITED TO, THE IMPLIED WARRANTIES OF NON-INFRINGEMENT,
MERCHANTABILITY OR FITNESS FOR A PARTICULAR PURPOSE\@.  Some jurisdictions
do not allow disclaimer of express or implied warranties in certain
transactions, therefore, this statement may not apply to you.

This information could include technical inaccuracies or typographical
errors.  Changes are periodically made to the information herein; these
changes will be incorporated in new editions of the publication.  IBM may
make improvements and/or changes in the product(s) and/or the program(s)
described in this publication at any time without notice.

Any references in this information to non-IBM websites are provided for
convenience only and do not in any manner serve as an endorsement of those
websites.  The materials at those websites are not part of the materials
for this IBM product and use of those websites is at your own risk.

IBM may use or distribute any of the information you provide in any way it
believes appropriate without incurring any obligation to you.

Licensees of this program who wish to have information about it for the
purpose of enabling: (i)~the exchange of information between independently
created programs and other programs (including this one) and (ii)~the
mutual use of the information which has been exchanged, should contact:
\begin{quote}
  IBM Director of Licensing\\
  IBM Corporation\\
  North Castle Drive, MD-NC119\\
  Armonk, NY 10504-1785\\
  US
\end{quote}

Such information may be available, subject to appropriate terms and
conditions, including in some cases, payment of a fee.

The licensed program described in this document and all licensed material
available for it are provided by IBM under terms of the IBM Customer
Agreement, IBM International Program License Agreement or any equivalent
agreement between us.

The performance data discussed herein is presented as derived under
specific operating conditions.  Actual results may vary.

The client examples cited are presented for illustrative purposes only.
Actual performance results may vary depending on specific configurations
and operating conditions.

The performance data and client examples cited are presented for
illustrative purposes only.  Actual performance results may vary depending
on specific configurations and operating conditions.

Information concerning non-IBM products was obtained from the suppliers of
those products, their published announcements or other publicly available
sources.  IBM has not tested those products and cannot confirm the
accuracy of performance, compatibility or any other claims related to
non-IBM products.  Questions on the capabilities of non-IBM products
should be addressed to the suppliers of those products.

Statements regarding IBM's future direction or intent are subject to
change or withdrawal without notice, and represent goals and objectives
only.

All IBM prices shown are IBM's suggested retail prices, are current and
are subject to change without notice.  Dealer prices may vary.

This information is for planning purposes only.  The information herein is
subject to change before the products described become available.

This information contains examples of data and reports used in daily
business operations.  To illustrate them as completely as possible, the
examples include the names of individuals, companies, brands, and
products.  All of these names are fictitious and any similarity to actual
people or business enterprises is entirely coincidental.

\medskip
COPYRIGHT LICENSE:\\\nopagebreak
This information contains sample application programs in source language,
which illustrate programming techniques on various operating platforms.
You may copy, modify, and distribute these sample programs in any form
without payment to IBM, for the purposes of developing, using, marketing
or distributing application programs conforming to the application
programming interface for the operating platform for which the sample
programs are written.  These examples have not been thoroughly tested
under all conditions.  IBM, therefore, cannot guarantee or imply
reliability, serviceability, or function of these programs.  The sample
programs are provided "AS IS", without warranty of any kind.  IBM shall
not be liable for any damages arising out of your use of the sample
programs.

Each copy or any portion of these sample programs or any derivative work
must include a copyright notice as follows:

\begin{quote}
  © (your company name) (year).\\
  Portions of this code are derived from IBM Corp. Sample Programs.\\
  © Copyright IBM Corp. 2001, 2021.
\end{quote}

\section{Trademarks}

IBM, the IBM logo, and \texttt{ibm.com} are trademarks or registered
trademarks of International Business Machines Corp., registered in many
jurisdictions worldwide. Other product and service names might be
trademarks of IBM or other companies. A current list of IBM trademarks is
available on the web at "Copyright and trademark information" at
\url{www.ibm.com/legal/copytrade.shtml}.

The registered trademark Linux\textregistered{} is used pursuant to a
sublicense from the Linux Foundation, the exclusive licensee of Linus
Torvalds, owner of the mark on a world­wide basis.



\newcommand{\bibTitle}[1]{``#1''}

\begin{thebibliography}{9}
\bibitem{tlshandling}
  Ulrich Drepper,
  \bibTitle{ELF Handling For Thread-Local Storage,}
  \url{https://akkadia.org/drepper/tls.pdf};
  2013
\bibitem{sysvabi}
  \bibTitle{System V Application Binary Interface,}
  edition 4.1,
  \url{http://www.sco.com/developers/devspecs/gabi41.pdf};
  1997
\bibitem{sysvabidraft}
  \bibTitle{System V Application Binary Interface,}
  chapters 4 and 5, latest snapshot,
  \url{http://www.sco.com/developers/gabi/latest/contents.html};
  2013
\bibitem{gnu-vec}
  \bibTitle{Using the GNU Compiler Collection,}
  vector extensions,
  \url{https://gcc.gnu.org/onlinedocs/gcc/Vector-Extensions.html}
\bibitem{sa22}
  \bibTitle{\ARCH{} Principles of Operation,}
  IBM Publication No.~SA22-{\ifzseries 7832-12\else 7201-08\fi};
  2019
\end{thebibliography}

\printindex

\end{document}
